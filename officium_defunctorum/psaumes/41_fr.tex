\item Comme un cerf altéré cherche l'eau vive, * ainsi mon âme te cherche toi, mon Dieu.

\item Mon âme a soif de Dieu, le Dieu vivant ; * quand pourrai-je m'avancer, paraître face à Dieu ?

\item Je n'ai d'autre pain que mes larmes, le jour, la nuit, * moi qui chaque jour entends dire : « Où est-il ton Dieu ? »

\item Je me souviens, et mon âme déborde : * en ce temps-là, je franchissais les portails ! 

\item Je conduisais vers la maison de mon Dieu la multitude en fête, * parmi les cris de joie et les actions de grâce.

\item Pourquoi te désoler, ô mon âme, et gémir sur moi ? * Espère en Dieu ! De nouveau je rendrai grâce : il est mon sauveur et mon Dieu !

\item Si mon âme se désole, je me souviens de toi, * depuis les terres du Jourdain et de l'Hermon, depuis mon humble montagne.

\item L'abîme appelant l'abîme à la voix de tes cataractes, * la masse de tes flots et de tes vagues a passé sur moi.

\item Au long du jour, le Seigneur m'envoie son amour ; * et la nuit, son chant est avec moi, prière au Dieu de ma vie.

\item Je dirai à Dieu, mon rocher : « Pourquoi m'oublies-tu ? * Pourquoi vais-je assombri, pressé par l'ennemi ? »

\item Outragé par mes adversaires, je suis meurtri jusqu'aux os, * moi qui chaque jour entends dire : « Où est-il ton Dieu ? »

\item Pourquoi te désoler, ô mon âme, et gémir sur moi ? * Espère en Dieu ! De nouveau je rendrai grâce : il est mon sauveur et mon Dieu !

\item Seigneur, donne-leur le repos éternel, * et fais luire pour eux la lumière sans déclin.