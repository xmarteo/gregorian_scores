\item Le Seigneur est ma lumière et mon salut ; * de qui aurais-je crainte ?

\item Le Seigneur est le rempart de ma vie ; * devant qui tremblerais-je ?

\item Si des méchants s'avancent contre moi * pour me déchirer,

\item Ce sont eux, mes ennemis, mes adversaires, * qui perdent pied et succombent.

\item Qu'une armée se déploie devant moi, * mon cœur est sans crainte ;

\item Que la bataille s'engage contre moi, * je garde confiance.

\item J'ai demandé une chose au Seigneur, * la seule que je cherche :

\item Habiter la maison du Seigneur * tous les jours de ma vie,

\item Pour admirer le Seigneur dans sa beauté * et m'attacher à son temple.

\item Oui, il me réserve un lieu sûr * au jour du malheur ;

\item il me cache au plus secret de sa tente, * il m'élève sur le roc.

\item Maintenant je relève la tête * devant mes ennemis.

\item J'irai célébrer dans sa tente le sacrifice d'ovation ; * je chanterai, je fêterai le Seigneur.

\item Écoute, Seigneur, je t'appelle ! * Pitié ! Réponds-moi !

\item Mon cœur m'a redit ta parole : * « Cherchez ma face. »

\item C'est ta face, Seigneur, que je cherche : * ne me cache pas ta face.

\item N'écarte pas ton serviteur avec colère : * tu restes mon secours.

\item Ne me laisse pas, ne m'abandonne pas, * Dieu, mon salut !

\item Mon père et ma mère m'abandonnent ; * le Seigneur me reçoit.

\item Enseigne-moi ton chemin, Seigneur, * conduis-moi par des routes sûres,

\item Malgré ceux qui me guettent. * Ne me livre pas à la merci de l'adversaire :

\item Contre moi se sont levés de faux témoins * qui soufflent la violence.

\item Mais j'en suis sûr, je verrai les bontés du Seigneur * sur la terre des vivants. *

\item « Espère le Seigneur, sois fort et prends courage ; * espère le Seigneur. »

\item Seigneur, donne-leur le repos éternel, * et fais luire pour eux la lumière sans déclin.