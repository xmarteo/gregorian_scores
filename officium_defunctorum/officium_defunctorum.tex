% !TEX TS-program = lualatex
% !TEX encoding = UTF-8

\documentclass[twoside]{book}

%%%%%%%%%%%%%%% STANDARD PACKAGES %%%%%%%%%%%%%%%

\usepackage[paperwidth=148mm, paperheight=210mm]{geometry}

\usepackage{fontspec}
\usepackage[medievallatin, french]{babel}
\usepackage{fancyhdr}
\usepackage{paracol}
\usepackage{expl3}
\usepackage{needspace}
\usepackage{etoolbox}
\usepackage{tableof}
\usepackage{setspace}
\usepackage{alltt}
\usepackage{titlesec}
\usepackage{xcolor}
\usepackage{xstring}
\usepackage{enumitem}
\usepackage{hyperref}
\usepackage{refcount}
%%%%%%%%%%%%%%% GEOMETRY %%%%%%%%%%%%%%%

%% This should mimic the layout of the recent Solesmes books.
\geometry{
inner=15mm,
outer=10mm,
top=15mm,
bottom=15mm,
headsep=3mm,
}

%% General scale of all graphical elements.
%% Values different from 1 are largely untested.
%% Used in those commands (e.g. everything FontSpec) that use a scale parameter.
\newcommand{\customscale}{1}

%% Provide the command \fpeval as a copy of the code-level \fp_eval:n.
%% \fpeval allows to evaluate floating point calculation for scaled parameters, e.g. \setSomeStretchFactor{\fpeval{0,6 * \customscale}}
\ExplSyntaxOn
\cs_new_eq:NN \fpeval \fp_eval:n
\ExplSyntaxOff

%% No indentation of paragraphs
\setlength{\parindent}{0mm}

%% We want to allow large inter-words space 
%% to avoid overfull boxes in two-columns rubrics.
\sloppy

%%%%%%%%%%%%%%% GREGORIO CONFIG %%%%%%%%%%%%%%%

\usepackage[autocompile]{gregoriotex}

%% disable NABC a the request of A. Guyard who will add SG neumes by hand
\gresetnabc{1}{invisible} 

%% text above lines shall be of color gregoriocolor
\grechangestyle{abovelinestext}{\color{gregoriocolor}\footnotesize}
%% fine-tuning of space beween the staff and the text above lines
\newcommand{\altraise}{-2.4mm} %% default is -0.1cm
\grechangedim{abovelinestextraise}{\altraise}{scalable}

%% fine-tuning of space between the staff and the lyrics
\newcommand{\textraise}{2.8ex} %% default is 3.48471 ex
\grechangedim{spacelinestext}{\textraise}{scalable}

%% fine-tuning of space between the initial and the annotations
\newcommand{\annraise}{0mm} %% default is -0.2mm
\grechangedim{annotationraise}{\annraise}{scalable}

%% \officepartannotation converts a letter (IHARPT) into the office part to be printed as annotation,
%% storing the result into \result.

\newcommand{\result}{}
\newcommand{\lookup}[3]{%
  \IfSubStr{#2}{#1}{ \renewcommand{\result}{#3} }{}%
}%
\newcommand{\officepartannotation}[1]{%
  \renewcommand{\result}{#1}%
  \lookup{#1}{T}{}%
  \lookup{#1}{H}{Hymn.}%
  \lookup{#1}{A}{Ant.}%
  \lookup{#1}{P}{}%
  \lookup{#1}{R}{Resp.}%
  \lookup{#1}{I}{Invit.}%
  \lookup{#1}{M}{Ad Magn.}%
  \lookup{#1}{B}{Ad Ben.}%
  \result%
}%

%% header capture setup for the mode
\gresetheadercapture{mode}{greannotation}{string}

%% outputs a score with no initials or annotations
%% for 
\newcommand{\smallscore}[1]{
  \gresetinitiallines{0}
  %% the use of a directory called "gabc" is linked
  %% to the management of gabc files by the website: do not change 
  %% without also changing the website static files structure
  \gregorioscore{partitions/#1}
  \gresetinitiallines{1}
}

%% outputs a score with annotations. no initials if [n] is passed
\newcommand{\gscore}[3]{
  %% #1 (passed as option) : y = initial, n = no initial
  %% #2 : name of the score file, should be a code, e.g. Q4F4A3 or 1225N1R1
  %% #4 : if applicable, a number between 1 and 9 (rank of the ant./resp.) - else: empty
  
  %% this prevents orphans
  \needspace{4\baselineskip} 
  \greannotation{\officepartannotation{#2}#3}
  \gregorioscore{partitions/#1}
}

%%%%%%%%%%%%%%% FONTS %%%%%%%%%%%%%%%

%%%%%%%%%%%%%%% Main font
\setmainfont[Ligatures=TeX, Scale=\customscale]{Charis SIL}
%\setmainfont[Ligatures=TeX, Scale=\customscale]{TeXGyreBonum-Regular}
\setstretch{\fpeval{1.05 * \customscale}}

%%%%%%%%%%%%%%% Score initials
%% \initialsize resizes the initials, with one argument (size in points)
\newcommand{\initialsize}[1]{
    \grechangestyle{initial}{\fontspec{Zallman}\fontsize{#1}{#1}\selectfont}
}
%% default initial size is 32 points
\newcommand{\defaultinitialsize}{32}
\initialsize{\defaultinitialsize}

%% spacing before and after initials to kern the Zallman Caps.
%% this should be changed if we move away from Zallman Caps.
\newcommand{\initialspace}[1]{
  \grechangedim{afterinitialshift}{#1}{scalable}
  \grechangedim{beforeinitialshift}{#1}{scalable}
}
%% default space before and after initials is 0cm.
\newcommand{\defaultinitialspace}{0.9mm}
\initialspace{\defaultinitialspace}

%%%%%%%%%%%%%%% Score annotations
\grechangestyle{annotation}{\small}

%%%%%%%%%%%%%%% GRAPHICAL ELEMENTS %%%%%%%%%%%%%%%

%% V/, R/, A/ and + signs for in-line use (\vv \rr \aa \cc) and in-score use (<sp>V/ R/ A/ +</sp>)
\newcommand{\specialcharhsep}{3mm} % space after invoking R/ or V/ or A/
\newcommand{\vv}{\textcolor{gregoriocolor}{\fontspec[Scale=\customscale]{Charis SIL}℣.\hspace{\specialcharhsep}}}
\newcommand{\rr}{\textcolor{gregoriocolor}{\fontspec[Scale=\customscale]{Charis SIL}℟.\hspace{\specialcharhsep}}}
\renewcommand{\aa}{\textcolor{gregoriocolor}{\fontspec[Scale=\customscale]{Charis SIL}\Abar.\hspace{\specialcharhsep}}}
\newcommand{\cc}{\textcolor{gregoriocolor}{\fontspec[Scale=\customscale]{FreeSerif}\symbol{"2720}~}}
\gresetspecial{V/}{\textcolor{gregoriocolor}{\fontspec[Scale=\customscale]{Charis SIL}℣.~}}
\gresetspecial{R/}{\textcolor{gregoriocolor}{\fontspec[Scale=\customscale]{Charis SIL}℟.~}}
\gresetspecial{A/}{\textcolor{gregoriocolor}{\fontspec[Scale=\customscale]{Charis SIL}\Abar.~}}
\gresetspecial{+}{\textcolor{gregoriocolor}{\fontspec[Scale=\customscale]{FreeSerif}†~}}

%% Roman Numerals
\usepackage{modroman}
\newcommand{\Rnum}[1]{\nbRoman{#1}}
\newcommand{\rnum}[1]{\nbshortroman{#1}}

%% Macro to print versicles
\newcommand{\versiculus}[2]{\rr #1 \\ \vv #2}

\newcommand{\versiculustpall}[2]
	{\versiculus{#1 \rubric{(T.P.} Allelúja. \rubric{)}}{#2 \rubric{(T.P.} Allelúja. \rubric{)}}}

%% Macro to print rubrics
\newcommand{\rubric}[1]{\textcolor{gregoriocolor}{\emph{#1}}}

%% Macro to print the name of a score in normal characters inside a \rubric
\newcommand{\normaltext}[1]{{\normalfont\normalcolor #1}}
\newcommand{\scorename}[1]{\normaltext{\nameref{#1}}}

%% Macro to print the common rubric that signals the Te Deum
\newcommand{\tedeumrubric}{\rubric{Lectione ultima peracta Hymnus \normaltext{Te Deum} cantatur.}}

%% Macro to print translations
\newcommand{\trans}[1]{	\emph{#1}}

%%%%%%%%%%%%%%% COLUMN MANAGEMENT %%%%%%%%%%%%%%%

\usepackage{multicol}
\usepackage{parcolumns}
\setlength\columnseprule{0.4pt}

%% Macros to print a psalm on two columns.
%% First, without title and incipitur

\newcommand{\psalmtext}[1]{
	\begin{parcolumns}[rulebetween]{2}%
	\colchunk{%
		\vspace{-\baselineskip}
		\begin{itemize}[%
			label=\null, %
			leftmargin=0pt, %
			itemindent=3mm, %
			labelsep=0pt, %
			labelwidth=0pt, %
			rightmargin=0pt, %
			parsep=0pt, %
			topsep=0pt, %
			itemsep=0pt]%
		\input{psaumes/#1_la.tex}%
		\end{itemize}%
	}%
	\colchunk{%
		\vspace{-\baselineskip}
		\begin{itemize}[%
			label=\null, %
			leftmargin=0pt, %
			itemindent=3mm, %
			labelsep=0pt, %
			labelwidth=0pt, %
			rightmargin=0pt, %
			parsep=0pt, %
			topsep=0pt, %
			itemsep=0pt]%
		\input{psaumes/#1_fr.tex} %
		\end{itemize} %
	}%
	\end{parcolumns}
}

%% then, adding a title and incipitur
\newcommand{\psalmus}[2]{
	\needspace{4\baselineskip}
	\smalltitle{Psaume #1}
	\smallscore{#2}
	~\\
	\psalmtext{#1}
}

%% Macro to print lessons on two columns.

\newcommand{\lesson}[1]{
	\begin{parcolumns}[rulebetween]{2}%
	\colchunk{%
		\input{lecons/#1_la.tex}%
	}%
	\colchunk{%
		\input{lecons/#1_fr.tex} %
	}%
	\end{parcolumns}
}

%%%%%%%%%%%%%%% HEADER STYLES %%%%%%%%%%%%%%%

\pagestyle{fancy}
\fancyhead{}
\fancyfoot{}
\renewcommand{\headrulewidth}{0pt}
\setlength{\headheight}{20pt}
\fancyhead[RO]{\small\rightmark\hspace{1cm}\thepage}
\fancyhead[LE]{\small\thepage\hspace{1cm}\leftmark}

\newcommand{\setheaders}[2]{
	\renewcommand{\rightmark}{{\sc#2}}
	\renewcommand{\leftmark}{{\sc#1}}
}
\setheaders{}{}

%%%%%%%%%%%%%%% TITLE STYLES %%%%%%%%%%%%%%%

%% Titles are centered and small-caps
\titleformat{\chapter}[block]{\Large\filcenter\sc}{}{}{}
\titleformat{\section}[block]{\large\filcenter\sc}{}{}{}
\titleformat{\subsection}[block]{\filcenter\sc}{}{}{}
\setcounter{secnumdepth}{0}
%% Fine-tuning of space around titles
\titlespacing*{\paragraph}{0pt}{1ex}{.6ex}


\newcommand{\officiumtitulum}[1]{
  \newpage
  \begin{center}
  {\scshape\LARGE #1}\par
  \end{center}
}

\newcommand{\smalltitle}[1]{
 ~\\
 {\centering\scshape #1\par}
}

\newcommand{\nocturnumtitulum}[1]{
  %% needspace: should be barely more than the vertical space for the titles, rubrics excluded.
  %% this is to ensure that the page does not get cut after the title
  \needspace{6\baselineskip}
  \begin{center}
  {\scshape\Large #1}\par
  \end{center}
}

\begin{document}

% ceci est pour conserver une numérotation ordinaire malgré paracol
\twosided[pb]

\begin{titlepage}
\centering\null

\vspace{1cm}

{\scshape\LARGE Officium Defunctorum}

\vspace{2cm}
{\scshape\Large juxta usum antiquior rituum romanum}

\vspace{5cm}

{\scshape\LARGE L'Office des Morts}

\vspace{2cm}
{\scshape\Large selon l'usage ancien du rite romain}


\end{titlepage}

\thispagestyle{empty}\null\newpage

% tolérance infinie sur les sauts de lignes pour les colonnes étroites
\sloppy

\officiumtitulum{à Vêpres}

\rubric{L'office débute par la première antienne, sans aucune introduction. Il n'y a ni hymne ni capitule.}

\gscore{VA1}{A}{1}
\trans{\aa Je plairai au Seigneur dans la terre des vivants.}
\psalmus{114}{VP1}
\smallscore{VA1}

\gscore{VA2}{A}{2}
\trans{\aa Malheur à moi car mon exil s'est prolongé.}
\psalmus{119}{VP2}
\smallscore{VA2}

\gscore{VA3}{A}{3}
\trans{\aa Le Seigneur te garde de tout mal, que le Seigneur garde ton âme.}
\psalmus{120}{VP3}
\smallscore{VA3}

\gscore{VA4}{A}{4}
\trans{\aa Si Tu retiens les fautes, Seigneur, Seigneur, qui subsistera ?}
\psalmus{129}{VP4}
\smallscore{VA4}

\gscore{VA5}{A}{5}
\trans{\aa Ne méprise pas, Seigneur, les œuvres de Tes mains.}
\psalmus{137}{VP5}
\smallscore{VA5}

\smalltitle{Versicule}
\smallscore{OR_audivi}
\trans{\vv J'entendis une voix du ciel qui me disait : \\ \rr Bienheureux les morts qui meurent dans le Seigneur.}

\gscore{VAM}{M}{}
\trans{\aa Tout ce que le Père Me donne viendra à Moi ; et celui qui vient à Moi, Je ne le jetterai pas dehors.}

\smalltitle{Cantique de Marie}
\smallscore{VPM}
\psalmtext{m}
\smallscore{VAM}

\newpage

\smalltitle{Conclusion}
\label{conclusion}

\rubric{Sauf mention contraire, toutes les réponses se disent sur ce ton :}

\smallscore{OR_tonusversus}
~\\
\begin{parcolumns}[rulebetween]{2}%
	\colchunk{%
		\vv Pater noster...\\
		\rubric{(en silence jusqu'à :)}\\
		\vv Et ne nos indúcas in tentatiónem: \\
		\rr Sed líbera nos a malo.\\
		\vv A porta ínferi.\\
		\rr Erue, Dómine, ánimam ejus [ánimas eórum].\\
		\vv Requiéscat in pace. [Requiéscant in pace.]\\
		\rr Amen.\\
		\vv Dómine, exáudi oratiónem meam.\\
		\rr Et clamor meus ad te véniat.\\
		\vv Orémus.
	}
	\colchunk{%
		\vv Notre Père...\\
		\vv Et ne nous laisse pas entrer en tentation.\\
		\rr Mais délivre-nous du mal.\\
		\vv De la puissance de l'enfer.\\
		\rr Délivre, Seigneur, son âme [leurs âmes].\\
		\vv Qu'il repose en paix. [Qu'ils reposent en paix.]\\
		\rr Amen.\\
		\vv Seigneur, entends ma prière.\\
		\rr Et que mon cri parvienne jusqu'à Toi.\\
		\vv Prions.
	}
\end{parcolumns}

\rubric{On dit l'oraison qui correspond à l'occasion, pages suivantes, et on répond \normaltext{Amen.} Puis, on dit, toujours au pluriel :}

\begin{parcolumns}[rulebetween]{2}%
	\colchunk{%
		\vv Réquiem ætérnam dona eis, Dómine.\\
		\rr Et lux perpétua lúceat eis.
	}
	\colchunk{%
		\vv Seigneur, donne-leur le repos éternel.\\
		\rr Et fais luire pour eux la lumière sans déclin.
	}
\end{parcolumns}

\smallscore{OR_requiescant}
\trans{\vv Qu'ils reposent en paix.\\ \rr Amen.}

\officiumtitulum{Oraisons}

{\centering \rubric{À dire à la fin des offices, en fonction de diverses circonstances.}\par}
\label{oraisons}

\smalltitle{devant le corps du défunt}

\begin{parcolumns}[rulebetween]{2}%
	\colchunk{%

	}
	\colchunk{%

	}
\end{parcolumns}

\smalltitle{le jour des funérailles}

\begin{parcolumns}[rulebetween]{2}%
	\colchunk{%
ici latin
	}
	\colchunk{%
ici francais
	}
\end{parcolumns}

\smalltitle{le troisième, septième, trentième jour après la mort}

\begin{parcolumns}[rulebetween]{2}%
	\colchunk{%
ici latin
	}
	\colchunk{%
ici francais
	}
\end{parcolumns}

\smalltitle{le jour anniversaire de la mort}

\begin{parcolumns}[rulebetween]{2}%
	\colchunk{%
etc
	}
	\colchunk{%
etc
	}
\end{parcolumns}

\smalltitle{pour le Pape}

\begin{parcolumns}[rulebetween]{2}%
	\colchunk{%

	}
	\colchunk{%

	}
\end{parcolumns}

\smalltitle{pour un évêque}

\begin{parcolumns}[rulebetween]{2}%
	\colchunk{%

	}
	\colchunk{%

	}
\end{parcolumns}

\smalltitle{pour un prêtre}

\begin{parcolumns}[rulebetween]{2}%
	\colchunk{%

	}
	\colchunk{%

	}
\end{parcolumns}

\smalltitle{pour un homme}

\begin{parcolumns}[rulebetween]{2}%
	\colchunk{%

	}
	\colchunk{%

	}
\end{parcolumns}

\smalltitle{pour une femme}

\begin{parcolumns}[rulebetween]{2}%
	\colchunk{%

	}
	\colchunk{%

	}
\end{parcolumns}

\smalltitle{pour un ami ou un bienfaiteur}

\begin{parcolumns}[rulebetween]{2}%
	\colchunk{%

	}
	\colchunk{%

	}
\end{parcolumns}
	
\smalltitle{pour un père et une mère}

\begin{parcolumns}[rulebetween]{2}%
	\colchunk{%

	}
	\colchunk{%

	}
\end{parcolumns}
	
\smalltitle{pour plusieurs défunts}

\begin{parcolumns}[rulebetween]{2}%
	\colchunk{%

	}
	\colchunk{%

	}
\end{parcolumns}

\smalltitle{à la commémoraison de tous les fidèles défunts}

\begin{parcolumns}[rulebetween]{2}%
	\colchunk{%

	}
	\colchunk{%

	}
\end{parcolumns}

\smalltitle{à l'office des défunts pendant l'année}

\begin{parcolumns}[rulebetween]{2}%
	\colchunk{%

	}
	\colchunk{%

	}
\end{parcolumns}

\officiumtitulum{à Matines}

\rubric{L'office débute par l'invitatoire, sans aucune introduction. Il n'y a pas d'hymne.  Les leçons sont chantées sans bénédiction ni absolution. On chante trois nocturnes le jour de la mort d'un défunt, le jour de sa sépulture, les troisième, septième, trentième jours après sa mort, et le jour anniversaire, et en toute autre occasion légitime. On chante un seul nocturne les autres jours, comme précisé ci-dessous.}

\smalltitle{Invitatoire}
\gscore{MI}{I}{}

\nocturnumtitulum{Premier Nocturne}
\rubric{Si l'on chante un seul nocturne, ce nocturne se chante les dimanche, lundi et jeudi.}

\gscore{MA1}{A}{1}
\trans{\aa Redresse, Seigneur mon Dieu, Ton chemin devant moi.}
\psalmus{5}{MP1}
\smallscore{MA1}


\gscore{MA2}{A}{2}
\trans{\aa Reviens, Seigneur, délivre mon âme : car, dans la mort, nul souvenir de Toi.}
\psalmus{6}{MP2}
\smallscore{MA2}

\gscore{MA3}{A}{3}
\trans{\aa Qu'ils ne m'égorgent pas, tous ces fauves, ne me déchirent pas, sans que personne me délivre.}
\psalmus{7}{MP3}
\smallscore{MA3}

\smalltitle{Versicule}
\smallscore{MV1}
\trans{\vv De la puissance de l'enfer.\\ \rr Délivrez, Seigneur, leurs âmes.}

\rubric{On dit un \normaltext{Pater noster} entièrement en silence.}

\smalltitle{Première leçon}

\lesson{1}

\smalltitle{Deuxième leçon}

\lesson{2}

\smalltitle{Troisième leçon}

\lesson{3}

\nocturnumtitulum{Deuxième Nocturne}
\rubric{Si l'on chante un seul nocturne, ce nocturne se chante les mardi et vendredi.}

\gscore{MA4}{A}{4}
\trans{\aa Sur des prés d'herbe fraîche, Il me fait reposer.}
\psalmus{22}{MP4}
\smallscore{MA4}

\gscore{MA5}{A}{5}
\trans{\aa Oublie les péchés de ma jeunesse, ne Te rappelle pas mes erreurs, Seigneur.}
\psalmus{24}{MP5}
\smallscore{MA5}

\gscore{MA6}{A}{6}
\trans{\aa J'en suis sûr, je verrai les bontés du Seigneur sur la terre des vivants.}
\psalmus{26}{MP6}
\smallscore{MA6}

\smalltitle{Versicule}
\smallscore{MV2}
\trans{\vv Le Seigneur les établira avec les princes. \\ \rr Avec les princes de son peuple.}

\rubric{On dit un \normaltext{Pater noster} entièrement en silence.}

\smalltitle{Quatrième leçon}

\lesson{4}

\smalltitle{Cinquième leçon}

\lesson{5}

\smalltitle{Sixième leçon}

\lesson{6}

\nocturnumtitulum{Troisième Nocturne}
\rubric{Si l'on chante un seul nocturne, ce nocturne se chante les mercredi et samedi.}

\gscore{MA7}{A}{7}
\trans{\aa Daigne, Seigneur, me délivrer ; Seigneur, viens vite à mon secours !}
\psalmus{39}{MP7}
\smallscore{MA7}

\gscore{MA8}{A}{8}
\trans{\aa Pitié pour moi, Seigneur, guéris-moi, car j'ai péché contre Toi.}
\psalmus{40}{MP8}
\smallscore{MA8}

\gscore{MA9}{A}{9}
\trans{\aa Mon âme a soif du Dieu vivant ; quand pourrai-je m'avancer, paraître face à Dieu ?}
\psalmus{41}{MP9}
\smallscore{MA9}

\smalltitle{Versicule}
\smallscore{MV3}
\trans{\vv Ne livre pas aux bêtes les âmes qui Te louent.\\ \rr Et les âmes de Tes pauvres, ne les oublie pas à jamais.}

\rubric{On dit un \normaltext{Pater noster} entièrement en silence.}

\smalltitle{Septième leçon}

\lesson{7}

\smalltitle{Huitième leçon}

\lesson{8}

\smalltitle{Neuvième leçon}

\lesson{9}

\rubric{Si on ne joint pas Matines et Laudes, on conclut par un \normaltext{Pater noster}, et le reste, comme à Vêpres, page~\pageref{conclusion}.}

\officiumtitulum{à Laudes}

\rubric{L'office débute par la première antienne, sans aucune introduction. Il n'y a ni hymne ni capitule.}

\gscore{LA1}{A}{1}
\trans{\aa Les os humiliés exulteront dans le Seigneur.}
\psalmus{50}{LP1}
\smallscore{LA1}

\gscore{LA2}{A}{2}
\trans{\aa Seigneur, exauce ma prière ; toute chair viendra à Toi.}
\psalmus{64}{LP2}
\smallscore{LA2}

\gscore{LA3}{A}{3}
\trans{\aa Ta main droite me soutient, Seigneur.}
\psalmus{62}{LP3}
\smallscore{LA3}

\gscore{LA4}{A}{4}
\trans{\aa De la puissance de l'enfer délivre mon âme, Seigneur.}
{\centering\scshape Cantique d'Ézéchias\par}
\smallscore{LP4}
\psalmtext{ez}
\smallscore{LA4}

\gscore{LA5}{A}{5}
\trans{\aa Que tout être vivant chante louange au Seigneur.}
\psalmus{150}{LP5}
\smallscore{LA5}

\smalltitle{Versicule}
\smallscore{OR_audivi}
\trans{\vv J'entendis une voix du ciel qui me disait : \\ \rr Bienheureux les morts qui meurent dans le Seigneur.}

\gscore{LAB}{B}{}
\trans{\aa Moi, je suis la résurrection et la vie. Celui qui croit en moi, même s’il meurt, vivra ; quiconque vit et croit en moi ne mourra jamais.}

\smalltitle{Cantique de Zacharie}
\smallscore{LPB}
\psalmtext{b}
\smallscore{LAB}

\rubric{On conclut par un \normaltext{Pater noster}, et le reste, comme à Vêpres, page~\pageref{conclusion}.}

\end{document} 
