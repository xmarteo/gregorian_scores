\documentclass[10pt, twoside, french]{article}


%%%%%%%%%%%% GEOMETRY
\usepackage{geometry}
\usepackage{fancyhdr}
\geometry{
	paperwidth=148mm,
	paperheight=210mm,
	inner=20mm,
	outer=12mm,
	top=15mm,
	bottom=12mm,
	headsep=2mm,
}
\pagestyle{empty}

%%%%%%%%%%%% LANGUAGE
\usepackage[nolocalmarks]{polyglossia}
\setdefaultlanguage[variant=french, frenchitemlabels=false]{french}

%%%%%%%%%%%% FONTS AND BASE STYLES
\usepackage{fontspec}
\setmainfont[Ligatures=TeX, Scale=1]{Charis}
\usepackage{paracol}
\usepackage[forcecompile]{gregoriotex}

%% No paragraph indentation
\setlength{\parindent}{0mm}

%% Macro to print rubrics
\newcommand{\rubric}[1]{\textcolor{gregoriocolor}{\emph{#1}}}

%% Macros to print V/ R/ A/ * + symbols in various contexts
\newcommand{\specialcharhsep}{3mm} % space after invoking R/ or V/ or A/ outside rubrics
\newcommand{\vv}{%
	{%
		\fontspec[Scale=1]{Charis}%
		℣.~%
		\nolinebreak[4]%
	}%
}
\newcommand{\redvv}{%
	\textcolor{gregoriocolor}%
	\vv%
	\hspace{\specialcharhsep}%
	\nolinebreak[4]%
}
\newcommand{\aarub}{%
	{%
		\fontspec[Scale=1]{Charis}%
		\Abar.~%
		\nolinebreak[4]%
	}%
}
\newcommand{\redaa}{%
	\textcolor{gregoriocolor}%
	\aarub%
	\hspace{\specialcharhsep}%
	\nolinebreak[4]%
}
\newcommand{\rr}{%
	{%
		\fontspec[Scale=1]{Charis}%
		℟.~%
		\nolinebreak[4]%
	}%
}
\newcommand{\redrr}{%
	\textcolor{gregoriocolor}%
	\rr%
	\hspace{\specialcharhsep}%
	\nolinebreak[4]%
}
\newcommand{\cc}{
	\textcolor{gregoriocolor}{
		\normalfont
		\fontspec[Scale=1]{FreeSerif}
		\symbol{"2720}
	}
}

%% Same special characters, for in-score use (<sp>V/ R/ A/ +</sp>)
\gresetspecial{V/}{\textcolor{gregoriocolor}{\fontspec[Scale=1]{Charis}℣.~}}
\gresetspecial{R/}{\textcolor{gregoriocolor}{\fontspec[Scale=1]{Charis}℟.~}}
\gresetspecial{A/}{\textcolor{gregoriocolor}{\fontspec[Scale=1]{Charis}\Abar.~}}
\gresetspecial{+}{{\fontspec[Scale=1]{FreeSerif}†~}}
\gresetspecial{*}{\gresixstar}
\gresetspecial{cross}{\textcolor{gregoriocolor}{\fontspec[Scale=1]{FreeSerif}\symbol{"2720}}}
\gresetspecial{labiacross}{\textcolor{gregoriocolor}{+}}

%% the asterisk as found in the mediants of text-only psalms
\newcommand{\psstar}{\GreSpecial{*}}
\newcommand{\pscross}{\GreSpecial{+}}

%% Macro to print versicles in two languages
\newcommand{\versiculus}[4]{%
	\begin{paracol}{2}%
	\par\redvv #1 \\ \redrr #2\par%
	\switchcolumn%
	\par\redvv #3 \\ \redrr #4\par%
	\end{paracol}%
}

%% Macro to print capitulum
\newcommand{\capitulum}[3]{%
	\smalltitle{Capitule}
	\begin{paracol}{2}%
	\rubric{#1}
	#2\\%
	\gresetinitiallines{0}%
	\gabcsnippet{(c3) <sp>R/</sp> De(h)o(h) <b>grá</b>(f)ti(e)as.(ef..) (::)}%
	\gresetinitiallines{1}%
	\switchcolumn
	#3\\
	\redrr Nous rendons grâces à Dieu.
	\end{paracol}%
}

%% Macro to print oratio
\newcommand{\oratio}[2]{%
	\versiculus{Orémus.\\#1}{Amen.}{Prions.\\#2}{Amen.}
}

%%%%%%%%%%%% GREGORIO CONFIG

%% \officepartannotation converts a letter (IHARPT) into the office part to be printed as annotation,
%% storing the result into \result.
\newcommand{\result}{}
\newcommand{\lookup}[3]{%
  \IfSubStr{#2}{#1}{ \renewcommand{\result}{#3} }{}%
}%
\newcommand{\officepartannotation}[1]{%
  \renewcommand{\result}{#1}%
  \lookup{#1}{T}{}%
  \lookup{#1}{H}{Hymn.}%
  \lookup{#1}{A}{Ant.}%
  \lookup{#1}{P}{}%
  \lookup{#1}{R}{Resp.}%
  \lookup{#1}{I}{Invit.}%
  \result%
}%

%% header capture setup for the mode
\newcommand{\defaultannotationshift}{-2mm}
\newcommand{\modeannotation}[1]{\greannotation{\hspace{\defaultannotationshift}\hspace{1mm}#1}}
\gresetheadercapture{mode}{modeannotation}{string}

%% outputs a score without annotations or initial
\newcommand{\smallscore}[1]{
	\gresetinitiallines{0}
	\gregorioscore{nocturnale-romanum/gabc/#1}
	\gresetinitiallines{1}
}

%% outputs a score with annotations and initial
\newcommand{\gscore}[3]{
	\greannotation[c]{
		\hspace{-1.4mm}
		\hspace{\defaultannotationshift}
		\officepartannotation{#2}#3
	}
	\gregorioscore{nocturnale-romanum/gabc/#1}
}

%% outputs a hymn with translation
\usepackage{multicol}
\setlength\columnseprule{0.4pt}
\setlength{\multicolsep}{6pt plus 2pt minus 1.5pt}
\newcommand{\hymnus}[2]{
	\smalltitle{Hymne}
	\gscore{#1}{H}{}
	\begin{multicols}{2}%
	\translation{#2}%
	\end{multicols}%
}

%% Initial style
\grechangestyle{initial}{\fontspec{Zallman Caps}\fontsize{28}{28}\selectfont}


%%%%%%%%%%%% TRANSLATION STYLE
\newcommand{\translation}[1]{
	\emph{#1}
}

%%%%%%%%%%%% PSALMODY STYLE
\usepackage{enumitem}
\usepackage{needspace}
%% We want to allow large inter-words space 
%% to avoid overfull boxes in two-columns rubrics.
\sloppy

\newcommand{\parallelitems}[2]{
	\begin{paracol}{2}
	\begin{itemize}[
		label=\null, 
		leftmargin=0pt, 
		itemindent=10pt, 
		labelsep=0pt, 
		labelwidth=0pt, 
		rightmargin=0pt, 
		parsep=0pt, 
		itemsep=0pt,
		topsep=-2mm]
	\input{nocturnale-romanum/psalmi/#1_#2.tex}
	\end{itemize}
	\switchcolumn
	\begin{itemize}[
		label=\null, 
		leftmargin=0pt, 
		itemindent=10pt, 
		labelsep=0pt, 
		labelwidth=0pt, 
		rightmargin=0pt, 
		parsep=0pt, 
		itemsep=0pt,
		topsep=-2mm]
	\input{psalmi_fr/#1.tex}
	\end{itemize}
	\end{paracol}
}

\newcommand{\psalmus}[2]{
	\needspace{4\baselineskip}
	\smalltitle{Psaume #1}
	\parallelitems{#1}{#2}
}

\newcommand{\magnificat}[1]{
	\needspace{4\baselineskip}
	\smalltitle{Magnificat}
	\parallelitems{magn}{#1}
}

%%%%%%%%%%%% TITLE STYLES

\newcommand{\smalltitle}[1]{
  \vspace{0.3\baselineskip}
  \par{\centering\textbf{#1}\par}
  \vspace{0.3\baselineskip}
}

\newcommand{\largetitle}[1]{
  \par{\centering\Huge\textsc{#1}\par}
}

\newcommand{\intermediatetitle}[1]{
  \par{\centering\Large\textsc{#1}\par}
}


%%%%%%%%%%%% GRAPHICS

\newcommand{\sep}{{\centering\greseparator{3}{20}\par}}


\begin{document}

\largetitle{Dimanche de la Passion\\À Matines}

\vfill

\smallscore{ORIa}

\vfill

\translation{Seigneur, ouvre mes lèvres, et ma bouche annoncera ta louange. Dieu, viens à mon aide, Seigneur, hâte-toi de me secourir.
Gloire au Père, et au Fils, et au Saint-Esprit comme il était au commencement, maintenant et toujours, pour les siècles des siècles. Amen. Louange à toi, Seigneur, Roi d'éternelle gloire.}

\pagebreak

\smalltitle{Invitatoire}

\gscore{Q5I}{I}{}
\translation{Aujourd’hui, si vous entendez la voix du Seigneur, n’endurcissez pas vos cœurs.}
\psalmus{94}{VLrepet}

\smalltitle{Hymne}

\gscore{Q5H}{H}{}

\translation{\rubric{1.} Chante, ma langue
la lutte et le glorieux combat;
célèbre le noble triomphe
dont la croix est le trophée,
et la victoire que le Rédempteur du monde
remporta en s'immolant.\\
\rubric{2.} Dieu compatit au malheur
du premier homme sorti de ses mains.
Dès que, mordant à la pomme funeste
Adam se précipita dans la mort,
Dieu lui-même désigna l'arbre nouveau
pour réparer les malheurs causés par le premier.\\
\rubric{3.} Tel fut le plan divin
dressé pour notre salut,
afin que la sagesse y déjouât
la ruse de notre cauteleux ennemi,
et que le remède nous arrivât par le moyen même
qui avait servi pour nous faire la blessure.\\
\rubric{4.} Lors donc que le temps marqué
par le décret divin fut arrivé,
celui par qui le monde a été créé
fut envoyé du trône de son Père,
et ayant pris chair au sein d'une Vierge,
il parut en ce monde.\\
\rubric{5.} Petit enfant, il vagit
couché dans une pauvre crèche,
la Vierge, sa Mère enveloppe de langes
ses membres délicats,
et des bandelettes étroites serrent
les mains et les pieds d'un Dieu.\\
\rubric{6.} Que toujours en sa béatitude
à la Trinité soit la gloire,
également au Père et au Fils;
pareil honneur au Paraclet:
que du Dieu trine et un, le nom
soit loué dans tout l’Univers.
Amen.}

\intermediatetitle{Premier nocturne}

\gscore{F1N1A1}{A}{1}
\translation{Bienheureux l'homme qui médite la loi du Seigneur.}
\psalmus{1}{8}
%\smallscore{F1N1A1}

\gscore{F1N1A2}{A}{2}
\translation{Servez le Seigneur dans la crainte, et exultez devant lui en tremblant.}
\psalmus{2}{7}
%\smallscore{F1N1A2}

\gscore{F1N1A3}{A}{3}
\translation{Lève-toi, Seigneur, sauve-moi, mon Dieu.}
\psalmus{3}{6}
%\smallscore{F1N1A3}

\versiculus{Erue a frámea, Deus, ánimam meam.}{Et de manu canis únicam meam.}{Délivre mon âme du glaive, ô Dieu.}{Et de l’atteinte du chien, mon unique.}

\smallscore{ORPN}
\translation{Notre Père...\\Et ne nous laisse pas entrer en tentation.\\Mais délivre-nous du Mal.}

\smalltitle{Absolution}

\smallscore{ORA}
\translation{Seigneur Jésus-Christ, exauce les prières de tes serviteurs, et aie pitié de nous,
toi qui vis et règnes avec le Père et le Saint-Esprit, dans les siècles des siècles.\\ \rubric{\rr} Amen.}

\smalltitle{Première leçon}

\rubric{Au début de chaque lecture, le lecteur commence par:}
\smallscore{ORLb}
\translation{Veuillez, Maître, bénir.\\
\rubric{Bén.} Que le Père éternel nous bénisse d'une bénédiction perpétuelle.\\
\rubric{\rr} Amen.}

Commencement du livre du Prophète Jérémie.

Paroles de Jérémie, fils de Helkias,
	l’un des prêtres qui étaient à Anatoth, au pays de Benjamin.
La parole du Seigneur lui fut adressée au temps de Josias,
		fils d’Amone, roi de Juda, la treizième année de son règne;
	puis au temps de Joakim, fils de Josias, roi de Juda,
	jusqu’à la fin de la onzième année de Sédécias, fils de Josias, roi de Juda,
	jusqu’à la déportation de Jérusalem, au cinquième mois.
La parole du Seigneur me fut adressée:
	«Avant même de te façonner dans le sein de ta mère, je te connaissais;
	avant que tu viennes au jour, je t’ai consacré;
	je fais de toi un prophète pour les nations.»
Et je dis: «Ah! Seigneur mon Dieu! Vois donc:
	je ne sais pas parler, je suis un enfant!»


\rubric{À la fin de chaque lecture, le lecteur ajoute:}
\smallscore{ORLd}
\translation{Et toi, Seigneur, aie pitié de nous. \rubric{\rr} Nous rendons grâces à Dieu.}

\gscore{Q5N1R1}{R}{1}
\translation{\rr Voici les jours de fête que vous observerez en leurs temps :
* Au quatorzième jour du premier mois, vers le soir, est la Pâque du Seigneur, et au quinzième jour vous célébrerez une solennité en l’honneur du Dieu très-haut.
\vv Le Seigneur parla à Moïse, disant : Parle aux enfants d’Israël, et tu leur diras.
* Au quatorzième jour...}

\smalltitle{Deuxième leçon}

\begin{paracol}{2}
\rubric{Lector:} Jube, domne, benedícere.\\
\rubric{\emph{Benedictio 2.}} Unigénitus \textit{Dei} \textbf{Fí}lius~\GreSpecial{*}
nos benedícere et adjuváre dignétur.
\hspace{\specialcharhsep}\redrr Amen.
\switchcolumn
\rubric{Le lecteur:} Veuillez, Maître, bénir.\\
\rubric{\emph{Bén. 2.}} Que le Fils unique de Dieu daigne nous bénir et nous secourir.
\hspace{\specialcharhsep}\redrr Amen.
\end{paracol}

Le Seigneur reprit: «Ne dis pas: “Je suis un enfant!”
	Tu iras vers tous ceux à qui je t’enverrai;
	tout ce que je t’ordonnerai, tu le diras.
Ne les crains pas, car je suis avec toi pour te délivrer --- oracle du Seigneur.»
Puis le Seigneur étendit la main et me toucha la bouche.
Il me dit:
	«Voici, je mets dans ta bouche mes paroles!
	Vois: aujourd’hui, je te donne autorité sur les nations et les royaumes,
	pour arracher et renverser, pour détruire et démolir, pour bâtir et planter.»
La parole du Seigneur me fut adressée:
	«Que vois-tu, Jérémie?»
	Je dis: «C’est une branche d’amandier que je vois.»
Le Seigneur me dit:
	«Tu as bien vu, car je veille sur ma parole pour l’accomplir.»
Une deuxième fois, la parole du Seigneur me fut adressée:
	«Que vois-tu?»
Je dis: «C’est un chaudron bouillonnant que je vois; il s’ouvre depuis le nord.»


\gscore{Q5N1R2}{R}{2}
\translation{\rr Ils se sont multipliés ceux qui me persécutent, et ils disent : II n’y a pas de salut pour lui en son Dieu.
* Lève-toi, Seigneur ; sauve-moi, mon Dieu.
\vv De peur qu’un jour mon ennemi ne dise : J’ai prévalu contre lui.
* Lève-toi...}

\smalltitle{Troisième leçon}

\begin{paracol}{2}
\rubric{Lector:} Jube, domne, benedícere.\\
\rubric{\emph{Benedictio 3.}} Spíritus \textit{Sancti} \textbf{grá}tia~\GreSpecial{*}
illúminet sensus et corda nostra.
\hspace{\specialcharhsep}\redrr Amen.
\switchcolumn
\rubric{Le lecteur:} Veuillez, Maître, bénir.\\
\rubric{\emph{Bén. 3.}} Que la grâce du Saint-Esprit illumine nos esprits et nos cœurs.
\hspace{\specialcharhsep}\redrr Amen.
\end{paracol}

Le Seigneur me dit:
	«Du nord, va déferler le malheur sur tous les habitants du pays.
Voici, je convoque tous les clans des royaumes du nord --- oracle du Seigneur.
Ils arrivent, et chacun placera son trône à l’entrée des portes de Jérusalem,
	contre tous les remparts qui l’entourent
	et contre toutes les villes de Juda.
Je vais prononcer sur eux mes jugements à cause de toute leur méchanceté,
	car ils m’ont abandonné, ils ont brûlé de l’encens pour d’autres dieux
	et se sont prosternés devant l’œuvre de leurs mains.
Toi, mets ta ceinture autour des reins et lève-toi,
	tu diras contre eux tout ce que je t’ordonnerai.
Ne tremble pas devant eux, sinon c’est moi qui te ferai trembler devant eux.
Moi, je fais de toi aujourd’hui une ville fortifiée,
	une colonne de fer, un rempart de bronze,
	pour faire face à tout le pays, aux rois de Juda et à ses princes,
	à ses prêtres et à tout le peuple du pays.
Ils te combattront, mais ils ne pourront rien contre toi,
	car je suis avec toi pour te délivrer --- oracle du Seigneur.»


\gscore{Q5N1R3}{R}{3}
\translation{\rr Jusques à quand mon ennemi s’élèvera-t-il au-dessus de moi ?
* Regarde et exauce-moi, Seigneur mon Dieu.
\vv Ceux qui me tourmentent tressailliront de joie, si je suis ébranlé ; mais moi, j’ai espéré dans votre miséricorde.
* Regardez et exaucez-moi, Seigneur mon Dieu.
\rr Jusques à quand...}

\intermediatetitle{Deuxième nocturne}

\gscore{F1N2A1}{A}{4}
\translation{Qu'il est admirable ton nom, Seigneur, par toute la terre!}
\psalmus{8}{1}
%\smallscore{F1N2A1}

\gscore{F1N2A2}{A}{5}
\translation{Tu sièges sur le trône, toi qui juges avec justice.}
\psalmus{9i}{8}
%\smallscore{F1N2A2}

\gscore{F1N2A3}{A}{6}
\translation{Lève-toi, Seigneur, que l'homme ne triomphe pas.}
\psalmus{9ii}{1}
%\smallscore{F1N2A3}

\versiculus{De ore leónis líbera me, Dómine.}{Et a cónibus unicónium humilitátem meam.}{De la gueule du lion, délivre-moi. Seigneur.}{Et des cornes des buffles, ma faiblesse.}

\smallscore{ORPN}
\translation{Notre Père...\\Et ne nous laisse pas entrer en tentation.\\Mais délivre-nous du Mal.}

\smalltitle{Absolution}

\begin{paracol}{2}
\rubric{\emph{Absolutio 2.}}
Ipsíus píetas et misericódi\textit{a nos} \textbf{ád}juvet,~\GreSpecial{*}
qui cum Patre et Spíritu Sancto vivit et regnat in sǽcula sæculórum.
\hspace{\specialcharhsep}\redrr Amen.
\switchcolumn
\rubric{\emph{Absolution 2.}}
Qu'il nous secoure par sa bonté et sa miséricorde, celui qui, avec le Père et le Saint-Esprit, vit et règne dans les siècles des siècles.
\hspace{\specialcharhsep}\redrr Amen.
\end{paracol}

\smalltitle{Quatrième leçon}

\begin{paracol}{2}
\rubric{Lector:} Jube, domne, benedícere.\\
\rubric{\emph{Benedictio 4.}} Deus Pa\textit{ter om}\textbf{ní}potens~\GreSpecial{*}
sit nobis propítius et clemens.
\hspace{\specialcharhsep}\redrr Amen.
\switchcolumn
\rubric{Le lecteur:} Veuillez, Maître, bénir.\\
\rubric{\emph{Bén. 4.}}
Que Dieu le Père tout-puissant soit pour nous propice et plein de clémence.
\hspace{\specialcharhsep}\redrr Amen.
\end{paracol}

Sermon de saint Léon, Pape.

Nous n’ignorons pas, mes bien-aimés, que le mystère pascal occupe le premier rang parmi toutes les solennités chrétiennes. Notre manière de vivre durant l’année tout entière doit, il est vrai, par la réforme de nos mœurs, nous disposer à le célébrer d’une manière digne et convenable ; mais les jours présents exigent au plus haut degré notre dévotion, car nous savons qu’ils sont proches de celui où nous célébrons le mystère très sublime de la divine miséricorde. C’est avec raison et par l’inspiration de l’Esprit-Saint, que les saints Apôtres ont ordonné pour ces jours des jeûnes plus austères, afin que par une participation commune à la croix du Christ, nous fassions, nous aussi, quelque chose qui nous unisse à ce qu’il a fait pour nous. Comme le dit l’Apôtre : « Si nous souffrons avec lui, nous serons glorifiés avec lui. » Là où il y a participation à la passion du Seigneur, on peut regarder comme certaine et assurée l’attente du bonheur qu’il a promis.
 
\gscore{Q5N2R1}{R}{4}
\translation{\rr Tu es mon Dieu, ne t'éloigne pas de moi :
* Parce que la tribulation est proche, et il n’y a personne qui me porte secours.
\vv Mais toi, Seigneur, n’éloigne pas ton secours de moi, viens à ma défense.
* Parce que...}

\smalltitle{Cinquième leçon}

\begin{paracol}{2}
\rubric{Lector:} Jube, domne, benedícere.\\
\rubric{\emph{Benedictio 5.}} Chris\textit{tus per}\textbf{pé}tuæ~\GreSpecial{*}
det nobis gaúdia vitæ.
\hspace{\specialcharhsep}\redrr Amen.
\switchcolumn
\rubric{Le lecteur:} Veuillez, Maître, bénir.\\
\rubric{\emph{Bén. 5.}}
Que le Christ nous donne les joies de l'éternelle vie.
\hspace{\specialcharhsep}\redrr Amen.
\end{paracol}

II n’est personne, mes bien-aimés à qui Dieu refuse de l’associer à cette gloire et la condition du temps n’y met pas obstacle, comme si dans la tranquillité et la paix il n’y avait point d’occasion de montrer du courage et de pratiquer la vertu. L’Apôtre l’a prédit en disant : « Tous ceux qui veulent vivre pieusement dans le Christ, souffriront persécution » ; et c’est pourquoi l’épreuve et la persécution ne manquent jamais, si la pratique de la piété ne fait jamais défaut. Le Seigneur en exhortant ses Apôtres, leur dit : « Celui qui ne prend pas sa croix et ne me suit pas, n’est pas digne de moi. ». Cette parole, nous n’en pouvons douter, s’applique non seulement aux disciples du Christ, mais à tous les fidèles, à l’Église entière, qui, dans son universalité, écoutait les conditions du salut en la personne de ceux qui étaient alors présents.

\gscore{Q5N2R2}{R}{5}
\translation{\rr C’est sur toi que j’ai été jeté en sortant du sein maternel ; depuis que j’étais dans les entrailles de ma mère, tu es mon Dieu ; ne t'éloigne pas de moi :
* Parce que la tribulation est proche, et il n’y a personne qui me porte secours.
\vv Sauve-moi de la gueule du lion, et ma faiblesse des cornes des licornes [rhinocéros].
* Parce que...}

\smalltitle{Sixième leçon}

\begin{paracol}{2}
\rubric{Lector:} Jube, domne, benedícere.\\
\rubric{\emph{Benedictio 6.}} Ignem su\textit{i a}\textbf{mó}ris~\GreSpecial{*}
accéndat Deus in córdibus nostris.
\hspace{\specialcharhsep}\redrr Amen.
\switchcolumn
\rubric{Le lecteur:} Veuillez, Maître, bénir.\\
\rubric{\emph{Bén. 6.}}
Que Dieu daigne allumer dans nos cœurs le feu de son amour.
\hspace{\specialcharhsep}\redrr Amen.
\end{paracol}

Comme il convient à tout ce corps de vivre pieusement, ainsi l’obligation de porter la croix est-elle de tous les temps ; ce n’est pas sans raison qu’il est conseillé à chacun de porter sa croix, car chacun s’en voit chargé d’une manière et dans une mesure qui lui sont propres. La persécution n’est désignée que par un seul mot, mais il existe plus d’une cause de combat, et il y a ordinairement plus à craindre d’un ennemi qui tend des pièges en secret que d’un adversaire déclaré. Le bienheureux Job, qui avait appris que les biens et les maux se succèdent en ce monde, disait avec piété et vérité : « N’est-ce pas une tentation que la vie de l’homme sur la terre ? ». Ce ne sont pas seulement les douleurs et les supplices du corps qui assaillent l’âme fidèle, car elle est menacée d’une grave maladie, encore que tous les membres demeurent parfaitement sains, si elle se laisse amollir par les plaisirs des sens. Mais comme « la chair convoite contre l’esprit, et l’esprit contre la chair », l’âme raisonnable est munie du secours de la croix du Christ, et moyennant ce secours, elle ne consent pas aux désirs coupables lorsqu’elle est tentée, parce qu’elle est transpercée et attachée par les clous de la continence et par la crainte de Dieu.

\gscore{Q5N2R3}{R}{6}
\translation{\rr Ma tribulation est proche, Seigneur, et il n’est personne qui me porte secours ; ils m’assiègent pour percer mes mains et mes pieds : sauve-moi de la gueule du lion,
* Afin que je raconte ton nom à mes frères.
\vv Arrache mon âme à l’épée à double tranchant ; et mon unique de la main du chien.
* Afin que je raconte ton nom à mes frères.
\rr Ma tribulation...}

\intermediatetitle{Troisième nocturne}

\gscore{F1N3A1}{A}{7}
\translation{Pourquoi, Seigneur, te tenir à l'écart?}
\psalmus{9iii}{2}
%\smallscore{F1N3A1}

\gscore{F1N3A2}{A}{8}
\translation{Lève-toi, Seigneur Dieu, que soit exaltée ta main.}
\psalmus{9iv}{5}
%\smallscore{F1N3A2}

\gscore{F1N3A3}{A}{9}
\translation{Juste est le Seigneur et il aime la justice.}
\psalmus{10}{1}
%\smallscore{F1N3A3}


\versiculus{Ne perdas cum impiis, Deus, ánimam meam.}{Et cum viris sánguinum vitam meam.}{Ne laisse pas mon âme se perdre avec les impies, ô Dieu.}{Ni ma vie avec les hommes de sang.}

\smallscore{ORPN}
\translation{Notre Père...\\Et ne nous laisse pas entrer en tentation.\\Mais délivre-nous du Mal.}

\smalltitle{Absolution}

\begin{paracol}{2}
\rubric{\emph{Absolutio 3.}}
A vínculis peccató\textit{rum nos}\textbf{tró}rum~\GreSpecial{*}
absólvat nos omnípotens et miséricors Dóminus.
\hspace{\specialcharhsep}\redrr Amen.
\switchcolumn
\rubric{\emph{Absolution 3.}}
Que le Dieu tout-puissant et miséricordieux daigne nous délivrer des liens de nos péchés.
\hspace{\specialcharhsep}\redrr Amen.
\end{paracol}

\smalltitle{Septième leçon}

\begin{paracol}{2}
\rubric{Lector:} Jube, domne, benedícere.\\
\rubric{\emph{Benedictio 7.}}
Evangé\textit{lica} \textbf{léc}tio~\GreSpecial{*}
sit nobis salus et protéctio.
\hspace{\specialcharhsep}\redrr Amen.
\switchcolumn
\rubric{Le lecteur:} Veuillez, Maître, bénir.\\
\rubric{\emph{Bén. 7.}}
Que la lecture du saint Évangile nous soit salut et protection.
\hspace{\specialcharhsep}\redrr Amen.
\end{paracol}

Lecture du saint Évangile selon saint Jean.
 
En ce temps-là : Jésus disait à la foule des Juifs : Qui de vous me convaincra de péché ? Si je vous dis la vérité, pourquoi ne me croyez-vous point ? Et ainsi de suite.

Homélie de saint Grégoire, Pape.

Considérez, mes très chers frères, la mansuétude de Dieu. Le Sauveur était venu effacer les péchés du monde, et il disait : « Qui de vous me convaincra de péché ? » Il ne dédaigne pas de montrer par le raisonnement qu’il n’est pas un pécheur, lui qui, par la vertu de sa divinité, avait le pouvoir de justifier les pécheurs. Les paroles qui suivent sont vraiment terribles : « Celui qui est de Dieu écoute les paroles de Dieu. Et si vous ne les écoutez peint c’est que vous n’êtes point de Dieu. » Si donc celui qui est de Dieu entend les paroles de Dieu, et si au contraire celui qui n’est pas de Dieu ne peut les entendre, que chacun se demande si l’oreille de son cœur perçoit les paroles de Dieu, et il connaîtra à qui il appartient. La Vérité ordonne de désirer la patrie céleste, de fouler aux pieds les désirs de la chair, de fuir la gloire du monde, de ne point convoiter le bien d’autrui, et de donner généreusement ce que l’on possède.

\gscore{Q5N3R1}{R}{7}
\translation{\rr Tout le jour je marchais contristé, Seigneur, parce que mon âme est remplie d’illusions,
* Et ceux qui cherchaient mon âme usaient de violence.
\vv Mes amis et mes proches se sont approchés vis-à-vis de moi, et ils se sont arrêtés ; et ceux qui étaient près de moi, s’en sont tenus éloignés.
* Et ceux...}

\smalltitle{Huitième leçon}

\begin{paracol}{2}
\rubric{Lector:} Jube, domne, benedícere.\\
\rubric{\emph{Benedictio 8.}}
Diví\textit{num au}\textbf{xí}lium~\GreSpecial{*}
máneat semper nobíscum.
\hspace{\specialcharhsep}\redrr Amen.
\switchcolumn
\rubric{Le lecteur:} Veuillez, Maître, bénir.\\
\rubric{\emph{Bén. 8.}}
Que le secours divin demeure toujours avec nous.
\hspace{\specialcharhsep}\redrr Amen.
\end{paracol}

Que chacun de vous examine donc en lui-même si cette voix de Dieu frappe fortement l’oreille de son cœur, et il connaîtra s’il est déjà de Dieu. Il y en a quelques-uns qui ne daignent pas même écouter des oreilles du corps, les préceptes divins. Il en est d’autres qui les entendent, il est vrai, de l’oreille du corps, mais sans avoir dans l’âme aucun désir de les pratiquer. Il y en a d’autres encore, qui reçoivent volontiers les paroles de Dieu, au point même d’en être touchés jusqu’aux larmes, mais, aussitôt que ce moment d’émotion est passé, ils retournent au péché. Tous ceux-là n’écoutent assurément point les paroles de Dieu, puisqu’ils négligent de les mettre en pratique par leurs œuvres. Remettez donc votre vie passée devant les yeux de votre âme, mes très chers frères, et imprimez profondément dans vos cœurs, le sentiment de crainte que doivent inspirer ces paroles qui ont été prononcées par la Vérité même : « Si vous ne les écoutez point, c’est que vous n’êtes point de Dieu. »

\gscore{Q5N3R2}{R}{8}
\translation{\rr Seigneur, ne détourne pas ta face de ton serviteur :
* Parce que je suis tourmenté, exauce-moi promptement.
\vv Sois attentif à mon âme et délivre-la à cause de mes ennemis ; sauve-moi.
* Parce que...}

\smalltitle{Neuvième leçon}

\begin{paracol}{2}
\rubric{Lector:} Jube, domne, benedícere.\\
\rubric{\emph{Benedictio 9.}}
Ad societátem cívium \textit{super}\textbf{nó}rum~\GreSpecial{*}
perdúcat nos Rex Angelórum.
\hspace{\specialcharhsep}\redrr Amen.
\switchcolumn
\rubric{Le lecteur:} Veuillez, Maître, bénir.\\
\rubric{\emph{Bén. 9.}}
Que le Roi des Anges nous fasse parvenir à la société des citoyens célestes.
\hspace{\specialcharhsep}\redrr Amen.
\end{paracol}

Mais ce que la Vérité dit de ces réprouvés, ces hommes condamnables le montrent eux-mêmes par leurs œuvres d’iniquité ; voici en effet ce qu’on lit après : « Les Juifs lui répondirent, et lui dirent : Ne disons-nous pas avec raison que tu es un Samaritain, et qu’un démon est en toi ? » Écoutez ce que repartit le Seigneur, après avoir reçu un tel outrage : « II n’y a pas de démon en moi ; mais j’honore mon Père, et vous, vous me déshonorez. » Le mot Samaritain signifie gardien, et le Sauveur est véritablement lui-même ce gardien dont le Psalmiste a dit : « Si le Seigneur ne garde une cité, inutilement veille celui qui la garde » ; et ce gardien auquel il est dit dans Isaïe : « Garde, où en est la nuit ? garde, où en est la nuit ? » Voilà pourquoi le Seigneur ne voulut pas répondre : Je ne suis pas un Samaritain, et dit seulement : « II n’y a pas de démon en moi. » Deux choses lui avaient été reprochées : il nia l’une, et convint de l’autre par son silence.

\gscore{Q5N3R3}{R}{9}
\translation{\rr Qui donnera à ma tête de l’eau, et à mes yeux une fontaine de larmes, et je pleurerai jour et nuit ? parce que mon frère, mon proche parent m’a supplanté,
* Et tous mes amis ont usé de fraude envers moi.
\vv Que leurs voies deviennent ténébreuses et glissantes, et qu’un Ange du Seigneur les poursuive.
* Et tous mes amis ont usé de fraude envers moi.
\rr Qui donnera...}

\versiculus{Dóminus vobíscum.}{Et cum spíritu tuo.}{Le Seigneur soit avec vous.}{Et avec votre esprit.}

\oratio{Quǽsumus, omnípotens Deus, famíliam tuam propítius réspice:~\pscross{} ut, te largiénte, regátur in córpore;~\psstar{} et, te servánte, custodiátur in mente. Per Dóminum nostrum Jesum Christum, Fílium tuum: qui tecum vivit et regnat in unitáte Spíritus Sancti, Deus, per ómnia sǽcula sæculórum.}{Nous t'en prions, Dieu tout-puissant, regarde tes enfants dans ta miséricorde ; accorde-leur ta grâce pour qu’ils soient gouvernés en leur corps, et veille sur eux pour qu’ils soient gardés en leur âme. Par Jésus-Christ, ton Fils, notre Seigneur, qui vit et règne avec toi et le Saint-Esprit, Dieu, maintenant et pour les siècles des siècles.}

\versiculus{Dóminus vobíscum.}{Et cum spíritu tuo.}{Le Seigneur soit avec vous.}{Et avec votre esprit.}

\smallscore{ORBDm}
\translation{\vv Bénissons le Seigneur. \rr Nous rendons grâces à Dieu.}
\versiculus{Fidélium ánimæ per misericórdiam Dei requiéscant in pace.}{Amen.}{Que par la miséricorde de Dieu, les âmes des fidèles trépassés reposent en paix.}{Ainsi soit-il.}


\end{document}