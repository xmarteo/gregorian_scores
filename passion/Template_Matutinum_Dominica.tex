\documentclass[10pt, twoside, french]{article}


%%%%%%%%%%%% GEOMETRY
\usepackage{geometry}
\usepackage{fancyhdr}
\geometry{
	paperwidth=148mm,
	paperheight=210mm,
	inner=20mm,
	outer=12mm,
	top=15mm,
	bottom=12mm,
	headsep=2mm,
}
\pagestyle{empty}

%%%%%%%%%%%% LANGUAGE
\usepackage[nolocalmarks]{polyglossia}
\setdefaultlanguage[variant=french, frenchitemlabels=false]{french}

%%%%%%%%%%%% FONTS AND BASE STYLES
\usepackage{fontspec}
\setmainfont[Ligatures=TeX, Scale=1]{Charis}
\usepackage{paracol}
\usepackage[forcecompile]{gregoriotex}

%% No paragraph indentation
\setlength{\parindent}{0mm}

%% Macro to print rubrics
\newcommand{\rubric}[1]{\textcolor{gregoriocolor}{\emph{#1}}}

%% Macros to print V/ R/ A/ * + symbols in various contexts
\newcommand{\specialcharhsep}{3mm} % space after invoking R/ or V/ or A/ outside rubrics
\newcommand{\vv}{%
	{%
		\fontspec[Scale=1]{Charis}%
		℣.~%
		\nolinebreak[4]%
	}%
}
\newcommand{\redvv}{%
	\textcolor{gregoriocolor}%
	\vv%
	\hspace{\specialcharhsep}%
	\nolinebreak[4]%
}
\newcommand{\aarub}{%
	{%
		\fontspec[Scale=1]{Charis}%
		\Abar.~%
		\nolinebreak[4]%
	}%
}
\newcommand{\redaa}{%
	\textcolor{gregoriocolor}%
	\aarub%
	\hspace{\specialcharhsep}%
	\nolinebreak[4]%
}
\newcommand{\rr}{%
	{%
		\fontspec[Scale=1]{Charis}%
		℟.~%
		\nolinebreak[4]%
	}%
}
\newcommand{\redrr}{%
	\textcolor{gregoriocolor}%
	\rr%
	\hspace{\specialcharhsep}%
	\nolinebreak[4]%
}
\newcommand{\cc}{
	\textcolor{gregoriocolor}{
		\normalfont
		\fontspec[Scale=1]{FreeSerif}
		\symbol{"2720}
	}
}

%% Same special characters, for in-score use (<sp>V/ R/ A/ +</sp>)
\gresetspecial{V/}{\textcolor{gregoriocolor}{\fontspec[Scale=1]{Charis}℣.~}}
\gresetspecial{R/}{\textcolor{gregoriocolor}{\fontspec[Scale=1]{Charis}℟.~}}
\gresetspecial{A/}{\textcolor{gregoriocolor}{\fontspec[Scale=1]{Charis}\Abar.~}}
\gresetspecial{+}{{\fontspec[Scale=1]{FreeSerif}†~}}
\gresetspecial{*}{\gresixstar}
\gresetspecial{cross}{\textcolor{gregoriocolor}{\fontspec[Scale=1]{FreeSerif}\symbol{"2720}}}
\gresetspecial{labiacross}{\textcolor{gregoriocolor}{+}}

%% the asterisk as found in the mediants of text-only psalms
\newcommand{\psstar}{\GreSpecial{*}}
\newcommand{\pscross}{\GreSpecial{+}}

%% Macro to print versicles in two languages
\newcommand{\versiculus}[4]{%
	\begin{paracol}{2}%
	\par\redvv #1 \\ \redrr #2\par%
	\switchcolumn%
	\par\redvv #3 \\ \redrr #4\par%
	\end{paracol}%
}

%% Macro to print capitulum
\newcommand{\capitulum}[3]{%
	\smalltitle{Capitule}
	\begin{paracol}{2}%
	\rubric{#1}
	#2\\%
	\gresetinitiallines{0}%
	\gabcsnippet{(c3) <sp>R/</sp> De(h)o(h) <b>grá</b>(f)ti(e)as.(ef..) (::)}%
	\gresetinitiallines{1}%
	\switchcolumn
	#3\\
	\redrr Nous rendons grâces à Dieu.
	\end{paracol}%
}

%% Macro to print oratio
\newcommand{\oratio}[2]{%
	\versiculus{Orémus.\\#1}{Amen.}{Prions.\\#2}{Amen.}
}

%%%%%%%%%%%% GREGORIO CONFIG

%% \officepartannotation converts a letter (IHARPT) into the office part to be printed as annotation,
%% storing the result into \result.
\newcommand{\result}{}
\newcommand{\lookup}[3]{%
  \IfSubStr{#2}{#1}{ \renewcommand{\result}{#3} }{}%
}%
\newcommand{\officepartannotation}[1]{%
  \renewcommand{\result}{#1}%
  \lookup{#1}{T}{}%
  \lookup{#1}{H}{Hymn.}%
  \lookup{#1}{A}{Ant.}%
  \lookup{#1}{P}{}%
  \lookup{#1}{R}{Resp.}%
  \lookup{#1}{I}{Invit.}%
  \result%
}%

%% header capture setup for the mode
\newcommand{\defaultannotationshift}{-2mm}
\newcommand{\modeannotation}[1]{\greannotation{\hspace{\defaultannotationshift}\hspace{1mm}#1}}
\gresetheadercapture{mode}{modeannotation}{string}

%% outputs a score without annotations or initial
\newcommand{\smallscore}[1]{
	\gresetinitiallines{0}
	\gregorioscore{nocturnale-romanum/gabc/#1}
	\gresetinitiallines{1}
}

%% outputs a score with annotations and initial
\newcommand{\gscore}[3]{
	\greannotation[c]{
		\hspace{-1.4mm}
		\hspace{\defaultannotationshift}
		\officepartannotation{#2}#3
	}
	\gregorioscore{nocturnale-romanum/gabc/#1}
}

%% outputs a hymn with translation
\usepackage{multicol}
\setlength\columnseprule{0.4pt}
\setlength{\multicolsep}{6pt plus 2pt minus 1.5pt}
\newcommand{\hymnus}[2]{
	\smalltitle{Hymne}
	\gscore{#1}{H}{}
	\begin{multicols}{2}%
	\translation{#2}%
	\end{multicols}%
}

%% Initial style
\grechangestyle{initial}{\fontspec{Zallman Caps}\fontsize{28}{28}\selectfont}


%%%%%%%%%%%% TRANSLATION STYLE
\newcommand{\translation}[1]{
	\emph{#1}
}

%%%%%%%%%%%% PSALMODY STYLE
\usepackage{enumitem}
\usepackage{needspace}
%% We want to allow large inter-words space 
%% to avoid overfull boxes in two-columns rubrics.
\sloppy

\newcommand{\parallelitems}[2]{
	\begin{paracol}{2}
	\begin{itemize}[
		label=\null, 
		leftmargin=0pt, 
		itemindent=10pt, 
		labelsep=0pt, 
		labelwidth=0pt, 
		rightmargin=0pt, 
		parsep=0pt, 
		itemsep=0pt,
		topsep=-2mm]
	\input{nocturnale-romanum/psalmi/#1_#2.tex}
	\end{itemize}
	\switchcolumn
	\begin{itemize}[
		label=\null, 
		leftmargin=0pt, 
		itemindent=10pt, 
		labelsep=0pt, 
		labelwidth=0pt, 
		rightmargin=0pt, 
		parsep=0pt, 
		itemsep=0pt,
		topsep=-2mm]
	\input{psalmi_fr/#1.tex}
	\end{itemize}
	\end{paracol}
}

\newcommand{\psalmus}[2]{
	\needspace{4\baselineskip}
	\smalltitle{Psaume #1}
	\parallelitems{#1}{#2}
}

\newcommand{\magnificat}[1]{
	\needspace{4\baselineskip}
	\smalltitle{Magnificat}
	\parallelitems{magn}{#1}
}

%%%%%%%%%%%% TITLE STYLES

\newcommand{\smalltitle}[1]{
  \vspace{0.3\baselineskip}
  \par{\centering\textbf{#1}\par}
  \vspace{0.3\baselineskip}
}

\newcommand{\largetitle}[1]{
  \par{\centering\Huge\textsc{#1}\par}
}

\newcommand{\intermediatetitle}[1]{
  \par{\centering\Large\textsc{#1}\par}
}


%%%%%%%%%%%% GRAPHICS

\newcommand{\sep}{{\centering\greseparator{3}{20}\par}}


\begin{document}

\largetitle{Dimanche de la Passion\\À Matines}

\smallscore{ORIa}
\translation{Seigneur, ouvre mes lèvres, et ma bouche annoncera ta louange. Dieu, viens à mon aide, Seigneur, hâte-toi de me secourir.
Gloire au Père, et au Fils, et au Saint-Esprit comme il était au commencement, maintenant et toujours, pour les siècles des siècles. Amen. Louange à toi, Seigneur, Roi d'éternelle gloire.}

\smalltitle{Invitatoire}

\gscore{Q5I}{I}{}
\translation{Aujourd’hui, si vous entendez la voix du Seigneur, n’endurcissez pas vos cœurs.}
\psalmus{94}{VLrepet}

\smalltitle{Hymne}

\gscore{Q5H}{H}{}

\translation{\rubric{1.} Chante, ma langue
la lutte et le glorieux combat;
célèbre le noble triomphe
dont la croix est le trophée,
et la victoire que le Rédempteur du monde
remporta en s'immolant.\\\\
\rubric{2.} Dieu compatit au malheur
du premier homme sorti de ses mains.
Dès que, mordant à la pomme funeste
Adam se précipita dans la mort,
Dieu lui-même désigna l'arbre nouveau
pour réparer les malheurs causés par le premier.\\\\
\rubric{3.} Tel fut le plan divin
dressé pour notre salut,
afin que la sagesse y déjouât
la ruse de notre cauteleux ennemi,
et que le remède nous arrivât par le moyen même
qui avait servi pour nous faire la blessure.\\\\
\rubric{4.} Lors donc que le temps marqué
par le décret divin fut arrivé,
celui par qui le monde a été créé
fut envoyé du trône de son Père,
et ayant pris chair au sein d'une Vierge,
il parut en ce monde.\\\\
\rubric{5.} Petit enfant, il vagit
couché dans une pauvre crèche,
la Vierge, sa Mère enveloppe de langes
ses membres délicats,
et des bandelettes étroites serrent
les mains et les pieds d'un Dieu.\\\\
\rubric{6.} Que toujours en sa béatitude
à la Trinité soit la gloire,
également au Père et au Fils;
pareil honneur au Paraclet:
que du Dieu trine et un, le nom
soit loué dans tout l’Univers.
Amen.}

\intermediatetitle{Premier nocturne}

\gscore{F1N1A1}{A}{1}
\translation{Bienheureux l'homme qui médite la loi du Seigneur.}
\psalmus{1}{8}
\smallscore{F1N1A1}

\gscore{F1N1A2}{A}{2}
\translation{Servez le Seigneur dans la crainte, et exultez devant lui en tremblant.}
\psalmus{2}{7}
\smallscore{F1N1A2}

\gscore{F1N1A3}{A}{3}
\translation{Lève-toi, Seigneur, sauve-moi, mon Dieu.}
\psalmus{3}{6}
\smallscore{F1N1A3}

\versiculus{Erue a frámea, Deus, ánimam meam.}{Et de manu canis únicam meam.}{Délivre mon âme du glaive, ô Dieu.}{Et de l’atteinte du chien, mon unique.}

\smallscore{ORPN}
\translation{Notre Père...\\Et ne nous laisse pas entrer en tentation.\\Mais délivre-nous du Mal.}

\smalltitle{Absolution}

\smallscore{ORA}
\translation{Seigneur Jésus-Christ, exauce les prières de tes serviteurs, et aie pitié de nous,
toi qui vis et règnes avec le Père et le Saint-Esprit, dans les siècles des siècles.\\ \rubric{\rr} Amen.}

\smalltitle{Première leçon}

\rubric{Au début de chaque lecture, le lecteur commence par:}
\smallscore{ORLb}
\translation{Veuillez, Maître, bénir.\\
\rubric{Bén.} Que le Père éternel nous bénisse d'une bénédiction perpétuelle.\\
\rubric{\rr} Amen.}

\rubric{À la fin de chaque lecture, le lecteur ajoute:}
\smallscore{ORLd}
\translation{Et toi, Seigneur, aie pitié de nous. \rubric{\rr} Nous rendons grâces à Dieu.}

\smalltitle{Deuxième leçon}

\begin{paracol}{2}
\rubric{Lector:} Jube, domne, benedícere.\\
\rubric{\emph{Benedictio 2.}} Unigénitus \textit{Dei} \textbf{Fí}lius~\GreSpecial{*}
nos benedícere et adjuváre dignétur.
\hspace{\specialcharhsep}\redrr Amen.
\switchcolumn
\rubric{Le lecteur:} Veuillez, Maître, bénir.\\
\rubric{\emph{Bén. 2.}} Que le Fils unique de Dieu daigne nous bénir et nous secourir.
\hspace{\specialcharhsep}\redrr Amen.
\end{paracol}

\smalltitle{Troisième leçon}

\begin{paracol}{2}
\rubric{Lector:} Jube, domne, benedícere.\\
\rubric{\emph{Benedictio 3.}} Spíritus \textit{Sancti} \textbf{grá}tia~\GreSpecial{*}
illúminet sensus et corda nostra.
\hspace{\specialcharhsep}\redrr Amen.
\switchcolumn
\rubric{Le lecteur:} Veuillez, Maître, bénir.\\
\rubric{\emph{Bén. 3.}} Que la grâce du Saint-Esprit illumine nos esprits et nos cœurs.
\hspace{\specialcharhsep}\redrr Amen.
\end{paracol}

\intermediatetitle{Deuxième nocturne}

\gscore{F1N2A1}{A}{4}
\translation{Qu'il est admirable ton nom, Seigneur, par toute la terre!}
\psalmus{8}{1}
\smallscore{F1N2A1}

\gscore{F1N2A2}{A}{5}
\translation{Tu sièges sur le trône, toi qui juges avec justice.}
\psalmus{9i}{8}
\smallscore{F1N2A2}

\gscore{F1N2A3}{A}{6}
\translation{Lève-toi, Seigneur, que l'homme ne triomphe pas.}
\psalmus{9ii}{1}
\smallscore{F1N2A3}

\versiculus{De ore leónis líbera me, Dómine.}{Et a cónibus unicónium humilitátem meam.}{De la gueule du lion, délivre-moi. Seigneur.}{Et des cornes des buffles, ma faiblesse.}

\smallscore{ORPN}
\translation{Notre Père...\\Et ne nous laisse pas entrer en tentation.\\Mais délivre-nous du Mal.}

\smalltitle{Absolution}

\begin{paracol}{2}
\rubric{\emph{Absolutio 2.}}
Ipsíus píetas et misericódi\textit{a nos} \textbf{ád}juvet,~\GreSpecial{*}
qui cum Patre et Spíritu Sancto vivit et regnat in sǽcula sæculórum.
\hspace{\specialcharhsep}\redrr Amen.
\switchcolumn
\rubric{\emph{Absolution 2.}}
Qu'il nous secoure par sa bonté et sa miséricorde, celui qui, avec le Père et le Saint-Esprit, vit et règne dans les siècles des siècles.
\hspace{\specialcharhsep}\redrr Amen.
\end{paracol}

\smalltitle{Quatrième leçon}

\begin{paracol}{2}
\rubric{Lector:} Jube, domne, benedícere.\\
\rubric{\emph{Benedictio 4.}} Deus Pa\textit{ter om}\textbf{ní}potens~\GreSpecial{*}
sit nobis propítius et clemens.
\hspace{\specialcharhsep}\redrr Amen.
\switchcolumn
\rubric{Le lecteur:} Veuillez, Maître, bénir.\\
\rubric{\emph{Bén. 4.}}
Que Dieu le Père tout-puissant soit pour nous propice et plein de clémence.
\hspace{\specialcharhsep}\redrr Amen.
\end{paracol}

\smalltitle{Cinquième leçon}

\begin{paracol}{2}
\rubric{Lector:} Jube, domne, benedícere.\\
\rubric{\emph{Benedictio 5.}} Chris\textit{tus per}\textbf{pé}tuæ~\GreSpecial{*}
det nobis gaúdia vitæ.
\hspace{\specialcharhsep}\redrr Amen.
\switchcolumn
\rubric{Le lecteur:} Veuillez, Maître, bénir.\\
\rubric{\emph{Bén. 5.}}
Que le Christ nous donne les joies de l'éternelle vie.
\hspace{\specialcharhsep}\redrr Amen.
\end{paracol}

\smalltitle{Sixième leçon}

\begin{paracol}{2}
\rubric{Lector:} Jube, domne, benedícere.\\
\rubric{\emph{Benedictio 6.}} Ignem su\textit{i a}\textbf{mó}ris~\GreSpecial{*}
accéndat Deus in córdibus nostris.
\hspace{\specialcharhsep}\redrr Amen.
\switchcolumn
\rubric{Le lecteur:} Veuillez, Maître, bénir.\\
\rubric{\emph{Bén. 6.}}
Que Dieu daigne allumer dans nos cœurs le feu de son amour.
\hspace{\specialcharhsep}\redrr Amen.
\end{paracol}

\intermediatetitle{Troisième nocturne}

\gscore{F1N3A1}{A}{7}
\translation{Pourquoi, Seigneur, te tenir à l'écart?}
\psalmus{9iii}{2}
\smallscore{F1N3A1}

\gscore{F1N3A2}{A}{8}
\translation{Lève-toi, Seigneur Dieu, que soit exaltée ta main.}
\psalmus{9iv}{5}
\smallscore{F1N3A2}

\gscore{F1N3A3}{A}{9}
\translation{Juste est le Seigneur et il aime la justice.}
\psalmus{10}{1}
\smallscore{F1N3A3}


\versiculus{Ne perdas cum impiis, Deus, ánimam meam.}{Et cum viris sánguinum vitam meam.}{Ne laisse pas mon âme se perdre avec les impies, ô Dieu.}{Ni ma vie avec les hommes de sang.}

\smallscore{ORPN}
\translation{Notre Père...\\Et ne nous laisse pas entrer en tentation.\\Mais délivre-nous du Mal.}

\smalltitle{Absolution}

\begin{paracol}{2}
\rubric{\emph{Absolutio 3.}}
A vínculis peccató\textit{rum nos}\textbf{tró}rum~\GreSpecial{*}
absólvat nos omnípotens et miséricors Dóminus.
\hspace{\specialcharhsep}\redrr Amen.
\switchcolumn
\rubric{\emph{Absolution 3.}}
Que le Dieu tout-puissant et miséricordieux daigne nous délivrer des liens de nos péchés.
\hspace{\specialcharhsep}\redrr Amen.
\end{paracol}

\smalltitle{Septième leçon}

\begin{paracol}{2}
\rubric{Lector:} Jube, domne, benedícere.\\
\rubric{\emph{Benedictio 7.}}
Evangé\textit{lica} \textbf{léc}tio~\GreSpecial{*}
sit nobis salus et protéctio.
\hspace{\specialcharhsep}\redrr Amen.
\switchcolumn
\rubric{Le lecteur:} Veuillez, Maître, bénir.\\
\rubric{\emph{Bén. 7.}}
Que la lecture du saint Évangile nous soit salut et protection.
\hspace{\specialcharhsep}\redrr Amen.
\end{paracol}

\smalltitle{Huitième leçon}

\begin{paracol}{2}
\rubric{Lector:} Jube, domne, benedícere.\\
\rubric{\emph{Benedictio 8.}}
Diví\textit{num au}\textbf{xí}lium~\GreSpecial{*}
máneat semper nobíscum.
\hspace{\specialcharhsep}\redrr Amen.
\switchcolumn
\rubric{Le lecteur:} Veuillez, Maître, bénir.\\
\rubric{\emph{Bén. 8.}}
Que le secours divin demeure toujours avec nous.
\hspace{\specialcharhsep}\redrr Amen.
\end{paracol}

\smalltitle{Neuvième leçon}

\begin{paracol}{2}
\rubric{Lector:} Jube, domne, benedícere.\\
\rubric{\emph{Benedictio 9.}}
Ad societátem cívium \textit{super}\textbf{nó}rum~\GreSpecial{*}
perdúcat nos Rex Angelórum.
\hspace{\specialcharhsep}\redrr Amen.
\switchcolumn
\rubric{Le lecteur:} Veuillez, Maître, bénir.\\
\rubric{\emph{Bén. 9.}}
Que le Roi des Anges nous fasse parvenir à la société des citoyens célestes.
\hspace{\specialcharhsep}\redrr Amen.
\end{paracol}


\end{document}