\documentclass[10pt, twoside, french]{article}


%%%%%%%%%%%% GEOMETRY
\usepackage{geometry}
\usepackage{fancyhdr}
\geometry{
	paperwidth=148mm,
	paperheight=210mm,
	inner=20mm,
	outer=12mm,
	top=15mm,
	bottom=12mm,
	headsep=2mm,
}
\pagestyle{empty}

%%%%%%%%%%%% LANGUAGE
\usepackage[nolocalmarks]{polyglossia}
\setdefaultlanguage[variant=french, frenchitemlabels=false]{french}

%%%%%%%%%%%% FONTS AND BASE STYLES
\usepackage{fontspec}
\setmainfont[Ligatures=TeX, Scale=1]{Charis}
\usepackage{paracol}
\usepackage[forcecompile]{gregoriotex}

%% No paragraph indentation
\setlength{\parindent}{0mm}

%% Macro to print rubrics
\newcommand{\rubric}[1]{\textcolor{gregoriocolor}{\emph{#1}}}

%% Macros to print V/ R/ A/ * + symbols in various contexts
\newcommand{\specialcharhsep}{3mm} % space after invoking R/ or V/ or A/ outside rubrics
\newcommand{\vv}{%
	{%
		\fontspec[Scale=1]{Charis}%
		℣.~%
		\nolinebreak[4]%
	}%
}
\newcommand{\redvv}{%
	\textcolor{gregoriocolor}%
	\vv%
	\hspace{\specialcharhsep}%
	\nolinebreak[4]%
}
\newcommand{\aarub}{%
	{%
		\fontspec[Scale=1]{Charis}%
		\Abar.~%
		\nolinebreak[4]%
	}%
}
\newcommand{\redaa}{%
	\textcolor{gregoriocolor}%
	\aarub%
	\hspace{\specialcharhsep}%
	\nolinebreak[4]%
}
\newcommand{\rr}{%
	{%
		\fontspec[Scale=1]{Charis}%
		℟.~%
		\nolinebreak[4]%
	}%
}
\newcommand{\redrr}{%
	\textcolor{gregoriocolor}%
	\rr%
	\hspace{\specialcharhsep}%
	\nolinebreak[4]%
}
\newcommand{\cc}{
	\textcolor{gregoriocolor}{
		\normalfont
		\fontspec[Scale=1]{FreeSerif}
		\symbol{"2720}
	}
}

%% Same special characters, for in-score use (<sp>V/ R/ A/ +</sp>)
\gresetspecial{V/}{\textcolor{gregoriocolor}{\fontspec[Scale=1]{Charis}℣.~}}
\gresetspecial{R/}{\textcolor{gregoriocolor}{\fontspec[Scale=1]{Charis}℟.~}}
\gresetspecial{A/}{\textcolor{gregoriocolor}{\fontspec[Scale=1]{Charis}\Abar.~}}
\gresetspecial{+}{{\fontspec[Scale=1]{FreeSerif}†~}}
\gresetspecial{*}{\gresixstar}
\gresetspecial{cross}{\textcolor{gregoriocolor}{\fontspec[Scale=1]{FreeSerif}\symbol{"2720}}}
\gresetspecial{labiacross}{\textcolor{gregoriocolor}{+}}

%% the asterisk as found in the mediants of text-only psalms
\newcommand{\psstar}{\GreSpecial{*}}
\newcommand{\pscross}{\GreSpecial{+}}

%% Macro to print versicles in two languages
\newcommand{\versiculus}[4]{%
	\begin{paracol}{2}%
	\par\redvv #1 \\ \redrr #2\par%
	\switchcolumn%
	\par\redvv #3 \\ \redrr #4\par%
	\end{paracol}%
}

%% Macro to print capitulum
\newcommand{\capitulum}[3]{%
	\smalltitle{Capitule}
	\begin{paracol}{2}%
	\rubric{#1}
	#2\\%
	\gresetinitiallines{0}%
	\gabcsnippet{(c3) <sp>R/</sp> De(h)o(h) <b>grá</b>(f)ti(e)as.(ef..) (::)}%
	\gresetinitiallines{1}%
	\switchcolumn
	#3\\
	\redrr Nous rendons grâces à Dieu.
	\end{paracol}%
}

%% Macro to print oratio
\newcommand{\oratio}[2]{%
	\versiculus{Orémus.\\#1}{Amen.}{Prions.\\#2}{Amen.}
}

%%%%%%%%%%%% GREGORIO CONFIG

%% \officepartannotation converts a letter (IHARPT) into the office part to be printed as annotation,
%% storing the result into \result.
\newcommand{\result}{}
\newcommand{\lookup}[3]{%
  \IfSubStr{#2}{#1}{ \renewcommand{\result}{#3} }{}%
}%
\newcommand{\officepartannotation}[1]{%
  \renewcommand{\result}{#1}%
  \lookup{#1}{T}{}%
  \lookup{#1}{H}{Hymn.}%
  \lookup{#1}{A}{Ant.}%
  \lookup{#1}{P}{}%
  \lookup{#1}{R}{Resp.}%
  \lookup{#1}{I}{Invit.}%
  \result%
}%

%% header capture setup for the mode
\newcommand{\defaultannotationshift}{-2mm}
\newcommand{\modeannotation}[1]{\greannotation{\hspace{\defaultannotationshift}\hspace{1mm}#1}}
\gresetheadercapture{mode}{modeannotation}{string}

%% outputs a score without annotations or initial
\newcommand{\smallscore}[1]{
	\gresetinitiallines{0}
	\gregorioscore{nocturnale-romanum/gabc/#1}
	\gresetinitiallines{1}
}

%% outputs a score with annotations and initial
\newcommand{\gscore}[3]{
	\greannotation[c]{
		\hspace{-1.4mm}
		\hspace{\defaultannotationshift}
		\officepartannotation{#2}#3
	}
	\gregorioscore{nocturnale-romanum/gabc/#1}
}

%% outputs a hymn with translation
\usepackage{multicol}
\setlength\columnseprule{0.4pt}
\setlength{\multicolsep}{6pt plus 2pt minus 1.5pt}
\newcommand{\hymnus}[2]{
	\smalltitle{Hymne}
	\gscore{#1}{H}{}
	\begin{multicols}{2}%
	\translation{#2}%
	\end{multicols}%
}

%% Initial style
\grechangestyle{initial}{\fontspec{Zallman Caps}\fontsize{28}{28}\selectfont}


%%%%%%%%%%%% TRANSLATION STYLE
\newcommand{\translation}[1]{
	\emph{#1}
}

%%%%%%%%%%%% PSALMODY STYLE
\usepackage{enumitem}
\usepackage{needspace}
%% We want to allow large inter-words space 
%% to avoid overfull boxes in two-columns rubrics.
\sloppy

\newcommand{\parallelitems}[2]{
	\begin{paracol}{2}
	\begin{itemize}[
		label=\null, 
		leftmargin=0pt, 
		itemindent=10pt, 
		labelsep=0pt, 
		labelwidth=0pt, 
		rightmargin=0pt, 
		parsep=0pt, 
		itemsep=0pt,
		topsep=-2mm]
	\input{nocturnale-romanum/psalmi/#1_#2.tex}
	\end{itemize}
	\switchcolumn
	\begin{itemize}[
		label=\null, 
		leftmargin=0pt, 
		itemindent=10pt, 
		labelsep=0pt, 
		labelwidth=0pt, 
		rightmargin=0pt, 
		parsep=0pt, 
		itemsep=0pt,
		topsep=-2mm]
	\input{psalmi_fr/#1.tex}
	\end{itemize}
	\end{paracol}
}

\newcommand{\psalmus}[2]{
	\needspace{4\baselineskip}
	\smalltitle{Psaume #1}
	\parallelitems{#1}{#2}
}

\newcommand{\magnificat}[1]{
	\needspace{4\baselineskip}
	\smalltitle{Magnificat}
	\parallelitems{magn}{#1}
}

%%%%%%%%%%%% TITLE STYLES

\newcommand{\smalltitle}[1]{
  \vspace{0.3\baselineskip}
  \par{\centering\textbf{#1}\par}
  \vspace{0.3\baselineskip}
}

\newcommand{\largetitle}[1]{
  \par{\centering\Huge\textsc{#1}\par}
}

\newcommand{\intermediatetitle}[1]{
  \par{\centering\Large\textsc{#1}\par}
}


%%%%%%%%%%%% GRAPHICS

\newcommand{\sep}{{\centering\greseparator{3}{20}\par}}


\begin{document}

\thispagestyle{empty}

\largetitle{Dimanches de Carême\\aux II\textsuperscript{ndes} Vêpres}

\vfill
%{\centering\includegraphics{TODO}\par}
\vfill

\smallscore{DIA_festivus}{}{}
\translation{\vv Dieu \cc venez à mon aide. \rr Seigneur, hâtez-vous de me secourir. Gloire au Père, au Fils, et au Saint-Esprit, comme il était au commencement, maintenant et toujours, et dans les siècles des siècles. Ainsi soit-il. Louange à vous, Seigneur, Roi d'éternelle gloire.}

\pagebreak

\gscore{F1A1}{A}{1}
\translation{Le Seigneur a dit à mon Seigneur : * Asseyez-Vous à ma droite.}
\rubric{On ne répète pas le texte de l'antienne au début du psaume.}
\psalmus{109}{7}
\rubric{On répète l'antienne, et on fait ainsi après chaque psaume.}
\gscore{F1A2}{A}{2}
\translation{Les œuvres du Seigneur sont grandes, * proportionnées à toutes Ses volontés.}
\psalmus{110}{3}
\gscore{F1A3}{A}{3}
\translation{Celui qui craint le Seigneur * met ses délices dans Ses commandements.}
\psalmus{111}{4g}
\gscore{F1A4}{A}{4}
\translation{Que le Nom du Seigneur * soit béni dans tous les siècles.}
\psalmus{112}{7}
\gscore{F1A5}{A}{5}
\translation{Notre Dieu est * dans le Ciel : tout ce qu'Il a voulu, Il l'a fait.}
\psalmus{113}{p}

\vspace{\baselineskip}

\smalltitle{Capitule}
\smalltitle{Premier dimanche}

\capitulum{2 Co 6: 1-2}{Fratres: Hortámur vos, ne in vácuum grátiam Dei recipi\textbf{á}tis.~\pscross{} Ait enim: Témpore accépto \textit{exau}\textbf{dí}vi te,~\psstar{} et in die salútis adjúvi te.}{Frères, nous vous exhortons à ne pas recevoir en vain la grâce de Dieu. Car il dit: «Au temps favorable, je t’ai exaucé et au jour du salut, je t’ai aidé.»}

\smalltitle{Deuxième dimanche}

\capitulum{1 Th 4: 1}{Fratres: Rogámus vos, et obsecrámus in Dómino \textbf{Je}su:~\pscross{} ut, quemádmodum accepístis a nobis, quómodo vos opórteat ambuláre, et pla\textit{cére} \textbf{De}o,~\psstar{} sic et ambulétis, ut abundétis magis.}{Frères, nous vous en prions et vous en conjurons par le Seigneur Jésus: ce que vous avez appris de nous, sur la manière dont il faut se conduire pour plaire à Dieu, pratiquez-le, afin d’avoir une grâce de plus en plus abondante.}

\smalltitle{Troisième dimanche}

\capitulum{Ep 5: 1-2}{Fratres: Estóte imitatóres Dei, sicut fílii ca\textbf{rís}simi:~\pscross{} et ambuláte in dilectióne, sicut et Christus diléxit nos, et trádidit semetíp\textit{sum} \textit{pro} \textbf{no}bis~\psstar{} oblatiónem et hóstiam Deo in odórem suavitátis.}{Frères, soyez des imitateurs de Dieu, comme des fils bien-aimés, et marchez dans la charité, à l’exemple du Christ qui nous a aimés et s’est livré lui-même pour nous, en offrande à Dieu et en sacrifice d’agréable odeur.}


\smalltitle{Quatrième dimanche}

\capitulum{Gal 4: 22-24}{Fratres: Scriptum est, quóniam Abraham duos fílios hábuit: unum de ancílla, et unum de \textbf{lí}bera:~\pscross{} sed qui de ancílla, secúndum carnem natus est: qui autem de líbera, per repro\textit{missi}\textbf{ó}nem:~\psstar{} quæ sunt per allegoríam dicta.}{Frères, il est écrit qu’Abraham eut deux fils, l’un de l’esclave, et l’autre de la femme libre. Mais celui de l’esclave naquit selon la chair ; et celui de la femme libre, naquit en vertu de la promesse. Cela a été dit par allégorie.}

\begin{paracol}{2}%
\gresetinitiallines{0}%
\gabcsnippet{(c3) <sp>R/</sp> De(h)o(h) <b>grá</b>(f)ti(e)as.(ef..) (::)}%
\gresetinitiallines{1}%
\switchcolumn
\redrr Nous rendons grâces à Dieu.
\end{paracol}%

\hymnus{Q1H}{Écoutez, Créateur bienveillant,\\
nos prières accompagnées de larmes,\\
répandues au milieu des jeûnes\\
de cette sainte Quarantaine.\\
\\
Vous qui scrutez le fond des cœurs,\\
vous connaissez notre faiblesse:\\
nous revenons à vous;\\
donnez-nous la grâce du pardon.\\
\\
Nous avons beaucoup péché;\\
pardonnez-nous à cause de notre aveu:\\
pour la gloire de votre Nom,\\
apportez le remède à nos langueurs.\\
\\
Faites que la résistance de notre corps\\
soit abattue par l’abstinence,\\
et que notre cœur soumis à un jeûne\\
spirituel ne se repaisse plus du péché.\\
\\
Exaucez-nous, Trinité bienheureuse,\\
accordez-nous, Unité simple,\\
que soit profitable à vos fidèles\\
le bienfait du jeûne.}

\smalltitle{Verset}
\versiculus{Ángelis suis Deus mandávit de te.}{Ut custódiant te in ómnibus viis tuis.}{Dieu a ordonné pour toi à Ses Anges.}{De te garder dans toutes tes voies.}

\intermediatetitle{Premier dimanche}

\gscore{Q1AM}{A}{}
\translation{Voici maintenant le temps favorable ; voici maintenant le jour du salut ; en ces jours donc, montrons-nous comme ministres de Dieu, en grande patience, dans les jeûnes, les veilles et une charité sincère.}
\magnificat{8}

\smalltitle{Oraison}

\versiculus{Dóminus vobíscum.}{Et cum spíritu tuo.}{Le Seigneur soit avec vous.}{Et avec votre esprit.}

\oratiowithoremus{Deus, qui Ecclésiam tuam ánnua quadragesimáli observatióne puríficas:~\pscross{} præsta famíliæ tuæ; ut, quod a te obtinére abstinéndo nítitur,~\psstar{} hoc bonis opéribus exsequátur. 
Per Dóminum nostrum Jesum Christum, Fílium tuum: qui tecum vivit et regnat in unitáte Spíritus Sancti, Deus, per ómnia sǽcula sæculórum.}{Dieu, qui purifiez votre Église par l’observance annuelle du Carême : aidez votre famille, afin que, ce qu’elle s’efforce d’obtenir de vous par l’abstinence, elle le réalise en bonnes œuvres.
Par Notre Seigneur Jésus Christ, Votre Fils, qui vit et règne avec Vous et le Saint-Esprit, Dieu, maintenant et pour les siècles des siècles.}

\rubric{Conclusion en dernière page.}

\vspace{\baselineskip}

\intermediatetitle{Deuxième dimanche}

\gscore{Q2AM}{A}{}
\pagebreak
\translation{La vision que vous avez eue, ne la dites à personne jusqu’à ce que le Fils de l’homme ressuscite d’entre les morts.}
\magnificat{1}

\smalltitle{Oraison}

\oratio{Deus, qui cónspicis omni nos virtúte destítui: intérius exteriúsque custódi;~\pscross{} ut ab ómnibus adversitátibus muniámur in córpore,~\psstar{} et a právis cogitatiónibus mundémur in mente. 
Per Dóminum.}{Dieu, qui nous voyez démunis de toute force, gardez-nous au dedans et au dehors, afin que notre corps soit défendu contre toute adversité, et notre âme purifiée des mauvaises pensées.
Par Notre Seigneur.}

\rubric{Conclusion en dernière page.}

\vfill

\sep

\vfill

\pagebreak

\intermediatetitle{Troisième dimanche}

\gscore{Q3AM}{A}{}
\translation{Élevant la voix, du milieu de la foule, une femme dit : Bienheureux le sein qui vous a porté et les mamelles qui vous ont allaité. Mais Jésus lui dit : Heureux plutôt ceux qui entendent la parole de Dieu et qui la gardent.}
\magnificat{8}

\smalltitle{Oraison}

\oratio{Quǽsumus, omnípotens Deus, vota humílium réspice:~\pscross{} atque ad defensiónem nostram,~\psstar{} déxteram tuæ majestátis exténde.
Per Dóminum.}{Nous vous le demandons, Dieu tout-puissant, agréez nos humbles prières et pour notre défense étendez le bras de votre majesté.
Par Notre Seigneur.}

\rubric{Conclusion en dernière page.}

\intermediatetitle{Quatrième dimanche}

\gscore{Q4AM}{A}{}
\translation{Jésus monta donc sur une montagne, et là il s’assit avec ses disciples.}
\magnificat{1}

\smalltitle{Oraison}

\oratio{Concéde, quǽsumus, omnípotens Deus:~\pscross{} ut, qui ex mérito nostræ actiónis afflígimur,~\psstar{} tuæ grátiæ consolatióne respirémus. 
Per Dóminum.}{Faites, s’il vous plaît, Dieu tout-puissant, que, justement affligés à cause de nos péchés, nous respirions par la consolation de votre grâce.
Par Notre Seigneur.}

\smalltitle{Conclusion}

\versiculus{Dóminus vobíscum.}{Et cum spíritu tuo.}{Le Seigneur soit avec vous.}{Et avec votre esprit.}

\smallscore{ORBDVA}{}{}
\translation{\vv Bénissons le Seigneur. \rr Nous rendons grâces à Dieu.}
\versiculus{Fidélium ánimæ per misericórdiam Dei requiéscant in pace.}{Amen.}{Que par la miséricorde de Dieu, les âmes des fidèles trépassés reposent en paix.}{Ainsi soit-il.}


\end{document}