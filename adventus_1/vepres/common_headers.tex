
%%%%%%%%%%%% GEOMETRY
\usepackage{geometry}
\geometry{
	paperwidth=148mm,
	paperheight=210mm,
	inner=15mm,
	outer=10mm,
	top=12mm,
	bottom=10mm,
	headsep=2mm,
}

%%%%%%%%%%%% LANGUAGE
\usepackage[nolocalmarks]{polyglossia}
\setdefaultlanguage[variant=french, frenchitemlabels=false]{french}

%%%%%%%%%%%% FONTS AND BASE STYLES
\usepackage{fontspec}
\setmainfont[Ligatures=TeX, Scale=1]{Charis SIL}
\usepackage{paracol}
\usepackage[forcecompile]{gregoriotex}

%% Macro to print rubrics
\newcommand{\rubric}[1]{\textcolor{gregoriocolor}{\emph{#1}}}

%% Macros to print V/ R/ A/ * + symbols in various contexts
\newcommand{\specialcharhsep}{3mm} % space after invoking R/ or V/ or A/ outside rubrics
\newcommand{\vv}{%
	{%
		\fontspec[Scale=1]{Charis SIL}%
		℣.%
		\nolinebreak[4]%
	}%
}
\newcommand{\redvv}{%
	\textcolor{gregoriocolor}%
	\vv%
	\hspace{\specialcharhsep}%
	\nolinebreak[4]%
}
\newcommand{\rr}{%
	{%
		\fontspec[Scale=1]{Charis SIL}%
		℟.%
		\nolinebreak[4]%
	}%
}
\newcommand{\redrr}{%
	\textcolor{gregoriocolor}%
	\rr%
	\hspace{\specialcharhsep}%
	\nolinebreak[4]%
}
\newcommand{\cc}{
	\textcolor{gregoriocolor}{
		\fontspec[Scale=1]{FreeSerif}
		\symbol{"2720}~
	}
}

%% Same special characters, for in-score use (<sp>V/ R/ A/ +</sp>)
\gresetspecial{V/}{\textcolor{gregoriocolor}{\fontspec[Scale=1]{Charis SIL}℣.~}}
\gresetspecial{R/}{\textcolor{gregoriocolor}{\fontspec[Scale=1]{Charis SIL}℟.~}}
\gresetspecial{A/}{\textcolor{gregoriocolor}{\fontspec[Scale=1]{Charis SIL}\Abar.~}}
\gresetspecial{+}{{\fontspec[Scale=1]{FreeSerif}†~}}
\gresetspecial{*}{\gresixstar}
\gresetspecial{cross}{\textcolor{gregoriocolor}{\fontspec[Scale=1]{FreeSerif}\symbol{"2720}}}
\gresetspecial{labiacross}{\textcolor{gregoriocolor}{+}}

%% the asterisk as found in the mediants of text-only psalms
\newcommand{\psstar}{\GreSpecial{*}}
\newcommand{\pscross}{\GreSpecial{+}}

%% Macro to print versicles in two languages
\newcommand{\versiculus}[4]{%
	\begin{paracol}{2}%
	\par\redvv #1 \\ \redrr #2\par%
	\switchcolumn%
	\par\redvv #3 \\ \redrr #4\par%
	\end{paracol}%
}

%% Macro to print capitulum
\newcommand{\capitulum}[3]{%
	\smalltitle{Capitule}
	\begin{paracol}{2}%
	\rubric{#1}
	#2
	\switchcolumn%
	#3
	\end{paracol}%
}

%% Macro to print oratio
\newcommand{\oratio}[2]{%
	\versiculus{Orémus.\\#1}{Amen.}{Prions.\\#2}{Amen.}
}

%%%%%%%%%%%% GREGORIO CONFIG

%% \officepartannotation converts a letter (IHARPT) into the office part to be printed as annotation,
%% storing the result into \result.
\newcommand{\result}{}
\newcommand{\lookup}[3]{%
  \IfSubStr{#2}{#1}{ \renewcommand{\result}{#3} }{}%
}%
\newcommand{\officepartannotation}[1]{%
  \renewcommand{\result}{#1}%
  \lookup{#1}{T}{}%
  \lookup{#1}{H}{Hymn.}%
  \lookup{#1}{A}{Ant.}%
  \lookup{#1}{P}{}%
  \lookup{#1}{R}{Resp.}%
  \lookup{#1}{I}{Invit.}%
  \result%
}%

%% header capture setup for the mode
\newcommand{\defaultannotationshift}{-2mm}
\newcommand{\modeannotation}[1]{\greannotation{\hspace{\defaultannotationshift}#1}}
\gresetheadercapture{mode}{modeannotation}{string}

%% outputs a score without annotations or initial
\newcommand{\smallscore}[1]{
	\gresetinitiallines{0}
	\gregorioscore{gabc/#1}
	\gresetinitiallines{1}
}

%% outputs a score with annotations and initial
\newcommand{\gscore}[3]{
	\greannotation[c]{
		\hspace{-1.4mm}
		\hspace{\defaultannotationshift}
		\officepartannotation{#2}#3
	}
	\gregorioscore{gabc/#1}
}

%% outputs a hymn with translation
\usepackage{multicol}
\setlength\columnseprule{0.4pt}
\setlength{\multicolsep}{6pt plus 2pt minus 1.5pt}
\newcommand{\hymnus}[2]{
	\gscore{#1}{H}{}
	\begin{multicols}{2}%
	\translation{#2}%
	\end{multicols}%
}

%% Initial style
\grechangestyle{initial}{\fontspec{Zallman Caps}\fontsize{28}{28}\selectfont}


%%%%%%%%%%%% TRANSLATION STYLE
\newcommand{\translation}[1]{
	\emph{#1}
}

%%%%%%%%%%%% PSALMODY STYLE
\usepackage{enumitem}
\usepackage{needspace}
%% We want to allow large inter-words space 
%% to avoid overfull boxes in two-columns rubrics.
\sloppy

\newcommand{\parallelitems}[2]{
	\begin{paracol}{2}
	\begin{itemize}[
		label=\null, 
		leftmargin=0pt, 
		itemindent=10pt, 
		labelsep=0pt, 
		labelwidth=0pt, 
		rightmargin=0pt, 
		parsep=0pt, 
		itemsep=0pt,
		topsep=-2mm]
	\input{psalmi_la/#1_#2.tex}
	\end{itemize}
	\switchcolumn
	\begin{itemize}[
		label=\null, 
		leftmargin=0pt, 
		itemindent=10pt, 
		labelsep=0pt, 
		labelwidth=0pt, 
		rightmargin=0pt, 
		parsep=0pt, 
		itemsep=0pt,
		topsep=-2mm]
	\input{psalmi_fr/#1.tex}
	\end{itemize}
	\end{paracol}
}

\newcommand{\psalmus}[2]{
	\vspace{0.5\baselineskip}
	\needspace{4\baselineskip}
	\smalltitle{Psaume #1}
	\vspace{2mm}
	\parallelitems{#1}{#2}
}

\newcommand{\magnificat}[1]{
	\vspace{0.5\baselineskip}
	\needspace{4\baselineskip}
	\smalltitle{Magnificat}
	\vspace{2mm}
	\parallelitems{magn}{#1}
}

%%%%%%%%%%%% TITLE STYLES

\newcommand{\smalltitle}[1]{
  \par{\centering\textbf{#1}\par}
}
