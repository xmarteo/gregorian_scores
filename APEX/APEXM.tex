% !TEX TS-program = lualatex
% !TEX encoding = UTF-8

\documentclass[11pt, twoside, french, openany]{book}


%%%%%%%%%%%% GEOMETRY
\usepackage{geometry}
\usepackage{fancyhdr}
\geometry{
	paperwidth=148mm,
	paperheight=210mm,
	inner=20mm,
	outer=12mm,
	top=15mm,
	bottom=12mm,
	headsep=2mm,
}
\pagestyle{empty}

%%%%%%%%%%%% LANGUAGE
\usepackage[nolocalmarks]{polyglossia}
\setdefaultlanguage[variant=french, frenchitemlabels=false]{french}

%%%%%%%%%%%% FONTS AND BASE STYLES
\usepackage{fontspec}
\setmainfont[Ligatures=TeX, Scale=1]{Charis}
\usepackage{paracol}
\usepackage[forcecompile]{gregoriotex}

%% No paragraph indentation
\setlength{\parindent}{0mm}

%% Macro to print rubrics
\newcommand{\rubric}[1]{\textcolor{gregoriocolor}{\emph{#1}}}

%% Macros to print V/ R/ A/ * + symbols in various contexts
\newcommand{\specialcharhsep}{3mm} % space after invoking R/ or V/ or A/ outside rubrics
\newcommand{\vv}{%
	{%
		\fontspec[Scale=1]{Charis}%
		℣.~%
		\nolinebreak[4]%
	}%
}
\newcommand{\redvv}{%
	\textcolor{gregoriocolor}%
	\vv%
	\hspace{\specialcharhsep}%
	\nolinebreak[4]%
}
\newcommand{\aarub}{%
	{%
		\fontspec[Scale=1]{Charis}%
		\Abar.~%
		\nolinebreak[4]%
	}%
}
\newcommand{\redaa}{%
	\textcolor{gregoriocolor}%
	\aarub%
	\hspace{\specialcharhsep}%
	\nolinebreak[4]%
}
\newcommand{\rr}{%
	{%
		\fontspec[Scale=1]{Charis}%
		℟.~%
		\nolinebreak[4]%
	}%
}
\newcommand{\redrr}{%
	\textcolor{gregoriocolor}%
	\rr%
	\hspace{\specialcharhsep}%
	\nolinebreak[4]%
}
\newcommand{\cc}{
	\textcolor{gregoriocolor}{
		\normalfont
		\fontspec[Scale=1]{FreeSerif}
		\symbol{"2720}
	}
}

%% Same special characters, for in-score use (<sp>V/ R/ A/ +</sp>)
\gresetspecial{V/}{\textcolor{gregoriocolor}{\fontspec[Scale=1]{Charis}℣.~}}
\gresetspecial{R/}{\textcolor{gregoriocolor}{\fontspec[Scale=1]{Charis}℟.~}}
\gresetspecial{A/}{\textcolor{gregoriocolor}{\fontspec[Scale=1]{Charis}\Abar.~}}
\gresetspecial{+}{{\fontspec[Scale=1]{FreeSerif}†~}}
\gresetspecial{*}{\gresixstar}
\gresetspecial{cross}{\textcolor{gregoriocolor}{\fontspec[Scale=1]{FreeSerif}\symbol{"2720}}}
\gresetspecial{labiacross}{\textcolor{gregoriocolor}{+}}

%% the asterisk as found in the mediants of text-only psalms
\newcommand{\psstar}{\GreSpecial{*}}
\newcommand{\pscross}{\GreSpecial{+}}

%% Macro to print versicles in two languages
\newcommand{\versiculus}[4]{%
	\begin{paracol}{2}%
	\par\redvv #1 \\ \redrr #2\par%
	\switchcolumn%
	\par\redvv #3 \\ \redrr #4\par%
	\end{paracol}%
}

%% Macro to print capitulum
\newcommand{\capitulum}[3]{%
	\smalltitle{Capitule}
	\begin{paracol}{2}%
	\rubric{#1}
	#2\\%
	\gresetinitiallines{0}%
	\gabcsnippet{(c3) <sp>R/</sp> De(h)o(h) <b>grá</b>(f)ti(e)as.(ef..) (::)}%
	\gresetinitiallines{1}%
	\switchcolumn
	#3\\
	\redrr Nous rendons grâces à Dieu.
	\end{paracol}%
}

%% Macro to print oratio
\newcommand{\oratio}[2]{%
	\versiculus{Orémus.\\#1}{Amen.}{Prions.\\#2}{Amen.}
}

%%%%%%%%%%%% GREGORIO CONFIG

%% \officepartannotation converts a letter (IHARPT) into the office part to be printed as annotation,
%% storing the result into \result.
\newcommand{\result}{}
\newcommand{\lookup}[3]{%
  \IfSubStr{#2}{#1}{ \renewcommand{\result}{#3} }{}%
}%
\newcommand{\officepartannotation}[1]{%
  \renewcommand{\result}{#1}%
  \lookup{#1}{T}{}%
  \lookup{#1}{H}{Hymn.}%
  \lookup{#1}{A}{Ant.}%
  \lookup{#1}{P}{}%
  \lookup{#1}{R}{Resp.}%
  \lookup{#1}{I}{Invit.}%
  \result%
}%

%% header capture setup for the mode
\newcommand{\defaultannotationshift}{-2mm}
\newcommand{\modeannotation}[1]{\greannotation{\hspace{\defaultannotationshift}\hspace{1mm}#1}}
\gresetheadercapture{mode}{modeannotation}{string}

%% outputs a score without annotations or initial
\newcommand{\smallscore}[1]{
	\gresetinitiallines{0}
	\gregorioscore{nocturnale-romanum/gabc/#1}
	\gresetinitiallines{1}
}

%% outputs a score with annotations and initial
\newcommand{\gscore}[3]{
	\greannotation[c]{
		\hspace{-1.4mm}
		\hspace{\defaultannotationshift}
		\officepartannotation{#2}#3
	}
	\gregorioscore{nocturnale-romanum/gabc/#1}
}

%% outputs a hymn with translation
\usepackage{multicol}
\setlength\columnseprule{0.4pt}
\setlength{\multicolsep}{6pt plus 2pt minus 1.5pt}
\newcommand{\hymnus}[2]{
	\smalltitle{Hymne}
	\gscore{#1}{H}{}
	\begin{multicols}{2}%
	\translation{#2}%
	\end{multicols}%
}

%% Initial style
\grechangestyle{initial}{\fontspec{Zallman Caps}\fontsize{28}{28}\selectfont}


%%%%%%%%%%%% TRANSLATION STYLE
\newcommand{\translation}[1]{
	\emph{#1}
}

%%%%%%%%%%%% PSALMODY STYLE
\usepackage{enumitem}
\usepackage{needspace}
%% We want to allow large inter-words space 
%% to avoid overfull boxes in two-columns rubrics.
\sloppy

\newcommand{\parallelitems}[2]{
	\begin{paracol}{2}
	\begin{itemize}[
		label=\null, 
		leftmargin=0pt, 
		itemindent=10pt, 
		labelsep=0pt, 
		labelwidth=0pt, 
		rightmargin=0pt, 
		parsep=0pt, 
		itemsep=0pt,
		topsep=-2mm]
	\input{nocturnale-romanum/psalmi/#1_#2.tex}
	\end{itemize}
	\switchcolumn
	\begin{itemize}[
		label=\null, 
		leftmargin=0pt, 
		itemindent=10pt, 
		labelsep=0pt, 
		labelwidth=0pt, 
		rightmargin=0pt, 
		parsep=0pt, 
		itemsep=0pt,
		topsep=-2mm]
	\input{psalmi_fr/#1.tex}
	\end{itemize}
	\end{paracol}
}

\newcommand{\psalmus}[2]{
	\needspace{4\baselineskip}
	\smalltitle{Psaume #1}
	\parallelitems{#1}{#2}
}

\newcommand{\magnificat}[1]{
	\needspace{4\baselineskip}
	\smalltitle{Magnificat}
	\parallelitems{magn}{#1}
}

%%%%%%%%%%%% TITLE STYLES

\newcommand{\smalltitle}[1]{
  \vspace{0.3\baselineskip}
  \par{\centering\textbf{#1}\par}
  \vspace{0.3\baselineskip}
}

\newcommand{\largetitle}[1]{
  \par{\centering\Huge\textsc{#1}\par}
}

\newcommand{\intermediatetitle}[1]{
  \par{\centering\Large\textsc{#1}\par}
}


%%%%%%%%%%%% GRAPHICS

\newcommand{\sep}{{\centering\greseparator{3}{20}\par}}


\begin{document}

% Marche pour Matthias (24 février) 
% 	pas Marc (25 avril) APTP 
% 	pas Jean Porte Latine (6 mai) APTP
% 	Pas Philippe et Jacques (1er mai/ 11 mai) APTP
% Barnabé (11 juin)
% 	pas P&P (29 juin) psalmodie ok mais RR propres
% Jacques (25 juillet)
% Barty (24 aout)
% Matthieu (21 septembre)
% Luc (18 octobre)
% Simon & Jude (28 octobre)
% 	pas André (30 nov) tout est propre
% Thomas (21 déc)
% 	pas Jean (27 déc) répons propres

\null\newpage

\feast{APEX}{Commune Apostolorum\\et Evangelistarum\\extra Tempus Paschale}
	{}{}{2}{}
	{}{}{}
	{}
	{}

\rubric{Ouverture de l'Office et psaume invitatoire, à l'Ordinaire, avec l'invitatoire ci-dessous.}

\gscore{APEXIa}{I}{}{Regem Apostolorum\idxnewline(tonus solemnis)}
\translation{Le Seigneur, roi des apôtres, venez, adorons-le.}

\gscore{APEXH}{H}{}{Aeterna Christi munera\idxnewline(pro Apostolis)}

\begin{multicols}{2}
Chantons avec des cœurs joyeux\\
les bienfaits éternels du Christ,\\
la gloire des Apôtres,\\
palmes et hymnes mérités.\\
~\\
Ils sont les princes de l’Église,\\
victorieux chefs de ses combats,\\
les soldats de la cour céleste,\\
et la vraie lumière du monde.\\

La foi généreuse des Saints,\\
l’invincible espérance de ceux qui croient,\\
la parfaite charité du Christ,\\
voilà ce qui écrase le tyran du monde.\\
~\\
En eux triomphe la gloire du Père,\\
en eux triomphe le Fils,\\
en eux triomphe la volonté de l’Esprit,\\
le ciel est rempli de joie.
\end{multicols}

\intermediatetitle{Premier noctune}

\gscore{APEXN1A1}{A}{1}{In omnem terram}
\translation{Dans toute la terre, leur bruit s’est répandu, et leurs paroles jusqu’aux confins du monde.}
\psalmus{18}{2}
\newpage
\gscore{APEXN1A2}{A}{2}{Clamaverunt justi}
\translation{Les justes ont crié, et le Seigneur les a exaucés.}
\psalmus{33}{7}
\gscore{APEXN1A3}{A}{3}{Constitues eos principes}
\translation{Tu les établis princes sur toute la terre ; ils se souviendront de ton nom, Seigneur, dans la suite des générations.}
\psalmus{44}{7}

\twocoltext{
	\versiculus{In omnem terram exívit sonus eórum.}{Et in fines orbis terræ verba eórum.}
}{
	\versiculus{Leur bruit s’est répandu dans toute la terre.}{Et leurs paroles jusqu’aux confins du monde.}
}

\rubric{Absolution et Bénédictions à l'Ordinaire. Lectures au Lectionnaire.}

\gscore{APEXN1R1}{R}{1}{Ecce ego mitto vos sicut}
\translation{Voici que je vous envoie comme des brebis au milieu des loups, dit le Seigneur : Soyez donc prudents comme les serpents, et simple, comme les colombes. Pendant que vous avez la lumière croyez en la lumière, afin que vous soyez des enfants de lumière.}
\gscore{APEXN1R2}{R}{2}{Tollite jugum meum super vos}
\translation{Prenez mon joug sur vous, dit le Seigneur, et apprenez de moi que je suis doux et humble de cœur : Car mon joug est doux, et mon fardeau léger. Et vous trouverez du repos pour vos âmes.}
\gscore{APEXN1R3}{R}{3}{Dum steteritis ante reges}
\translation{Lorsque vous serez conduits devant les rois et les gouverneurs, ne pensez ni comment, ni ce que vous devrez dire : Il vous sera donné en effet, à l’heure même, ce que vous devrez dire. Car ce n’est pas vous qui parlez, mais l’Esprit de votre Père qui parle en vous.}

\intermediatetitle{Deuxième noctune}

\gscore{APEXN2A1}{A}{4}{Principes populorum}
\translation{Les chefs des peuples se sont rassemblés : c'est le peuple du Dieu d'Abraham.}
\psalmus{46}{8}
\gscore{APEXN2A2}{A}{5}{Dedisti hereditatem}
\translation{Tu as donné un héritage à ceux qui craignent ton nom, Seigneur.}
\psalmus{60}{8}
\gscore{APEXN2A3}{A}{6}{Annuntiaverunt opera Dei}
\translation{Ils ont annoncé les œuvres de Dieu, et ils ont compris les choses qu’il a faites.}
\psalmus{63}{8}

\twocoltext{
	\versiculus{Constítues eos príncipes super omnem terram.}{Mémores erunt nóminis tui, Domine.}
}{
	\versiculus{Tu les établis princes sur toute la terre.}{Ils se souviendront de ton nom, Seigneur.}
}

\rubric{Absolution et Bénédictions à l'Ordinaire. Lectures hagiographiques.}

\gscore{APEXN2R1}{R}{4}{Vidi conjunctos viros}
\translation{Je vis des hommes assemblés, ayant des vêtements splendides, et l’Ange du Seigneur me parla, disant :
Ceux-ci sont des hommes saints, devenus les amis de Dieu.
Je vis un ange de Dieu, fort, et volant au milieu du ciel, il criait d’une voix puissante et proclamait.
Ceux-ci sont des hommes saints, devenus les amis de Dieu.}
\gscore{APEXN2R2}{R}{5}{Beati eritis cum maledixerint}
\translation{Vous êtes heureux lorsque les hommes vous maudissent et vous persécutent et disent faussement toute sorte de mal de vous, à cause de moi :
Réjouissez-vous et tressaillez de joie, car votre récompense est grande dans les cieux.
Lorsque les hommes vous haïront, vous éloigneront, vous injurieront, et rejetteront votre nom comme mauvais à cause du Fils de l’homme.
Réjouissez-vous et tressaillez de joie, car votre récompense est grande dans les cieux.}
\gscore{APEXN2R3}{R}{6}{Isti sunt triumphatores}
\translation{Ceux-ci sont des triomphateurs et des amis de Dieu ; méprisant les ordres des princes, ils ont mérité d’éternelles récompenses :
Maintenant ils sont couronnés et reçoivent la palme.
Ce sont ceux qui sont venus de la grande tribulation, et qui ont lavé et blanchi leurs robes dans le sang de l’Agneau.}

\intermediatetitle{Troisième noctune}

\gscore{APEXN3A1}{A}{7}{Exaltabuntur cornua justi}
\translation{Elle sera élevée, la puissance des justes. (Allelúia.)}
\psalmus{74}{6}
\gscore{APEXN3A2}{A}{8}{Lux orta \emph{cum} Alleluia}
\translation{Une lumière s’est levée pour le juste, (alléluia) : une joie pour les hommes droits de cœur. (Allelúia.)}
\rubric{Septuagésime et Carême:}
\gscore{0806N3A2}{A}{8}{Lux orta \emph{sine} Alleluia}
\psalmus{96}{6}
\gscore{APEXN3A3}{A}{9}{Custodiebant}
\translation{Ils gardaient les témoignages du Seigneur et ses préceptes. (Allelúia.)}
\psalmus{98}{4e}

\twocoltext{
	\versiculus{Nimis honoráti sunt amíci tui, Dómine.}{Nimis confortátus est principátus eórum.}
}{
	\versiculus{Tes amis ont été grandement honorés, ô Dieu.}{Leur autorité de princes a été puissamment établie.}
}

\rubric{Absolution et Bénédictions à l'Ordinaire. Lectures à l'Homéliaire.}

\gscore{APEXN3R1}{R}{7}{Isti sunt qui viventes}
\translation{Ce sont ceux-ci qui, tandis qu’ils vivaient dans la chair, ont planté l’Église dans leur sang : ils ne sont pas de la terre : leurs corps sont mis à part,
eux dont les mérites sont aux cieux : ils sont égaux à l'âme des Saints.
Leur voix a retenti par toute la terre, et leurs paroles jusqu’aux extrémités du monde.}
\gscore{APEXN3R2}{R}{8}{Isti sunt viri sancti}
\translation{Ceux-ci sont des hommes saints que Dieu a choisis dans une charité sincère, et il leur a donné une gloire éternelle :
L’Église est éclairée par leur doctrine comme la lune est éclairée par le soleil.
Les Saints, par la foi, ont vaincu des royaumes et pratiqué la justice.}

\rubric{Te Deum à l'Ordinaire, oraison ci-dessous, et conclusion à l'Ordinaire.}

\rubric{24 février, saint Matthias}

\twocoltext{
Deus, qui beátum Matthíam Apostolórum tuórum collégio sociásti: tríbue, quǽsumus; ut, ejus interventióne, tuæ circa nos pietátis semper víscera sentiámus.
Per Dóminum.
}{
Dieu, qui as agrégé le bienheureux Matthias au collège de tes apôtres, accorde à notre demande, que, par son intercession, nous sentions toujours l’action de ton cœur rempli pour nous de miséricorde.
}

\rubric{11 juin, saint Barnabé}

\twocoltext{
Deus, qui nos beáti Bárnabæ Apóstoli tui méritis et intercessióne lætíficas:~
concéde propítius; ut, qui tua per eum benefícia póscimus, dono tuæ grátiæ~
consequámur.
Per Dóminum.
}{
Dieu, tu nous donnes un motif de joie dans les mérites et l’intercession du bienheureux Barnabé, ton apôtre : accorde-nous avec bonté, qu’en recourant à cette intercession pour solliciter tes bienfaits, nous les obtenions au moyen de ta grâce.
}

\rubric{25 juillet, saint Jacques}

\twocoltext{
Esto, Dómine, plebi tuæ sanctificátor et custos: ut, Apóstoli tui Jacóbi muníta præsídiis, et conversatióne tibi pláceat, et secúra mente desérviat.
Per Dóminum.
}{
Seigneur, sois le sanctificateur de ton peuple et son gardien, afin qu’aidé par l’assistance de ton apôtre Jacques, il mène une vie qui te soit agréable et te serve avec tranquillité et confiance.
}

\rubric{24 août, saint Barthélémy}

\twocoltext{
Omnípotens sempitérne Deus, qui hujus diéi venerándam sanctámque lætítiam in beáti Apóstoli tui Bartholomǽi festivitáte tribuísti: da Ecclésiæ tuæ, quǽsumus; et amáre quod crédidit, et prædicáre quod dócuit.
Per Dóminum.
}{
Dieu tout-puissant et éternel, de qui nous vient la religieuse et sainte joie que nous éprouvons à célébrer aujourd’hui la fête de ton bienheureux apôtre Barthélemy, accorde à ton Église, nous t'en prions, la grâce d’aimer ce qu’il a cru et de prêcher ce qu’il a enseigné.
}

\rubric{21 septembre, saint Matthieu}

\twocoltext{
Beáti Apóstoli et Evangelístæ Matthǽi, Dómine, précibus adjuvémur: ut, quod possibílitas nostra non óbtinet, ejus nobis intercessióne donétur. 
Per Dóminum.
}{
Que les prières du bienheureux apôtre et évangéliste Matthieu nous viennent en aide, Seigneur, afin que les grâces que notre insuffisance ne peut obtenir nous soient accordées par son intercession.
}

\rubric{18 octobre, saint Luc}

\twocoltext{
Intervéniat pro nobis, quǽsumus, Dómine, sanctus tuus Lucas Evangelísta: qui crucis mortificatiónem júgiter in suo córpore, pro tui nóminis honóre, portávit.
Per Dóminum.
}{
Nous t'en prions, Seigneur, que ton saint évangéliste Luc intercède pour nous, lui qui n’a jamais cessé de porter dans son corps la mortification de la croix, pour l’honneur de ton nom.
}

\rubric{28 octobre, saints Simon et Jude}

\twocoltext{
Deus, qui nos per beátos Apóstolos tuos Simónem et Judam ad agnitiónem tui nóminis veníre tribuísti: da nobis eórum glóriam sempitérnam et proficiéndo celebráre, et celebrándo profícere.
Per Dóminum.
}{
Dieu, tu nous as accordé la grâce de parvenir à la connaissance de ton nom par tes bienheureux apôtres Simon et Jude : fais qu’en progressant, nous célébrions leur gloire éternelle, et qu'en la célébrant nous progressions.
}

\rubric{21 décembre, saint Thomas}

\twocoltext{
Da nobis, quǽsumus, Dómine, beáti Apóstoli tui Thomæ solemnitátibus gloriári: ut ejus semper et patrocíniis sublevémur; et fidem cóngrua devotióne sectémur.
Per Dóminum.
}{
Donne-nous, Seigneur, nous te le demandons, de nous glorifier des solennités de ton bienheureux apôtre Thomas, afin que nous soyons toujours soutenus par son patronage, et que nous cultivions la foi avec la dévotion qui convient.
}

\end{document}

 