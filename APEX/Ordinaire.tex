% !TEX TS-program = lualatex
% !TEX encoding = UTF-8

\documentclass[11pt, twoside, french, openany]{book}


%%%%%%%%%%%% GEOMETRY
\usepackage{geometry}
\usepackage{fancyhdr}
\geometry{
	paperwidth=148mm,
	paperheight=210mm,
	inner=20mm,
	outer=12mm,
	top=15mm,
	bottom=12mm,
	headsep=2mm,
}
\pagestyle{empty}

%%%%%%%%%%%% LANGUAGE
\usepackage[nolocalmarks]{polyglossia}
\setdefaultlanguage[variant=french, frenchitemlabels=false]{french}

%%%%%%%%%%%% FONTS AND BASE STYLES
\usepackage{fontspec}
\setmainfont[Ligatures=TeX, Scale=1]{Charis}
\usepackage{paracol}
\usepackage[forcecompile]{gregoriotex}

%% No paragraph indentation
\setlength{\parindent}{0mm}

%% Macro to print rubrics
\newcommand{\rubric}[1]{\textcolor{gregoriocolor}{\emph{#1}}}

%% Macros to print V/ R/ A/ * + symbols in various contexts
\newcommand{\specialcharhsep}{3mm} % space after invoking R/ or V/ or A/ outside rubrics
\newcommand{\vv}{%
	{%
		\fontspec[Scale=1]{Charis}%
		℣.~%
		\nolinebreak[4]%
	}%
}
\newcommand{\redvv}{%
	\textcolor{gregoriocolor}%
	\vv%
	\hspace{\specialcharhsep}%
	\nolinebreak[4]%
}
\newcommand{\aarub}{%
	{%
		\fontspec[Scale=1]{Charis}%
		\Abar.~%
		\nolinebreak[4]%
	}%
}
\newcommand{\redaa}{%
	\textcolor{gregoriocolor}%
	\aarub%
	\hspace{\specialcharhsep}%
	\nolinebreak[4]%
}
\newcommand{\rr}{%
	{%
		\fontspec[Scale=1]{Charis}%
		℟.~%
		\nolinebreak[4]%
	}%
}
\newcommand{\redrr}{%
	\textcolor{gregoriocolor}%
	\rr%
	\hspace{\specialcharhsep}%
	\nolinebreak[4]%
}
\newcommand{\cc}{
	\textcolor{gregoriocolor}{
		\normalfont
		\fontspec[Scale=1]{FreeSerif}
		\symbol{"2720}
	}
}

%% Same special characters, for in-score use (<sp>V/ R/ A/ +</sp>)
\gresetspecial{V/}{\textcolor{gregoriocolor}{\fontspec[Scale=1]{Charis}℣.~}}
\gresetspecial{R/}{\textcolor{gregoriocolor}{\fontspec[Scale=1]{Charis}℟.~}}
\gresetspecial{A/}{\textcolor{gregoriocolor}{\fontspec[Scale=1]{Charis}\Abar.~}}
\gresetspecial{+}{{\fontspec[Scale=1]{FreeSerif}†~}}
\gresetspecial{*}{\gresixstar}
\gresetspecial{cross}{\textcolor{gregoriocolor}{\fontspec[Scale=1]{FreeSerif}\symbol{"2720}}}
\gresetspecial{labiacross}{\textcolor{gregoriocolor}{+}}

%% the asterisk as found in the mediants of text-only psalms
\newcommand{\psstar}{\GreSpecial{*}}
\newcommand{\pscross}{\GreSpecial{+}}

%% Macro to print versicles in two languages
\newcommand{\versiculus}[4]{%
	\begin{paracol}{2}%
	\par\redvv #1 \\ \redrr #2\par%
	\switchcolumn%
	\par\redvv #3 \\ \redrr #4\par%
	\end{paracol}%
}

%% Macro to print capitulum
\newcommand{\capitulum}[3]{%
	\smalltitle{Capitule}
	\begin{paracol}{2}%
	\rubric{#1}
	#2\\%
	\gresetinitiallines{0}%
	\gabcsnippet{(c3) <sp>R/</sp> De(h)o(h) <b>grá</b>(f)ti(e)as.(ef..) (::)}%
	\gresetinitiallines{1}%
	\switchcolumn
	#3\\
	\redrr Nous rendons grâces à Dieu.
	\end{paracol}%
}

%% Macro to print oratio
\newcommand{\oratio}[2]{%
	\versiculus{Orémus.\\#1}{Amen.}{Prions.\\#2}{Amen.}
}

%%%%%%%%%%%% GREGORIO CONFIG

%% \officepartannotation converts a letter (IHARPT) into the office part to be printed as annotation,
%% storing the result into \result.
\newcommand{\result}{}
\newcommand{\lookup}[3]{%
  \IfSubStr{#2}{#1}{ \renewcommand{\result}{#3} }{}%
}%
\newcommand{\officepartannotation}[1]{%
  \renewcommand{\result}{#1}%
  \lookup{#1}{T}{}%
  \lookup{#1}{H}{Hymn.}%
  \lookup{#1}{A}{Ant.}%
  \lookup{#1}{P}{}%
  \lookup{#1}{R}{Resp.}%
  \lookup{#1}{I}{Invit.}%
  \result%
}%

%% header capture setup for the mode
\newcommand{\defaultannotationshift}{-2mm}
\newcommand{\modeannotation}[1]{\greannotation{\hspace{\defaultannotationshift}\hspace{1mm}#1}}
\gresetheadercapture{mode}{modeannotation}{string}

%% outputs a score without annotations or initial
\newcommand{\smallscore}[1]{
	\gresetinitiallines{0}
	\gregorioscore{nocturnale-romanum/gabc/#1}
	\gresetinitiallines{1}
}

%% outputs a score with annotations and initial
\newcommand{\gscore}[3]{
	\greannotation[c]{
		\hspace{-1.4mm}
		\hspace{\defaultannotationshift}
		\officepartannotation{#2}#3
	}
	\gregorioscore{nocturnale-romanum/gabc/#1}
}

%% outputs a hymn with translation
\usepackage{multicol}
\setlength\columnseprule{0.4pt}
\setlength{\multicolsep}{6pt plus 2pt minus 1.5pt}
\newcommand{\hymnus}[2]{
	\smalltitle{Hymne}
	\gscore{#1}{H}{}
	\begin{multicols}{2}%
	\translation{#2}%
	\end{multicols}%
}

%% Initial style
\grechangestyle{initial}{\fontspec{Zallman Caps}\fontsize{28}{28}\selectfont}


%%%%%%%%%%%% TRANSLATION STYLE
\newcommand{\translation}[1]{
	\emph{#1}
}

%%%%%%%%%%%% PSALMODY STYLE
\usepackage{enumitem}
\usepackage{needspace}
%% We want to allow large inter-words space 
%% to avoid overfull boxes in two-columns rubrics.
\sloppy

\newcommand{\parallelitems}[2]{
	\begin{paracol}{2}
	\begin{itemize}[
		label=\null, 
		leftmargin=0pt, 
		itemindent=10pt, 
		labelsep=0pt, 
		labelwidth=0pt, 
		rightmargin=0pt, 
		parsep=0pt, 
		itemsep=0pt,
		topsep=-2mm]
	\input{nocturnale-romanum/psalmi/#1_#2.tex}
	\end{itemize}
	\switchcolumn
	\begin{itemize}[
		label=\null, 
		leftmargin=0pt, 
		itemindent=10pt, 
		labelsep=0pt, 
		labelwidth=0pt, 
		rightmargin=0pt, 
		parsep=0pt, 
		itemsep=0pt,
		topsep=-2mm]
	\input{psalmi_fr/#1.tex}
	\end{itemize}
	\end{paracol}
}

\newcommand{\psalmus}[2]{
	\needspace{4\baselineskip}
	\smalltitle{Psaume #1}
	\parallelitems{#1}{#2}
}

\newcommand{\magnificat}[1]{
	\needspace{4\baselineskip}
	\smalltitle{Magnificat}
	\parallelitems{magn}{#1}
}

%%%%%%%%%%%% TITLE STYLES

\newcommand{\smalltitle}[1]{
  \vspace{0.3\baselineskip}
  \par{\centering\textbf{#1}\par}
  \vspace{0.3\baselineskip}
}

\newcommand{\largetitle}[1]{
  \par{\centering\Huge\textsc{#1}\par}
}

\newcommand{\intermediatetitle}[1]{
  \par{\centering\Large\textsc{#1}\par}
}


%%%%%%%%%%%% GRAPHICS

\newcommand{\sep}{{\centering\greseparator{3}{20}\par}}


\begin{document}

\gresetnabc{1}{invisible} % none of the chants of the Ordinary really needs neumes

\null\vfill
\feast{OR}{Ordinaire de l'Office Divin\\à Matines}
	{}{}{1}{}{}{}{}{}{}
\vfill
\pagebreak

\feast{OR}{Ordinaire de l'Office Divin\\à Matines}
	{}{}{1}{}{}{}{}{}{}

\intermediatetitle{Prières avant l'Office}

\twocoltext{
\lettrine{A}{peri}, Dómine, os meum ad benedicéndum nomen sanctum tuum:
munda quoque cor meum ab ómnibus vanis, pervérsis et aliénis cogitatiónibus; 
intelléctum illúmina, afféctum inflámma, ut digne, atténte ac devóte
hoc officium recitáre váleam, et exaudíri mérear
ante conspéctum divinæ Majestátis tuæ. Per Christum Dóminum nostrum. Amen.
}{
\lettrine{O}{uvre} ma bouche, Seigneur, pour bénir ton saint nom, purifie mon cœur de toute pensée vaine, perverse et profane. Illumine mon intelligence, enflamme mon cœur, pour que je puisse réciter cet office dignement, attentivement et pieusement, et mériter d’être exaucé en présence de ta majesté divine. Par le Christ notre Seigneur. Amen.
}

\vspace{\baselineskip}

\twocoltext{
\lettrine{D}{ómine}, in unióne illíus divínæ intentiónis,
qua ipse in terris laudes Deo persolvísti,
has tibi horas \rubric{(vel} hanc tibi horam\rubric{)} persólvo.
}{
\lettrine{S}{eigneur}, en m'unissant à la divine intention avec laquelle, sur terre, tu offrais à Dieu le service de louange, je t'offre cette (ces) heure(s).
}

\vspace{2\baselineskip}

\sep

\vspace{2\baselineskip}

\twocoltext{
\lettrine{P}{ater noster}, qui es in cælis, sanctificétur nomen tuum.
Advéniat regnum tuum. Fiat volúntas tua, sicut in cælo et in terra.
Panem nostrum quotidiánum da nobis hódie.
Et dimítte nobis débita nostra, sicut et nos dimíttimus debitóribus nostris.
Et ne nos indúcas in tentatiónem: sed líbera nos a malo. Amen.
}{
\lettrine{N}{otre Père}, qui es aux cieux, que ton nom soit sanctifié, que ton règne vienne,
que ta volonté soit faite sur la terre comme au ciel.
Donne-nous aujourd’hui notre pain de ce jour. Pardonne-nous nos offenses, comme nous pardonnons aussi à ceux qui nous ont offensés.
Et ne nous laisse pas entrer en tentation mais délivre-nous du Mal. Amen.
}

\vspace{\baselineskip}

\twocoltext{
\lettrine{A}{ve María}, grátia plena, Dóminus tecum:
benedícta tu in muliéribus, et benedíctus fructus ventris tui Jesus.
Sancta María, Mater Dei, ora pro nobis peccatóribus,
nunc et in hora mortis nostræ. Amen.
}{
\lettrine{J}{e vous salue Marie}, pleine de grâce, le Seigneur est avec vous.
Vous êtes bénie entre toutes les femmes et Jésus, le fruit de vos entrailles, est béni.
Sainte Marie, Mère de Dieu, priez pour nous pauvres pécheurs, maintenant et à l’heure de notre mort.
Amen.
}

\vspace{\baselineskip}

\twocoltext{
\lettrine{C}{redo in Deum}, Patrem omnipoténtem, Creatórem cæli et terræ.
Et in Jesum Christum, Fílium ejus únicum, Dóminum nostrum,
qui concéptus est de spíritu Sancto, natus ex María Virgine,
passus sub Póntio Piláto, crucifixus, mórtuus et sepúltus:
descéndit ad ínferos: tértia die resurréxit a mórtuis;
ascéndit ad cælos, sedet ad déxteram Patris omnipoténtis:
inde ventúrus est judicáre vivos et mórtuos.
Credo in Spíritum sanctum, sanctam Ecclésiam cathólicam,
Sanctórum communiónem, remissiónem peccatórum,
carnis resurrectiónem, vitam ætérnam. Amen.
}{
\lettrine{J}{e crois en Dieu}, le Père tout-puissant, créateur du ciel et de la terre.
Et en Jésus Christ, son Fils unique, notre Seigneur;
qui a été conçu du Saint Esprit, est né de la Vierge Marie,
a souffert sous Ponce Pilate, a été crucifié,
est mort et a été enseveli, est descendu aux enfers;
le troisième jour est ressuscité des morts,
est monté aux cieux, est assis à la droite de Dieu le Père tout-puissant,
d’où il viendra juger les vivants et les morts.
Je crois en l’Esprit Saint, à la sainte Église catholique, à la communion des saints,
à la rémission des péchés, à la résurrection de la chair, à la vie éternelle. Amen.
}

\pagebreak

\intermediatetitle{Ouverture}

\rubric{Le dimanche et aux fêtes à trois nocturnes:}
\gscore{ORIa}{T}{}{A@Domine labia mea!Tonus festivus}
\translation{Seigneur, ouvre mes lèvres, et ma bouche annoncera ta louange. Dieu, viens à mon aide, Seigneur, viens vite à mon secours. Gloire au Père, au Fils, et au Saint-Esprit, comme il était au commencement, maintenant et toujours, dans les siècles des siècles. Amen. Alléluia. (\rubric{Septuagésime et Carême} Louange à toi, Seigneur, Roi d'éternelle gloire.)}

\pagebreak

\rubric{Aux féries, vigiles et fêtes à un nocturne:}
\gscore{ORIc}{T}{}{A@Domine labia mea!Tonus ferialis}

\vspace{\baselineskip}

\rubric{L'ouverture est omise dans l'Office des Défunts, au Triduum Pascal, et à l'Épiphanie.}

\newpage

\intermediatetitle{Invitatoire}

\rubric{Le chantre chante l'invitatoire au Propre ou au Commun, et tous le répètent. Il est chanté après les versets impairs du psaume qui suit, et sa deuxième partie après les versets pairs.}

\psalmus{94}{VLrepet}

\vspace{\baselineskip}

\rubric{L'invitatoire est omis au Triduum Pascal et à l'Épiphanie.}

\vfill

\intermediatetitle{Hymne}

\rubric{On chante l'hymne au Propre, au Commun, ou, pour les féries de l'année, au Psautier.}

\rubric{L'hymne est omis dans l'Office des Défunts, au Triduum Pascal, et à l'Épiphanie.}

\vfill

\rubric{Le dimanche et aux fêtes à trois nocturnes, on chante, à chacun des trois nocturnes, trois psaumes, précédés chacun d'une antienne que l'on répète à la fin du psaume après le Gloria Patri. Au temps pascal, les trois psaumes sont dits sous la même antienne, sans répétition de l'antienne, mais avec répétition du Gloria Patri. Aux féries, vigiles, et fêtes à un nocturne, on chante de la même manière neuf psaumes à l'unique nocturne. Après la psalmodie de chaque nocturne, on chante l'absolution, les bénédictions, les leçons et les répons, comme ci-après.}

\pagebreak

\intermediatetitle{Premier Nocturne}

\smalltitle{Absolution}
\gscore[n]{ORA}{T}{}{H@Absolutio}
\translation{Seigneur Jésus-Christ, exauce les prières de tes serviteurs, et aie pitié de nous,
toi qui vis et règnes avec le Père et le Saint-Esprit, dans les siècles des siècles. \rubric{\rrrub} Amen.}

\gscore[n]{ORLb}{T}{}{L@Benedictio}
\translation{Maître, veuillez bénir. \rubric{Bén.} Que le Père éternel nous bénisse d'une bénédiction perpétuelle.}
\gscore[n]{ORLc}{T}{}{}

\rubric{Les leçons sont au Propre ou, parfois, au Commun.}

\pagebreak

\rubric{À la fin des leçons:}

\gscore[n]{ORLd}{T}{}{T@In fine lectionum}
\translation{Et toi Seigneur, aie pitié de nous.}
\gscore[n]{ORLe}{T}{}{}

\rubric{Chaque leçon est suivie d'un répons, au Propre ou au Commun.}

\rubric{Bénédictions pour les leçons suivantes:}

\twocoltext{
	\rubric{\emph{Ben. 2.}} Unigénitus \textit{Dei} \textbf{Fí}lius~\GreSpecial{*}
	nos benedícere et adjuváre dignétur.
	\hspace{\specialcharhsep}\rr Amen.
}{
	\rubric{\emph{Bén. 2.}} Que le Fils unique de Dieu daigne nous bénir et nous secourir.
	\hspace{\specialcharhsep}\rr Amen.
}

\twocoltext{
	\rubric{\emph{Ben. 3.}} Spíritus \textit{Sancti} \textbf{grá}tia~\GreSpecial{*}
	illúminet sensus et corda nostra.
	\hspace{\specialcharhsep}\rr Amen.
}{
	\rubric{\emph{Bén. 3.}} Que la grâce du Saint-Esprit illumine nos esprits et nos cœurs.
	\hspace{\specialcharhsep}\rr Amen.
}

\intermediatetitle{Deuxième Nocturne}

\smalltitle{Absolution}

\twocoltext{
	\rubric{\emph{Absolutio 2.}}
	Ipsíus píetas et misericódi\textit{a nos} \textbf{ád}juvet,~\GreSpecial{*}
	qui cum Patre et Spíritu Sancto vivit et regnat in sǽcula sæculórum.
	\hspace{\specialcharhsep}\rr Amen.
}{
	\rubric{\emph{Absolution 2.}}
	Qu'il nous secoure par sa bonté et sa miséricorde, celui qui, 
	avec le Père et le Saint-Esprit, vit et règne dans les siècles des siècles.
	\hspace{\specialcharhsep}\rr Amen.
}

\smalltitle{Bénédictions}

\twocoltext{
	\rubric{\emph{Ben. 4.}} Deus Pa\textit{ter om}\textbf{ní}potens~\GreSpecial{*}
	sit nobis propítius et clemens.
	\hspace{\specialcharhsep}\rr Amen.
}{
	\rubric{\emph{Bén. 4.}}
	Que Dieu le Père tout-puissant soit pour nous propice et plein de clémence.
	\hspace{\specialcharhsep}\rr Amen.
}

\twocoltext{
	\rubric{\emph{Ben. 5.}} Chris\textit{tus per}\textbf{pé}tuæ~\GreSpecial{*}
	det nobis gaúdia vitæ.
	\hspace{\specialcharhsep}\rr Amen.
}{
	\rubric{\emph{Bén. 5.}}
	Que le Christ nous donne les joies de l'éternelle vie.
	\hspace{\specialcharhsep}\rr Amen.
}

\twocoltext{
	\rubric{\emph{Ben. 6.}} Ignem su\textit{i a}\textbf{mó}ris~\GreSpecial{*}
	accéndat Deus in córdibus nostris.
	\hspace{\specialcharhsep}\rr Amen.
}{
	\rubric{\emph{Bén. 6.}}
	Que Dieu daigne allumer dans nos cœurs le feu de son amour.
	\hspace{\specialcharhsep}\rr Amen.
}

\intermediatetitle{Troisième Nocturne}

\smalltitle{Absolution}

\twocoltext{
	\rubric{\emph{Absolutio 3.}}
	A vínculis peccató\textit{rum nos}\textbf{tró}rum~\GreSpecial{*}
	absólvat nos omnípotens et miséricors Dóminus.
	\hspace{\specialcharhsep}\rr Amen.
}{
	\rubric{\emph{Absolution 3.}}
	Que le Dieu tout-puissant et miséricordieux 
	daigne nous délivrer des liens de nos péchés.
	\hspace{\specialcharhsep}\rr Amen.
}

\smalltitle{Bénédictions}

\twocoltext{
	\rubric{\emph{Ben. 7.}}
	Evangé\textit{lica} \textbf{léc}tio~\GreSpecial{*}
	sit nobis salus et protéctio.
	\hspace{\specialcharhsep}\rr Amen.
}{
	\rubric{\emph{Bén. 7.}}
	Que la lecture du saint Évangile nous soit salut et protection.
	\hspace{\specialcharhsep}\rr Amen.
}

\rubric{Le dimanche et aux fêtes du Seigneur:}
\twocoltext{
	\rubric{\emph{Ben. 8.}}
	Diví\textit{num au}\textbf{xí}lium~\GreSpecial{*}
	máneat semper nobíscum.
	\hspace{\specialcharhsep}\rr Amen.
}{
	\rubric{\emph{Bén. 8.}}
	Que le secours divin demeure toujours avec nous.
	\hspace{\specialcharhsep}\rr Amen.
}

\rubric{Aux fêtes de la Sainte Vierge:}
\twocoltext{
	\rubric{\emph{Ben. 8.}}
	Cujus \textit{festum} \textbf{có}limus,~\GreSpecial{*}
	ipsa Virgo vírginum intercédat pro nobis ad Dóminum.
	\hspace{\specialcharhsep}\rr Amen.
}{
	\rubric{\emph{Bén. 8.}}
	Que celle dont nous célébrons la fête, la Vierge des vierges elle-même, 
	intercède pour nous auprès du Seigneur.
	\hspace{\specialcharhsep}\rr Amen.
}

\rubric{Aux fêtes des saints:}
\twocoltext{
	\rubric{\emph{Ben. 8.}}
	Cujus \rubric{(vel} Quarum\rubric{)} \textit{festum} \textbf{có}limus,~\GreSpecial{*}
	ipse \rubric{(vel} ipsa \rubric{aut} ipsæ\rubric{)}
	intercédat \rubric{(vel} intercédant\rubric{)} pro nobis ad Dóminum.
	\hspace{\specialcharhsep}\rr Amen.
}{
	\rubric{\emph{Bén. 8.}}
	Que celui \rubric{(ou \normaltext{celle}, \normaltext{ceux}, \normaltext{celles})} 
	dont nous célébrons la fête intercède\rubric{(}nt\rubric{)} pour nous 
	auprès du Seigneur.
	\hspace{\specialcharhsep}\rr Amen.
}

\vspace{\baselineskip}

\twocoltext{
	\rubric{\emph{Ben. 9.}}
	Ad societátem cívium \textit{super}\textbf{nó}rum~\GreSpecial{*}
	perdúcat nos Rex Angelórum.
	\hspace{\specialcharhsep}\rr Amen.
}{
	\rubric{\emph{Bén. 9.}}
	Que le Roi des Anges nous fasse parvenir à la société des citoyens célestes.
	\hspace{\specialcharhsep}\rr Amen.
}

\newpage

\rubric{Si la dernière leçon est une commémoraison de l'évangile du dimanche, de la férie ou de la vigile:}
\twocoltext{
	\rubric{\emph{Ben. 9.}}
	Per evangé\textit{lica} \textbf{dic}ta~\GreSpecial{*}
	deleántur nostra delícta.
	\hspace{\specialcharhsep}\rr Amen.
}{
	\rubric{\emph{Bén. 9.}}
	Par les paroles de l'Évangile, que nos péchés soient effacés.
	\hspace{\specialcharhsep}\rr Amen.
}

\intermediatetitle{À l'Office d'un seul Nocturne}

\smalltitle{Absolution}

\rubric{Lundi et jeudi comme au premier nocturne, mardi et vendredi comme au deuxième nocturne, mercredi et samedi comme au troisième nocturne.}

\smalltitle{Bénédictions}

\rubric{Aux féries où l'on lit une homélie, comme au troisième nocturne des dimanches.}

\rubric{Aux féries où l'on lit l'Écriture courante,
lundi et jeudi comme au premier nocturne, mardi et vendredi comme au deuxième nocturne, mercredi et samedi comme ci-dessous.}

\twocoltext{
	\rubric{\emph{Ben. 1.}}
	Ille nos \textit{bene}\textbf{dí}cat,~\GreSpecial{*}
	qui sine fine vivit et regnat.
	\hspace{\specialcharhsep}\rr Amen.
}{
	\rubric{\emph{Bén. 1.}}
	Qu'il nous bénisse, celui qui vit et règne sans fin.
	\hspace{\specialcharhsep}\rr Amen.
}

\twocoltext{
	\rubric{\emph{Ben. 2.}}
	Diví\textit{num au}\textbf{xí}lium~\GreSpecial{*}
	máneat semper nobíscum.
	\hspace{\specialcharhsep}\rr Amen.
}{
	\rubric{\emph{Bén. 2.}}
	Que le secours divin demeure toujours avec nous.
	\hspace{\specialcharhsep}\rr Amen.
}

\twocoltext{
	\rubric{\emph{Ben. 3.}}
	Ad societátem cívium \textit{super}\textbf{nó}rum~\GreSpecial{*}
	perdúcat nos Rex Angelórum.
	\hspace{\specialcharhsep}\rr Amen.
}{
	\rubric{\emph{Bén. 3.}}
	Que le Roi des Anges nous fasse parvenir à la société des citoyens célestes.
	\hspace{\specialcharhsep}\rr Amen.
}

\rubric{Aux fêtes des saints:}

\twocoltext{
	\rubric{\emph{Ben. 1.}}
	Ille nos \textit{bene}\textbf{dí}cat,~\GreSpecial{*}
	qui sine fine vivit et regnat.
	\hspace{\specialcharhsep}\rr Amen.
}{
	\rubric{\emph{Bén. 1.}}
	Qu'il nous bénisse, celui qui vit et règne sans fin.
	\hspace{\specialcharhsep}\rr Amen.
}

\twocoltext{
	\rubric{\emph{Ben. 2.}}
	Cujus \rubric{(vel} Quarum\rubric{)} \textit{festum} \textbf{có}limus,~\GreSpecial{*}
	ipse \rubric{(vel} ipsa \rubric{aut} ipsæ\rubric{)}
	intercédat \rubric{(vel} intercédant\rubric{)} pro nobis ad Dóminum.
	\hspace{\specialcharhsep}\rr Amen.
}{
	\rubric{\emph{Bén. 2.}}
	Que celui \rubric{(ou \normaltext{celle}, \normaltext{ceux}, \normaltext{celles})} 
	dont nous célébrons la fête intercède\rubric{(}nt\rubric{)} pour nous 
	auprès du Seigneur.
	\hspace{\specialcharhsep}\rr Amen.
}

\twocoltext{
	\rubric{\emph{Ben. 3.}}
	Ad societátem cívium \textit{super}\textbf{nó}rum~\GreSpecial{*}
	perdúcat nos Rex Angelórum.
	\hspace{\specialcharhsep}\rr Amen.
}{
	\rubric{\emph{Bén. 3.}}
	Que le Roi des Anges nous fasse parvenir à la société des citoyens célestes.
	\hspace{\specialcharhsep}\rr Amen.
}

\intermediatetitle{À l'Office de Sainte Marie le samedi}

\smalltitle{Absolution}

\twocoltext{
	\rubric{\emph{Absolutio.}}
	Précibus et méritis beátæ Maríæ semper Virginis
	et ómni\textit{um San}\textbf{ctó}rum,~\GreSpecial{*}
	perdúcat nos Dóminus ad regna cælórum.
	\hspace{\specialcharhsep}\rr Amen.
}{
	\rubric{\emph{Absolution.}}
	Que par les prières et les mérites de la bienheureuse Marie toujours Vierge 
	et de tous les saints, le Seigneur nous conduise au royaume des Cieux.
	\hspace{\specialcharhsep}\rr Amen.
}

\smalltitle{Bénédictions}

\twocoltext{
	\rubric{\emph{Ben. 1.}}
	Nos cum \textit{prole} \textbf{pi}a~\GreSpecial{*}
	benedícat Virgo María.
	\hspace{\specialcharhsep}\rr Amen.
}{
	\rubric{\emph{Bén. 1.}}
	Que la Vierge Marie nous obtienne la bénédiction de Son divin Fils.
	\hspace{\specialcharhsep}\rr Amen.
}

\twocoltext{
	\rubric{\emph{Ben. 2.}}
	Ipsa \textit{Virgo} \textbf{vír}ginum~\GreSpecial{*}
	intercédat pro nobis ad Dóminum.
	\hspace{\specialcharhsep}\rr Amen.
}{
	\rubric{\emph{Bén. 2.}}
	Que la Vierge des vierges elle-même, 
	intercède pour nous auprès du Seigneur.
	\hspace{\specialcharhsep}\rr Amen.
}

\twocoltext{
	\rubric{\emph{Ben. 3.}}
	Per Vír\textit{ginem} \textbf{ma}trem~\GreSpecial{*}
	concédat nobis Dóminus salútem et pacem.
	\hspace{\specialcharhsep}\rr Amen.
}{
	\rubric{\emph{Bén. 3.}}
	Par la Vierge Mère, donnez nous Seigneur, le salut et la paix.
	\hspace{\specialcharhsep}\rr Amen.
}

\newpage

\intermediatetitle{Te Deum}

\rubric{Le Te Deum remplace le dernier répons, le dimanche hors de l'Avent et du Carême, aux fêtes, et aux féries du Temps Pascal.}

\vspace{\baselineskip}

\rubric{On incline la tête aux mots \normaltext{Sanctus, Sanctus, Sanctus}, on s'incline profondément aux mots \normaltext{Tu ad liberándum}, on se met à genoux aux mots \normaltext{Te ergo quǽsumus}, et on incline la tête aux mots \normaltext{nomen tuum}.}

\rubric{Ton solennel:}
\gscore{ORTDa}{T}{}{W@Te Deum laudamus!Tonus solemnis}

\pagebreak

\rubric{Ton simple:}
\gscore{ORTDb}{T}{}{W@Te Deum laudamus!Tonus simplex}

\vspace{\baselineskip}

\translation{Nous te louons ô Dieu : nous te reconnaissons pour le Seigneur.\\
Ô Père éternel, toute la terre te révère.\\
Tous les Anges les Cieux, et toutes les Puissances,\\
Les Chérubins et les Séraphins te proclament sans cesse :\\
Saint, Saint, Saint le Seigneur, le Dieu des armées.\\
Les Cieux et la terre sont remplis de la majesté de ta gloire.\\
Le chœur glorieux des Apôtres,\\
La phalange vénérable des Prophètes,\\
l'armée des Martyrs éclatante de blancheur célèbre tes louanges;\\
La sainte Église confesse ton nom par toute la terre,\\
Ô Père d'infinie majesté!\\
Et elle vénère ton Fils véritable et unique,\\
Ainsi que le Saint-Esprit consolateur.\\
Tu es le Roi de gloire ô Christ!\\
Tu es du Père le Fils éternel.\\
Pour délivrer l'homme, tu n'as pas eu horreur du sein d'une Vierge.\\
Tu as brisé l'aiguillon de la mort\\
~\hfill et ouvert aux fidèles le royaume des cieux.\\
Tu es assis à la droite de Dieu dans la gloire du Père.\\
Nous croyons que tu es le juge qui doit venir.\\
Nous te supplions donc de secourir tes serviteurs\\
~\hfill que tu as rachetés par ton Sang précieux.\\
Fais qu'ils soient au nombre des saints, dans la gloire éternelle.\\
Sauve ton peuple, Seigneur  et bénis ton héritage.\\
Conduis tes serviteurs et élèves-les jusque dans l'éternité.\\
Chaque jour nous te bénissons.\\
Et nous louons ton nom dans les siècles; et dans les siècles des siècles.\\
Daigne Seigneur, en ce jour nous préserver de tout péché.\\
Aie pitié de nous Seigneur, aie pitié de nous.\\
Que ta miséricorde, Seigneur soit sur nous,\\
~\hfill comme notre espérance est en toi.\\
J'ai éspéré en toi Seigneur; que je ne sois pas confondu à jamais.}

\newpage

\intermediatetitle{Conclusion}

\rubric{La conclusion est omise si on chante les Laudes en suivant.}

{\grechangedim{spaceabovelines}{-2.5mm}{scalable}\grechangedim{spacebeneathtext}{2mm}{scalable}

\gscore[n]{ORDV}{T}{}{Dominus vobiscum}
\translation{Le Seigneur soit avec vous. \rubric{\rrrub} Et avec votre esprit.}

\vspace{0.5\baselineskip}
\rubric{Ton simple, aux féries et aux fêtes mineures:}
\gscore[n]{ORDVb}{T}{}{}


\vspace{\baselineskip}

\rubric{Si le célébrant n'est pas au moins diacre:}

\gscore[n]{ORDE}{T}{}{Domine exaudi}
\translation{Seigneur, entends ma prière. \rubric{\rrrub} Et que mon cri parvienne jusqu'à toi.}

\vspace{0.5\baselineskip}
\rubric{Ton simple, aux féries et aux fêtes mineures:}
\gscore[n]{ORDEb}{T}{}{}

}% end grechangedim spaceabovelines/spacebeneathtext

\vspace{\baselineskip}

\rubric{Oraison au Propre, à laquelle on répond \normaltext{Amen.}}

\pagebreak

\twocoltext{
	\versiculus{Dóminus vobíscum.}{Et cum spíritu tuo.}
	\rubric{vel \normaltext{Dómine, exáudi}, etc.}
}{
	\versiculus{Le Seigneur soit avec vous.}{Et avec votre esprit.}
	\rubric{ou \normaltext{Seigneur, entends}, etc.}
}

\rubric{\normaltext{Benedicámus Dómino} propre au jour, ci-après, puis:}

\twocoltext{
	\versiculus{Fidélium ánimæ per misericórdiam Dei requiéscant in pace.}{Amen.}
}{
	\versiculus{Que par la miséricorde de Dieu, les âmes des fidèles trépassés reposent en paix.}{Amen.}
}

\vspace{1\baselineskip}

\rubric{On reste en silence le temps d'un Notre Père.}

\vspace{2\baselineskip}

\sep

\vspace{2\baselineskip}

\intermediatetitle{Tons du Benedicamus}

\vspace{1\baselineskip}

\rubric{Aux fêtes solennelles}
\gscore{ORBDa}{T}{}{Z@Benedicamus Domino!In Festis Solemnibus}
\vfill
\rubric{Aux fêtes doubles}
\gscore{ORBDb}{T}{}{Z@Benedicamus Domino!In Festis Duplicibus}
\newpage
\rubric{Dans l'Octave de Pâques}
\gscore{ORBDj}{T}{}{Z@Benedicamus Domino!Per Octavam Paschae}
\vfill
\rubric{Au Temps Pascal}
\gscore{ORBDk}{T}{}{Z@Benedicamus Domino!Tempore Paschali}
\vfill
\rubric{Aux dimanches de l'Avent et du Carême}
\gscore{ORBDm}{T}{}{Z@Benedicamus Domino!In Dominicis Adventus et Quadragesimae}
\vfill
\rubric{Aux dimanches pendant l'année}
\gscore{ORBDe}{T}{}{Z@Benedicamus Domino!In Dominicis per Annum}
\pagebreak
\vfill
\rubric{Aux fêtes de la Sainte Vierge}
\gscore{ORBDd}{T}{}{Z@Benedicamus Domino!In Festis B.M.V.}
\vfill
\rubric{Aux fêtes semi-doubles}
\gscore{ORBDc}{T}{}{Z@Benedicamus Domino!In Festis Semiduplicibus}
\vfill
\rubric{Aux fêtes simples}
\gscore{ORBDf}{T}{}{Z@Benedicamus Domino!In Festis Simplicibus}
\vfill
\rubric{À l'Office de Sainte Marie le samedi}
\gscore{ORBDg}{T}{}{Z@Benedicamus Domino!In Officio B.M.V. in Sabbato\linebreak\null}
\vfill
\rubric{Aux féries, hors du Temps Pascal}
\gscore{ORBDi}{T}{}{Z@Benedicamus Domino!In Feriis}
\pagebreak

~\\

\thispagestyle{empty}

\vspace{3cm}

\intermediatetitle{Prière après l'Office}

\vspace{\baselineskip}

\twocoltext{
\vspace{\baselineskip}

\lettrine{S}{acrosánctæ} et indivíduæ Trinitáti, crucifíxi Dómini nostri Jesu Christi humanitáti, beatíssimæ et gloriosíssimæ sempérque Vírginis Maríæ fœcúndæ integritáti, et ómnium Sanctórum universitáti sit sempitérna laus, honor, virtus et glória ab omni creatúra, nobísque remíssio ómnium peccatórum, per infiníta sǽcula sæculórum. Amen.
}{
\lettrine{À}{ la} Très Sainte et indivisible Trinité, à l'humanité de Notre Seigneur Jésus-Christ crucifié et à la féconde intégrité de la Bienheureuse et très glorieuse Marie toujours Vierge, ainsi qu'à toute l'assemblée des Saints, soient éternelle louange, honneur, puissance et gloire de la part de toute créature, et à nous rémission de tous nos péchés, pour l'infinie durée des siècles et des siècles. Amen.
}

\end{document}

 