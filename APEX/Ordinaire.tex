% !TEX TS-program = lualatex
% !TEX encoding = UTF-8

\documentclass[11pt, twoside, french, openany]{book}

%%%%%%%%%%%%%%% INDICES %%%%%%%%%%%%%%%

\usepackage{imakeidx}

%% This is deep magic required to make dotted_mode.ist work.
%% Found in https://tex.stackexchange.com/questions/594568/how-to-integrate-a-tabular-environment-in-the-index
\setlength{\columnseprule}{0.4pt}
\newsavebox\ltmcbox
\newlength\mysavecolroom
%% This is added before automatically-inserted instances of \\ in longtable-based indices in order to ignore those
\newcommand{\ignoreNL}[1]{} 
%% This is a special macro to be used for explicit newlines
%% inside indices where a tabular is used,
%% which are indices where the mode is printed,
%% to be used in the indexed value (i.e. the piece's title)
\newcommand{\idxnewline}{\newline\null\hspace{4mm}}
%% a simple newline\null can be added if it is at the end of the title.

\indexsetup{level=\section*,toclevel=section,noclearpage,othercode=\footnotesize\thispagestyle{empty}}
\makeindex[name=I,title=Index Invitatoriorum, columns=2,columnseprule, options=-s dotted_mode.ist]
\makeindex[name=H,title=Index Hymnorum, columns=2,columnseprule, options=-s dotted_thinmode.ist]
\makeindex[name=A,title=Index Antiphonarum, columns=2,columnseprule, options=-s dotted_mode.ist]
\makeindex[name=R,title=Index Responsoriorum, columns=2,columnseprule, options=-s dotted_thinmode.ist]
\makeindex[name=P,title=Index Psalmorum, columns=2,columnseprule, options=-s dotted.ist]
\makeindex[name=T,title=Toni Communes, columns=2,columnseprule, options=-s dotted.ist]
\makeindex[name=F,title=Index Festorum, columns=2,columnseprule, options=-s dotted.ist]

%%%%%%%%%%%%%%% STANDARD PACKAGES %%%%%%%%%%%%%%%

%% This is the format of the recent Solesmes books.
%\usepackage[paperwidth=135mm, paperheight=205mm]{geometry}

%% This is made to be printed on KDP 6inx9in paperback without changing the reference column width
\usepackage[paperwidth=145mm, paperheight=217.5mm]{geometry}

%% This is the format of the 1912 Antiphonale Romanum
%\usepackage[paperwidth=160mm, paperheight=240mm]{geometry}

\usepackage{fontspec}
\usepackage[nolocalmarks]{polyglossia}
\usepackage[table]{xcolor}
\usepackage{fancyhdr}
\usepackage{titlesec}
\usepackage{setspace}
\usepackage{expl3}
\usepackage{hyperref}
\usepackage{refcount}
\usepackage{needspace}
\usepackage{etoolbox}
\usepackage{enumitem}
\usepackage{lettrine}
\usepackage{longtable}
\usepackage{luacode}
\usepackage{paracol}


%%%%%%%%%%%%%%% HYPHENATION AND TYPOGRAPHICAL CONVENTIONS %%%%%%%%%%%%%%

%\setdefaultlanguage[variant=ecclesiastic, hyphenation=liturgical, usej=true, babelshorthands=false]{latin}
\setdefaultlanguage[variant=french, frenchitemlabels=false]{french}
\setotherlanguage{english}
%% this latin option basically boils down to French with slightly thinner pre-punctuation spaces 
%% and slightly different hyphenation (but how different?)
%% for some reason, it produces more underfull hboxes in two-col psalms, than French; and about 0.5% more pages.
%% French is therefore to be kept until further notice.
%\setotherlanguage[variant=ecclesiastic, hyphenation=liturgical, usej=true]{latin}

%%%%%%%%%%%%%%% GEOMETRY %%%%%%%%%%%%%%%

%% This should mimic the layout of the recent Solesmes books.
%\geometry{
%inner=15mm,
%outer=10mm,
%top=12mm,
%bottom=15mm,
%headsep=3mm,
%}

%% This should ensures the same column width (for line breaks) as the Solesmes reference layout, but to print on 6x9in KDP paperback
\geometry{
inner=25mm,
outer=10mm,
top=12mm,
bottom=15mm,
headsep=3mm,
}

%% General scale of all graphical elements.
%% Values different from 1 are largely untested.
%% Used in those commands (e.g. everything FontSpec) that use a scale parameter.
\newcommand{\customscale}{1}

%% Provide the command \fpevalc as a copy of the code-level \fp_eval:n.
%% \fpevalc allows to evaluate floating point calculation for scaled parameters, e.g. \setSomeStretchFactor{\fpevalc{0,6 * \customscale}}
\ExplSyntaxOn
\cs_new_eq:NN \fpevalc \fp_eval:n
\ExplSyntaxOff

%% No indentation of paragraphs
\setlength{\parindent}{0mm}

%% We want to allow large inter-words space 
%% to avoid overfull boxes in two-columns rubrics.
\sloppy

%%%%%%%%%%%%%%% GREGORIO CONFIG %%%%%%%%%%%%%%%

\usepackage[forcecompile]{gregoriotex}

%% this limits how much scores can stretch vertically
%% when they are forced to adhere to the bottom of a page (i.e followed by \pagebreak)
%\grechangedim{baselineskip}{55pt plus 1pt minus 5pt}{scalable}
%% this however has been found to break the page layout of the psalterium festivum, somehow

%% text above lines shall be of color gregoriocolor
\grechangestyle{abovelinestext}{\color{gregoriocolor}\footnotesize\itshape}
%% fine-tuning of space beween the staff and the text above lines
\newcommand{\altraise}{-0.4mm} %% default is -0.1cm
\grechangedim{abovelinestextraise}{\altraise}{scalable}
\grechangedim{abovelinestextheight}{10mm}{scalable}

%% fine-tuning of space between the staff and the lyrics
\newcommand{\textraise}{2.8ex} %% default is 3.48471 ex
\grechangedim{spacelinestext}{\textraise}{scalable}

%% fine-tuning of space between the initial and the annotations
\newcommand{\annraise}{0mm} %% default is -0.2mm
\grechangedim{annotationraise}{\annraise}{scalable}

%% fine-tuning the behavior of text placed under bars. We use the so-called "new algorithm" which
%% places the bar in the middle of surrounding notes, and the text in the middle of surrounding text.
%% however, we restrict drastically the deviation of the text from the position of the bar.
\grechangedim{maxbaroffsettextleft}{0.5mm}{scalable}
\grechangedim{maxbaroffsettextright}{0.5mm}{scalable}

%% in case we show NABC, font selection
\gresetnabcfont{gregall}{12} 

%% \officepartannotation converts a letter (IHARPT) into the office part to be printed as annotation,
%% storing the result into \result.

\newcommand{\result}{}
\newcommand{\lookup}[3]{%
  \IfSubStr{#2}{#1}{ \renewcommand{\result}{#3} }{}%
}%
\newcommand{\officepartannotation}[1]{%
  \renewcommand{\result}{#1}%
  \lookup{#1}{T}{}%
  \lookup{#1}{H}{Hymn.}%
  \lookup{#1}{A}{Ant.}%
  \lookup{#1}{P}{}%
  \lookup{#1}{R}{Resp.}%
  \lookup{#1}{I}{Invit.}%
  \result%
}%

%% header capture setup for the mode, with indexation of the score
%% IMPORTANT NOTE: giving mode info is necessary for a score to be indexed.
%% in this context, the mode may not be left blank in the gabc, as in: "mode:;",
%% but must have some sort of text going on, as in: "mode:~;" which will produce the expected result
\newcommand{\defaultannotationshift}{-2mm}
\newcommand{\indexnamebuffer}{}
\newcommand{\piecenamebuffer}{}
\newcommand{\modeannotation}[1]{
  \greannotation{\hspace{\defaultannotationshift}#1}
  \ifblank{\piecenamebuffer}{}{% if the indexed name of the piece is not given, no indexation effort is made
    \ifthenelse{\equal{I}{\indexnamebuffer}}{
      \index[\indexnamebuffer]{{\piecenamebuffer}@#1 & \piecenamebuffer}
    }{
	  \ifthenelse{\equal{H}{\indexnamebuffer}}{
        \index[\indexnamebuffer]{{\piecenamebuffer}@#1 & \piecenamebuffer}
      }{
	    \ifthenelse{\equal{A}{\indexnamebuffer}}{
          \index[\indexnamebuffer]{{\piecenamebuffer}@#1 & \piecenamebuffer}
        }{
	      % if the piece is not among the given office_parts, no indexation effort is made
        }
      }
    }
  }
}
\gresetheadercapture{mode}{modeannotation}{string}

%% outputs a score with no label, indexing, initials or annotations
%% for 
\newcommand{\unindexedscore}[1]{
  \gresetinitiallines{0}
  %% the use of a directory called "gabc" is linked
  %% to the management of gabc files by the website: do not change 
  %% without also changing the website static files structure
  \gregorioscore{\subfix{nocturnale-romanum/gabc/#1}}
  \gresetinitiallines{1}
}

%% this removes instances of \newline from a string.
%% used in order to allow new lines in the indexed name of a piece, 
%% but not have those new lines in the \@currentlabelname for reference purposes
\begin{luacode*}
function delete_newline ( s )
   s = string.gsub ( s, 'newline', '')
   s = string.gsub ( s, 'hspace', '')
   s = string.gsub ( s, '4mm', '')
   s = string.gsub ( s, 'protect', '')
   s = string.gsub ( s, 'hbox', '')
   s = string.gsub ( s, [[\]], '')
   s = string.gsub ( s, "{}", '')
   tex.sprint ( s )
end
\end{luacode*}

%% outputs a score with label, indexing, and annotations. no initials if [n] is passed
\makeatletter
\newcommand{\gscore}[5][y]{
  %% #1 (passed as option) : y = initial, n = no initial
  %% #2 : name of the score file, should be a code, e.g. Q4F4A3 or 1225N1R1
  %% #3 : office-part among the values: T, H, A, P, R, I (toni communes, hy., ant., psalmus, resp., invit.)
  %% #4 : if applicable, a number between 1 and 9 (rank of the ant./resp.) - else: empty
  %% #5 : the indexed name of the piece
  
  %% this prevents page breaks between the phantom section and its label, and the actual score.
  \needspace{4\baselineskip} 
  \protected@edef\@currentlabelname{\directlua{delete_newline(\luastring{#5})}}
  \phantomsection
  \label{#2}
  %% we add the office part, and number of that ant. or resp. in the current office, if applicable
  %% todo : the negative hspace is here because somehow the initial and annotation (first line only) are misaligned by 1mm _with this initial font size_.
  %% this should probably be fixed in a more elegant way.
  \greannotation[c]{\hspace{-1.4mm}\hspace{\defaultannotationshift}\officepartannotation{#3}#4}
  %% if #5 (indexed name) is blank, nothing is indexed.
  %% this is for pieces that are repetitions of another piece (antiphons after psalms)
  \ifblank{#5}{
    \renewcommand{\indexnamebuffer}{}
    \renewcommand{\piecenamebuffer}{}
  }{
    \renewcommand{\indexnamebuffer}{#3}
    \renewcommand{\piecenamebuffer}{#5}
    \ifthenelse{\equal{I}{\indexnamebuffer}}{
    % in this case, the mode is added to the index entry, and therefore indexation will be done by the header capture
    }{
      \ifthenelse{\equal{A}{\indexnamebuffer}}{
      % in this case, the mode is added to the index entry, and therefore indexation will be done by the header capture
      }{
	    \ifthenelse{\equal{H}{\indexnamebuffer}}{
        % in this case, the mode is added to the index entry, and therefore indexation will be done by the header capture
        }{
	      \index[#3]{#5}
	    }
	  }
	}
  }
  %% if optional arg #1 has been passed as 'n', set no initial
  \ifx n#1\gresetinitiallines{0}\fi
  %% the use of a directory called "gabc" is linked
  %% to the management of gabc files by the website: do not change 
  %% without also changing the website static files structure
  \gregorioscore{\subfix{nocturnale-romanum/gabc/#2}}
  %% if optional arg #1 has been passed as 'n', unset no initial
  \ifx n#1\gresetinitiallines{1}\fi
  \vspace{1mm}
}
\makeatother

%%% We select if we want to show NABC. Here, we do.

\gresetnabc{1}{visible}
\gresetnabc{2}{invisible}

%%%%%%%%%%%%%%% FONTS %%%%%%%%%%%%%%%

%%%%%%%%%%%%%%% Main font
\setmainfont[Ligatures=TeX, Scale=\customscale]{Charis SIL}
%\setmainfont[Ligatures=TeX, Scale=\customscale]{TeXGyreBonum-Regular}
\setstretch{\fpevalc{1.05 * \customscale}}

%%%%%%%%%%%%%%% Score initials
%% \initialsize resizes the initials, with one argument (size in points)
\newcommand{\initialsize}[1]{
    \grechangestyle{initial}{\fontspec{Zallman Caps}\fontsize{#1}{#1}\selectfont}
}
%% default initial size is 32 points
\newcommand{\defaultinitialsize}{28}
\initialsize{\defaultinitialsize}

%% spacing before and after initials to kern the Zallman Caps.
%% this should be changed if we move away from Zallman Caps.
\newcommand{\initialspace}[2]{
  \grechangedim{afterinitialshift}{#2}{scalable}
  \grechangedim{beforeinitialshift}{#1}{scalable}
}
%% default space before and after initials is 0mm to the left and 2mm to the right.
\newcommand{\defaultinitialspace}{\initialspace{0mm}{-\defaultannotationshift}}
\defaultinitialspace{}

%%%%%%%%%%%%%%% Score annotations
\grechangestyle{annotation}{\small}

%%%%%%%%%%%%%%% GRAPHICAL ELEMENTS %%%%%%%%%%%%%%%

%% V/, R/, A/ and + signs for in-line use (\vv \rr \aa \cc)
\newcommand{\specialcharhsep}{3mm} % space after invoking R/ or V/ or A/ outside rubrics
\newcommand{\vv}{\textcolor{gregoriocolor}{\fontspec[Scale=\customscale]{Charis SIL}℣.\nolinebreak[4]\hspace{\specialcharhsep}\nolinebreak[4]}}
\newcommand{\rr}{\textcolor{gregoriocolor}{\fontspec[Scale=\customscale]{Charis SIL}℟.\nolinebreak[4]\hspace{\specialcharhsep}\nolinebreak[4]}}
\renewcommand{\aa}{\textcolor{gregoriocolor}{\fontspec[Scale=\customscale]{Charis SIL}\Abar.\nolinebreak[4]\hspace{\specialcharhsep}\nolinebreak[4]}}
\newcommand{\cc}{\textcolor{gregoriocolor}{\fontspec[Scale=\customscale]{FreeSerif}\symbol{"2720}~}}
%% Same special characters, for in-score use (<sp>V/ R/ A/ +</sp>)
\gresetspecial{V/}{\textcolor{gregoriocolor}{\fontspec[Scale=\customscale]{Charis SIL}℣.~}}
\gresetspecial{R/}{\textcolor{gregoriocolor}{\fontspec[Scale=\customscale]{Charis SIL}℟.~}}
\gresetspecial{A/}{\textcolor{gregoriocolor}{\fontspec[Scale=\customscale]{Charis SIL}\Abar.~}}
\gresetspecial{+}{{\fontspec[Scale=\customscale]{FreeSerif}†~}}
\gresetspecial{*}{\gresixstar}
\gresetspecial{cross}{\textcolor{gregoriocolor}{\fontspec[Scale=\customscale]{FreeSerif}\symbol{"2720}}}
\gresetspecial{labiacross}{\textcolor{gregoriocolor}{+}}
%% Same special characters, for use in rubrics (no space, and no red command since it will be reddified with the rest)
\newcommand{\vvrub}{{\fontspec[Scale=\customscale]{Charis SIL}℣.~}}
\newcommand{\rrrub}{{\fontspec[Scale=\customscale]{Charis SIL}℟.~}}
\newcommand{\aarub}{{\fontspec[Scale=\customscale]{Charis SIL}\Abar.~}}

%% the asterisk as found in the mediants of text-only psalms
\newcommand{\psstar}{\GreSpecial{*}}
\newcommand{\pscross}{\GreSpecial{+}}
%% also, most psalms do not call those but use † and * - todo

%% Roman Numerals
\usepackage{modroman}
\newcommand{\Rnum}[1]{\nbRoman{#1}}
\newcommand{\rnum}[1]{\nbshortroman{#1}}

%% Macro to print versicles
\newcommand{\versiculus}[2]{\par\vv #1 \\ \rr #2\par}

\newcommand{\versiculustpall}[2]
	{\versiculus{#1 \rubric{(T.P.} Allelúia.\rubric{)}}{#2 \rubric{(T.P.} Allelúia.\rubric{)}}}

%% Macro to print rubrics
\newcommand{\rubric}[1]{\textcolor{gregoriocolor}{\emph{#1}}}

%% Macro to print the name of a score in normal characters inside a \rubric
\newcommand{\normaltext}[1]{{\normalfont\normalcolor #1}}
\newcommand{\scorename}[1]{\normaltext{\nameref{M-#1}}}

%% Macro to print a full reference to a responsory
%% #1 is the R/ number in the feast, #2 is the R/ code, #3 is an optional additional text, like "sine Gloria Patri".
\newcommand{\respref}[3]{\rubric{%
\rrrub #1 \scorename{#2}, pag.\ \pageref{M-#2}%
\if\relax\detokenize{#3}\relax%
.%
\else%
, #3.%
\fi%
}}

\newcommand{\resprefcumgp}[2]
	{\respref{#1}{#2}{sed cum \normaltext{Glória Patri} in fine}}
	
\newcommand{\resprefsinegp}[2]
	{\respref{#1}{#2}{sine \normaltext{Glória Patri}}}

%% Macro to print the common rubric that signals the Te Deum
\newcommand{\tedeumrubric}{\rubric{Lectione ultima peracta Hymnus \normaltext{Te Deum} cantatur.}}

%% Macro to print the common rubric that signals the Paschaltide three-psalm one-antiphon rule
\newcommand{\tptresrubric}{\rubric{Tempore Paschali tres Psalmi sub hac Antiphona dicuntur.}}

%% Macro to print a reference to a single Invitatory
\newcommand{\invitref}[1]{\rubric{Invitatorium \scorename{#1}, pag.\ \pageref{M-#1}.}}

%% Macro to print the common rubric "invitatorium de feria"
\newcommand{\invitferia}{\rubric{Invitatorium de Feria.}}

%% Macros to print the common rubrics regarding feast solemnity
\newcommand{\rubricsimplefeasts}{\rubric{Tonus simplex:}}
\newcommand{\rubricdoublefeasts}{\rubric{Tonus festivus:}}
\newcommand{\rubricsolemnfeasts}{\rubric{Tonus solemnis:}}
\newcommand{\rubrictp}{\rubric{Tempore Paschali:}}

%% Macro to print a separator

\newcommand{\sep}{{\centering\greseparator{3}{20}\par}}

%% Macro to print alternative text (feminine, plural...) between red parentheses
\newcommand{\bracketed}[1]{{\textcolor{gregoriocolor}(}#1{\textcolor{gregoriocolor})}}

%% Macro to print the translation of a full-width score below it, except hymns
\newcommand{\translation}[1]{
	\emph{#1}
}

%%%%%%%%%%%%%%% COLUMN MANAGEMENT %%%%%%%%%%%%%%%

\usepackage{multicol}
\usepackage{parcolumns}
\setlength\columnseprule{0.4pt}
\setlength{\multicolsep}{6pt plus 2pt minus 1.5pt}

%% Macro to print an itemized list on two columns (psalm, canticle, etc) from a tex file
\newcommand{\twocolitemized}[1]{
	\begin{multicols}{2}%
	\begin{itemize}[
		label=\null, 
		leftmargin=0pt, 
		itemindent=5pt, 
		labelsep=0pt, 
		labelwidth=0pt, 
		rightmargin=0pt, 
		parsep=0pt, 
		itemsep=0pt plus 1pt,
		topsep=-2mm]
	\input{#1}%
	\end{itemize}%
	\end{multicols}%
}

\newcommand{\itemized}[1]{
	\begin{itemize}[
		label=\null, 
		leftmargin=0pt, 
		itemindent=5pt, 
		labelsep=0pt, 
		labelwidth=0pt, 
		rightmargin=0pt, 
		parsep=0pt, 
		itemsep=0pt plus 1pt,
		topsep=-2mm]
	\input{#1}%
	\end{itemize}%
}

%% Macro to print a psalm on two columns with the title "psalmus N".
\newcommand{\psalmus}[3][]{
	%% #1, passed as option, label or blank.
	%% Every psalm_tone combination should be printed once with label, and the other times without label.
	%% #2, psalm number
	%% #3, tone
	\vspace{0.5\baselineskip}
	\needspace{5\baselineskip}
	\phantomsection
	\ifblank{#1}{}{\label{Psalm#2_#3}}
	\index[P]{#2 (mode #3)}
	\smalltitle{Psaume #2}
	\vspace{2mm}
	\begin{paracol}{2}
		\itemized{nocturnale-romanum/psalmi/#2_#3.tex}%
	\switchcolumn
		\itemized{psalmi_fr/#2.tex}
	\end{paracol}
}

%% Macro to print a psalm on two columns with the title "in tono N"
\newcommand{\psalmustonus}[3][]{
	%% #1, passed as option, label or blank.
	%% Every psalm_tone combination should be printed once with label, and the other times without label.
	%% #2, psalm number
	%% #3, tone
	\vspace{0.5\baselineskip}
	\needspace{3\baselineskip}
	\phantomsection
	\ifblank{#1}{}{\label{Psalm#2_#3}}
	\index[P]{#2 (mode #3)}
	\smalltitle{in tono #3}
	\twocolitemized{nocturnale-romanum/psalmi/#2_#3.tex}%
}

%% Macro to reference a psalm printed elsewhere
\newcommand{\psalmusref}[2]{
	\rubric{Psalmus \normaltext{#1} in tono \normaltext{#2}, pag.\ \pageref{M-Psalm#1_#2}.}\par
}

\newcommand{\twocolrubric}[2]{
	\begin{parcolumns}[rulebetween]{2}
	\colchunk{%
      \rubric{#1}
	}
	\colchunk{%
	  \rubric{#2} 
    }
	\end{parcolumns}
}

\newcommand{\twocoltext}[2]{
	\begin{paracol}{2}
#1
	\switchcolumn
#2
	\end{paracol}
}

%%%%%%%%%%%%%%% HEADER STYLES %%%%%%%%%%%%%%%

\pagestyle{fancy}
\fancyhead{}
\fancyfoot{}
\renewcommand{\headrulewidth}{0pt}
\setlength{\headheight}{20pt}
\fancyhead[RO]{\small\rightmark\hspace{1cm}\thepage}
\fancyhead[LE]{\small\thepage\hspace{1cm}\leftmark}

% this command is called every time the left and right header texts are set (e.g. by calling \feast)
% the hyphenpenalty override is neede only on older versions of gregorio which do not reset it correctly after typesetting the score.
% see https://tex.stackexchange.com/questions/581013/lualatex-hyphenation-issue-in-fancyhdr-with-gregoriotex-and-multicols-latin-te
\newcommand{\setheaders}[2]{
	\renewcommand{\rightmark}{\hyphenpenalty=50{\sc#2}}
	\renewcommand{\leftmark}{\hyphenpenalty=50{\sc#1}}
}
\setheaders{}{}

%%%%%%%%%%%%%%% TITLE STYLES %%%%%%%%%%%%%%%

%% Titles are centered and small-caps
\titleformat{\chapter}[block]{\Large\filcenter\sc}{}{}{}
\titleformat{\section}[block]{\large\filcenter\sc}{}{}{}
\titleformat{\subsection}[block]{\filcenter\sc}{}{}{}
\setcounter{secnumdepth}{0}
%% Fine-tuning of space around titles
\titlespacing*{\paragraph}{0pt}{1ex}{.6ex}

%% Typesets all titles throughout the NR except Nocturn titles and a few special titles.
%% Using a continuation is necessary because there are 11 arguments.
%% Only \feast, \nocturn, \intermediatetitle and \smalltitle should ever be used.
\newcommand{\feast}[6]{
  %% #1: feast code, e.g. 1225 or A1F1
  %% #2: feast title
  %% #3: left header title
  %% #4: right header title
  %% #5: title level
    %% title level 1 : full page width, a few major feasts + titles of temporale, sanctorale, etc.
	%% title level 2 : all feasts, sundays and major ferias
	%% title level 3 : ferias
  %% #6: incipit date (goes above feast title)
  %% cont'd #1: 1954 rank
  %% cont'd #2: 1960 rank
  %% cont'd #3: name of the feast as it shows up in the index
  %% cont'd #4: 1945 feast-wide rubrics
  %% cont'd #5: 1960 feast-wide rubrics

  %% needspace: should be barely more than the vertical space for the titles, rubrics excluded.
  %% this is to ensure that the page does not get cut after the title or the phantomsection.
  \needspace{8\baselineskip}
  %% phantomsection is to allow the label to attach to the title and not the previous counter object.
  \phantomsection
  \label{#1}
  \begin{center}
  %% we typeset a line for the date if the date is not blank
  \ifblank{#6}{}{
    {\large #6}\\%
  }%
  %% the actual title
  \ifx 1#5{\setstretch{1.2}\sc\huge #2\par}\fi
  \ifx 2#5{\Large #2\par}\fi
  \ifx 3#5{\large #2\par}\fi
  \end{center}
  %% If this is a level 1 title, empty pagestyle
  \ifx 1#5\thispagestyle{empty}\fi
  
  %% we define the header titles manually
  \setheaders{#3}{#4}
  
  %% moving on to a continuation macro to unpack the last 5 arguments
  \feastcontinued
}
\newcommand{\feastcontinued}[5]
{
  %% we name the last 5 arguments
  \def\oldrank{#1}%
  \def\newrank{#2}%
  \def\indexfeastname{#3}%
  \def\oldrubric{#4}%
  \def\newrubric{#5}%
  %% we index the feast if the indexing name is given
  \ifblank{#3}{}{
	\index[F]{\indexfeastname}
  }
  %% we typeset a two-column rank & rubrics block if one rank is filled in
  %% if the left rubric is filled in, we add a rule between the columns
  \ifblank{#5}{\def\ourrule{}}{\def\ourrule{rulebetween}}
  \ifblank{#1}{}{\vspace{-2mm}%
    \begin{parcolumns}[\ourrule]{2}
	\colchunk{%
      {\centering\oldrank\par}#4}%
	\colchunk{%
	  {\centering\newrank\par}#5}%
	\end{parcolumns}
	\vspace{2mm}
  }
}

\newcommand{\smalltitle}[1]{
  \needspace{7\baselineskip}
  \par{\centering\textbf{#1}\par}
}

\newcommand{\intermediatetitle}[1]{
  \needspace{10\baselineskip}
  \begin{center}
  {\large #1}
  \end{center}
 }

\newcommand{\nocturn}[1]{
  \intermediatetitle{In \Rnum{#1} Nocturno}
}

%% command to wrap printindex and set the headers for indices
\newcommand{\cprintindex}[2]{
	\setheaders{Indices}{#2}
	\pagestyle{fancy}
	\thispagestyle{empty}
	\printindex[#1]
}

%%%%%%%%%%%%%%% SUBFILES %%%%%%%%%%%%%%%

\usepackage{xr}
\usepackage{subfiles}

%% When we start a new subfile (new chapter), 
%% we start on a new page (with blank filling on the previous page) and create a corresponding label.
\newcommand{\customsubfile}[1]{\newpage\label{#1}\thispagestyle{empty}\subfile{#1}}


\begin{document}

\gresetnabc{1}{invisible} % none of the chants of the Ordinary really needs neumes

\null\vfill
\feast{OR}{Ordinaire de l'Office Divin\\à Matines}
	{}{}{1}{}{}{}{}{}{}
\vfill
\pagebreak

\feast{OR}{Ordinaire de l'Office Divin\\à Matines}
	{}{}{1}{}{}{}{}{}{}

\intermediatetitle{Prières avant l'Office}

\twocoltext{
\lettrine{A}{peri}, Dómine, os meum ad benedicéndum nomen sanctum tuum:
munda quoque cor meum ab ómnibus vanis, pervérsis et aliénis cogitatiónibus; 
intelléctum illúmina, afféctum inflámma, ut digne, atténte ac devóte
hoc officium recitáre váleam, et exaudíri mérear
ante conspéctum divinæ Majestátis tuæ. Per Christum Dóminum nostrum. Amen.
}{
\lettrine{O}{uvre} ma bouche, Seigneur, pour bénir ton saint nom, purifie mon cœur de toute pensée vaine, perverse et profane. Illumine mon intelligence, enflamme mon cœur, pour que je puisse réciter cet office dignement, attentivement et pieusement, et mériter d’être exaucé en présence de ta majesté divine. Par le Christ notre Seigneur. Amen.
}

\vspace{\baselineskip}

\twocoltext{
\lettrine{D}{ómine}, in unióne illíus divínæ intentiónis,
qua ipse in terris laudes Deo persolvísti,
has tibi horas \rubric{(vel} hanc tibi horam\rubric{)} persólvo.
}{
\lettrine{S}{eigneur}, en m'unissant à la divine intention avec laquelle, sur terre, tu offrais à Dieu le service de louange, je t'offre cette (ces) heure(s).
}

\vspace{2\baselineskip}

\sep

\vspace{2\baselineskip}

\twocoltext{
\lettrine{P}{ater noster}, qui es in cælis, sanctificétur nomen tuum.
Advéniat regnum tuum. Fiat volúntas tua, sicut in cælo et in terra.
Panem nostrum quotidiánum da nobis hódie.
Et dimítte nobis débita nostra, sicut et nos dimíttimus debitóribus nostris.
Et ne nos indúcas in tentatiónem: sed líbera nos a malo. Amen.
}{
\lettrine{N}{otre Père}, qui es aux cieux, que ton nom soit sanctifié, que ton règne vienne,
que ta volonté soit faite sur la terre comme au ciel.
Donne-nous aujourd’hui notre pain de ce jour. Pardonne-nous nos offenses, comme nous pardonnons aussi à ceux qui nous ont offensés.
Et ne nous laisse pas entrer en tentation mais délivre-nous du Mal. Amen.
}

\vspace{\baselineskip}

\twocoltext{
\lettrine{A}{ve María}, grátia plena, Dóminus tecum:
benedícta tu in muliéribus, et benedíctus fructus ventris tui Jesus.
Sancta María, Mater Dei, ora pro nobis peccatóribus,
nunc et in hora mortis nostræ. Amen.
}{
\lettrine{J}{e vous salue Marie}, pleine de grâce, le Seigneur est avec vous.
Vous êtes bénie entre toutes les femmes et Jésus, le fruit de vos entrailles, est béni.
Sainte Marie, Mère de Dieu, priez pour nous pauvres pécheurs, maintenant et à l’heure de notre mort.
Amen.
}

\vspace{\baselineskip}

\twocoltext{
\lettrine{C}{redo in Deum}, Patrem omnipoténtem, Creatórem cæli et terræ.
Et in Jesum Christum, Fílium ejus únicum, Dóminum nostrum,
qui concéptus est de spíritu Sancto, natus ex María Virgine,
passus sub Póntio Piláto, crucifixus, mórtuus et sepúltus:
descéndit ad ínferos: tértia die resurréxit a mórtuis;
ascéndit ad cælos, sedet ad déxteram Patris omnipoténtis:
inde ventúrus est judicáre vivos et mórtuos.
Credo in Spíritum sanctum, sanctam Ecclésiam cathólicam,
Sanctórum communiónem, remissiónem peccatórum,
carnis resurrectiónem, vitam ætérnam. Amen.
}{
\lettrine{J}{e crois en Dieu}, le Père tout-puissant, créateur du ciel et de la terre.
Et en Jésus Christ, son Fils unique, notre Seigneur;
qui a été conçu du Saint Esprit, est né de la Vierge Marie,
a souffert sous Ponce Pilate, a été crucifié,
est mort et a été enseveli, est descendu aux enfers;
le troisième jour est ressuscité des morts,
est monté aux cieux, est assis à la droite de Dieu le Père tout-puissant,
d’où il viendra juger les vivants et les morts.
Je crois en l’Esprit Saint, à la sainte Église catholique, à la communion des saints,
à la rémission des péchés, à la résurrection de la chair, à la vie éternelle. Amen.
}

\pagebreak

\intermediatetitle{Ouverture}

\rubric{Le dimanche et aux fêtes à trois nocturnes:}
\gscore{ORIa}{T}{}{A@Domine labia mea!Tonus festivus}
\translation{Seigneur, ouvre mes lèvres, et ma bouche annoncera ta louange. Dieu, viens à mon aide, Seigneur, viens vite à mon secours. Gloire au Père, au Fils, et au Saint-Esprit, comme il était au commencement, maintenant et toujours, dans les siècles des siècles. Amen. Alléluia. (\rubric{Septuagésime et Carême} Louange à toi, Seigneur, Roi d'éternelle gloire.)}

\pagebreak

\rubric{Aux féries, vigiles et fêtes à un nocturne:}
\gscore{ORIc}{T}{}{A@Domine labia mea!Tonus ferialis}

\vspace{\baselineskip}

\rubric{L'ouverture est omise dans l'Office des Défunts, au Triduum Pascal, et à l'Épiphanie.}

\newpage

\intermediatetitle{Invitatoire}

\rubric{Le chantre chante l'invitatoire au Propre ou au Commun, et tous le répètent. Il est chanté après les versets impairs du psaume qui suit, et sa deuxième partie après les versets pairs.}

\psalmus{94}{VLrepet}

\vspace{\baselineskip}

\rubric{L'invitatoire est omis au Triduum Pascal et à l'Épiphanie.}

\vfill

\intermediatetitle{Hymne}

\rubric{On chante l'hymne au Propre, au Commun, ou, pour les féries de l'année, au Psautier.}

\rubric{L'hymne est omis dans l'Office des Défunts, au Triduum Pascal, et à l'Épiphanie.}

\vfill

\rubric{Le dimanche et aux fêtes à trois nocturnes, on chante, à chacun des trois nocturnes, trois psaumes, précédés chacun d'une antienne que l'on répète à la fin du psaume après le Gloria Patri. Au temps pascal, les trois psaumes sont dits sous la même antienne, sans répétition de l'antienne, mais avec répétition du Gloria Patri. Aux féries, vigiles, et fêtes à un nocturne, on chante de la même manière neuf psaumes à l'unique nocturne. Après la psalmodie de chaque nocturne, on chante l'absolution, les bénédictions, les leçons et les répons, comme ci-après.}

\pagebreak

\intermediatetitle{Premier Nocturne}

\smalltitle{Absolution}
\gscore[n]{ORA}{T}{}{H@Absolutio}
\translation{Seigneur Jésus-Christ, exauce les prières de tes serviteurs, et aie pitié de nous,
toi qui vis et règnes avec le Père et le Saint-Esprit, dans les siècles des siècles. \rubric{\rrrub} Amen.}

\gscore[n]{ORLb}{T}{}{L@Benedictio}
\translation{Maître, veuillez bénir. \rubric{Bén.} Que le Père éternel nous bénisse d'une bénédiction perpétuelle.}
\gscore[n]{ORLc}{T}{}{}

\rubric{Les leçons sont au Propre ou, parfois, au Commun.}

\pagebreak

\rubric{À la fin des leçons:}

\gscore[n]{ORLd}{T}{}{T@In fine lectionum}
\translation{Et toi Seigneur, aie pitié de nous.}
\gscore[n]{ORLe}{T}{}{}

\rubric{Chaque leçon est suivie d'un répons, au Propre ou au Commun.}

\rubric{Bénédictions pour les leçons suivantes:}

\twocoltext{
	\rubric{\emph{Ben. 2.}} Unigénitus \textit{Dei} \textbf{Fí}lius~\GreSpecial{*}
	nos benedícere et adjuváre dignétur.
	\hspace{\specialcharhsep}\rr Amen.
}{
	\rubric{\emph{Bén. 2.}} Que le Fils unique de Dieu daigne nous bénir et nous secourir.
	\hspace{\specialcharhsep}\rr Amen.
}

\twocoltext{
	\rubric{\emph{Ben. 3.}} Spíritus \textit{Sancti} \textbf{grá}tia~\GreSpecial{*}
	illúminet sensus et corda nostra.
	\hspace{\specialcharhsep}\rr Amen.
}{
	\rubric{\emph{Bén. 3.}} Que la grâce du Saint-Esprit illumine nos esprits et nos cœurs.
	\hspace{\specialcharhsep}\rr Amen.
}

\intermediatetitle{Deuxième Nocturne}

\smalltitle{Absolution}

\twocoltext{
	\rubric{\emph{Absolutio 2.}}
	Ipsíus píetas et misericódi\textit{a nos} \textbf{ád}juvet,~\GreSpecial{*}
	qui cum Patre et Spíritu Sancto vivit et regnat in sǽcula sæculórum.
	\hspace{\specialcharhsep}\rr Amen.
}{
	\rubric{\emph{Absolution 2.}}
	Qu'il nous secoure par sa bonté et sa miséricorde, celui qui, 
	avec le Père et le Saint-Esprit, vit et règne dans les siècles des siècles.
	\hspace{\specialcharhsep}\rr Amen.
}

\smalltitle{Bénédictions}

\twocoltext{
	\rubric{\emph{Ben. 4.}} Deus Pa\textit{ter om}\textbf{ní}potens~\GreSpecial{*}
	sit nobis propítius et clemens.
	\hspace{\specialcharhsep}\rr Amen.
}{
	\rubric{\emph{Bén. 4.}}
	Que Dieu le Père tout-puissant soit pour nous propice et plein de clémence.
	\hspace{\specialcharhsep}\rr Amen.
}

\twocoltext{
	\rubric{\emph{Ben. 5.}} Chris\textit{tus per}\textbf{pé}tuæ~\GreSpecial{*}
	det nobis gaúdia vitæ.
	\hspace{\specialcharhsep}\rr Amen.
}{
	\rubric{\emph{Bén. 5.}}
	Que le Christ nous donne les joies de l'éternelle vie.
	\hspace{\specialcharhsep}\rr Amen.
}

\twocoltext{
	\rubric{\emph{Ben. 6.}} Ignem su\textit{i a}\textbf{mó}ris~\GreSpecial{*}
	accéndat Deus in córdibus nostris.
	\hspace{\specialcharhsep}\rr Amen.
}{
	\rubric{\emph{Bén. 6.}}
	Que Dieu daigne allumer dans nos cœurs le feu de son amour.
	\hspace{\specialcharhsep}\rr Amen.
}

\intermediatetitle{Troisième Nocturne}

\smalltitle{Absolution}

\twocoltext{
	\rubric{\emph{Absolutio 3.}}
	A vínculis peccató\textit{rum nos}\textbf{tró}rum~\GreSpecial{*}
	absólvat nos omnípotens et miséricors Dóminus.
	\hspace{\specialcharhsep}\rr Amen.
}{
	\rubric{\emph{Absolution 3.}}
	Que le Dieu tout-puissant et miséricordieux 
	daigne nous délivrer des liens de nos péchés.
	\hspace{\specialcharhsep}\rr Amen.
}

\smalltitle{Bénédictions}

\twocoltext{
	\rubric{\emph{Ben. 7.}}
	Evangé\textit{lica} \textbf{léc}tio~\GreSpecial{*}
	sit nobis salus et protéctio.
	\hspace{\specialcharhsep}\rr Amen.
}{
	\rubric{\emph{Bén. 7.}}
	Que la lecture du saint Évangile nous soit salut et protection.
	\hspace{\specialcharhsep}\rr Amen.
}

\rubric{Le dimanche et aux fêtes du Seigneur:}
\twocoltext{
	\rubric{\emph{Ben. 8.}}
	Diví\textit{num au}\textbf{xí}lium~\GreSpecial{*}
	máneat semper nobíscum.
	\hspace{\specialcharhsep}\rr Amen.
}{
	\rubric{\emph{Bén. 8.}}
	Que le secours divin demeure toujours avec nous.
	\hspace{\specialcharhsep}\rr Amen.
}

\rubric{Aux fêtes de la Sainte Vierge:}
\twocoltext{
	\rubric{\emph{Ben. 8.}}
	Cujus \textit{festum} \textbf{có}limus,~\GreSpecial{*}
	ipsa Virgo vírginum intercédat pro nobis ad Dóminum.
	\hspace{\specialcharhsep}\rr Amen.
}{
	\rubric{\emph{Bén. 8.}}
	Que celle dont nous célébrons la fête, la Vierge des vierges elle-même, 
	intercède pour nous auprès du Seigneur.
	\hspace{\specialcharhsep}\rr Amen.
}

\rubric{Aux fêtes des saints:}
\twocoltext{
	\rubric{\emph{Ben. 8.}}
	Cujus \rubric{(vel} Quarum\rubric{)} \textit{festum} \textbf{có}limus,~\GreSpecial{*}
	ipse \rubric{(vel} ipsa \rubric{aut} ipsæ\rubric{)}
	intercédat \rubric{(vel} intercédant\rubric{)} pro nobis ad Dóminum.
	\hspace{\specialcharhsep}\rr Amen.
}{
	\rubric{\emph{Bén. 8.}}
	Que celui \rubric{(ou \normaltext{celle}, \normaltext{ceux}, \normaltext{celles})} 
	dont nous célébrons la fête intercède\rubric{(}nt\rubric{)} pour nous 
	auprès du Seigneur.
	\hspace{\specialcharhsep}\rr Amen.
}

\vspace{\baselineskip}

\twocoltext{
	\rubric{\emph{Ben. 9.}}
	Ad societátem cívium \textit{super}\textbf{nó}rum~\GreSpecial{*}
	perdúcat nos Rex Angelórum.
	\hspace{\specialcharhsep}\rr Amen.
}{
	\rubric{\emph{Bén. 9.}}
	Que le Roi des Anges nous fasse parvenir à la société des citoyens célestes.
	\hspace{\specialcharhsep}\rr Amen.
}

\newpage

\rubric{Si la dernière leçon est une commémoraison de l'évangile du dimanche, de la férie ou de la vigile:}
\twocoltext{
	\rubric{\emph{Ben. 9.}}
	Per evangé\textit{lica} \textbf{dic}ta~\GreSpecial{*}
	deleántur nostra delícta.
	\hspace{\specialcharhsep}\rr Amen.
}{
	\rubric{\emph{Bén. 9.}}
	Par les paroles de l'Évangile, que nos péchés soient effacés.
	\hspace{\specialcharhsep}\rr Amen.
}

\intermediatetitle{À l'Office d'un seul Nocturne}

\smalltitle{Absolution}

\rubric{Lundi et jeudi comme au premier nocturne, mardi et vendredi comme au deuxième nocturne, mercredi et samedi comme au troisième nocturne.}

\smalltitle{Bénédictions}

\rubric{Aux féries où l'on lit une homélie, comme au troisième nocturne des dimanches.}

\rubric{Aux féries où l'on lit l'Écriture courante,
lundi et jeudi comme au premier nocturne, mardi et vendredi comme au deuxième nocturne, mercredi et samedi comme ci-dessous.}

\twocoltext{
	\rubric{\emph{Ben. 1.}}
	Ille nos \textit{bene}\textbf{dí}cat,~\GreSpecial{*}
	qui sine fine vivit et regnat.
	\hspace{\specialcharhsep}\rr Amen.
}{
	\rubric{\emph{Bén. 1.}}
	Qu'il nous bénisse, celui qui vit et règne sans fin.
	\hspace{\specialcharhsep}\rr Amen.
}

\twocoltext{
	\rubric{\emph{Ben. 2.}}
	Diví\textit{num au}\textbf{xí}lium~\GreSpecial{*}
	máneat semper nobíscum.
	\hspace{\specialcharhsep}\rr Amen.
}{
	\rubric{\emph{Bén. 2.}}
	Que le secours divin demeure toujours avec nous.
	\hspace{\specialcharhsep}\rr Amen.
}

\twocoltext{
	\rubric{\emph{Ben. 3.}}
	Ad societátem cívium \textit{super}\textbf{nó}rum~\GreSpecial{*}
	perdúcat nos Rex Angelórum.
	\hspace{\specialcharhsep}\rr Amen.
}{
	\rubric{\emph{Bén. 3.}}
	Que le Roi des Anges nous fasse parvenir à la société des citoyens célestes.
	\hspace{\specialcharhsep}\rr Amen.
}

\rubric{Aux fêtes des saints:}

\twocoltext{
	\rubric{\emph{Ben. 1.}}
	Ille nos \textit{bene}\textbf{dí}cat,~\GreSpecial{*}
	qui sine fine vivit et regnat.
	\hspace{\specialcharhsep}\rr Amen.
}{
	\rubric{\emph{Bén. 1.}}
	Qu'il nous bénisse, celui qui vit et règne sans fin.
	\hspace{\specialcharhsep}\rr Amen.
}

\twocoltext{
	\rubric{\emph{Ben. 2.}}
	Cujus \rubric{(vel} Quarum\rubric{)} \textit{festum} \textbf{có}limus,~\GreSpecial{*}
	ipse \rubric{(vel} ipsa \rubric{aut} ipsæ\rubric{)}
	intercédat \rubric{(vel} intercédant\rubric{)} pro nobis ad Dóminum.
	\hspace{\specialcharhsep}\rr Amen.
}{
	\rubric{\emph{Bén. 2.}}
	Que celui \rubric{(ou \normaltext{celle}, \normaltext{ceux}, \normaltext{celles})} 
	dont nous célébrons la fête intercède\rubric{(}nt\rubric{)} pour nous 
	auprès du Seigneur.
	\hspace{\specialcharhsep}\rr Amen.
}

\twocoltext{
	\rubric{\emph{Ben. 3.}}
	Ad societátem cívium \textit{super}\textbf{nó}rum~\GreSpecial{*}
	perdúcat nos Rex Angelórum.
	\hspace{\specialcharhsep}\rr Amen.
}{
	\rubric{\emph{Bén. 3.}}
	Que le Roi des Anges nous fasse parvenir à la société des citoyens célestes.
	\hspace{\specialcharhsep}\rr Amen.
}

\intermediatetitle{À l'Office de Sainte Marie le samedi}

\smalltitle{Absolution}

\twocoltext{
	\rubric{\emph{Absolutio.}}
	Précibus et méritis beátæ Maríæ semper Virginis
	et ómni\textit{um San}\textbf{ctó}rum,~\GreSpecial{*}
	perdúcat nos Dóminus ad regna cælórum.
	\hspace{\specialcharhsep}\rr Amen.
}{
	\rubric{\emph{Absolution.}}
	Que par les prières et les mérites de la bienheureuse Marie toujours Vierge 
	et de tous les saints, le Seigneur nous conduise au royaume des Cieux.
	\hspace{\specialcharhsep}\rr Amen.
}

\smalltitle{Bénédictions}

\twocoltext{
	\rubric{\emph{Ben. 1.}}
	Nos cum \textit{prole} \textbf{pi}a~\GreSpecial{*}
	benedícat Virgo María.
	\hspace{\specialcharhsep}\rr Amen.
}{
	\rubric{\emph{Bén. 1.}}
	Que la Vierge Marie nous obtienne la bénédiction de Son divin Fils.
	\hspace{\specialcharhsep}\rr Amen.
}

\twocoltext{
	\rubric{\emph{Ben. 2.}}
	Ipsa \textit{Virgo} \textbf{vír}ginum~\GreSpecial{*}
	intercédat pro nobis ad Dóminum.
	\hspace{\specialcharhsep}\rr Amen.
}{
	\rubric{\emph{Bén. 2.}}
	Que la Vierge des vierges elle-même, 
	intercède pour nous auprès du Seigneur.
	\hspace{\specialcharhsep}\rr Amen.
}

\twocoltext{
	\rubric{\emph{Ben. 3.}}
	Per Vír\textit{ginem} \textbf{ma}trem~\GreSpecial{*}
	concédat nobis Dóminus salútem et pacem.
	\hspace{\specialcharhsep}\rr Amen.
}{
	\rubric{\emph{Bén. 3.}}
	Par la Vierge Mère, donnez nous Seigneur, le salut et la paix.
	\hspace{\specialcharhsep}\rr Amen.
}

\newpage

\intermediatetitle{Te Deum}

\rubric{Le Te Deum remplace le dernier répons, le dimanche hors de l'Avent et du Carême, aux fêtes, et aux féries du Temps Pascal.}

\vspace{\baselineskip}

\rubric{On incline la tête aux mots \normaltext{Sanctus, Sanctus, Sanctus}, on s'incline profondément aux mots \normaltext{Tu ad liberándum}, on se met à genoux aux mots \normaltext{Te ergo quǽsumus}, et on incline la tête aux mots \normaltext{nomen tuum}.}

\rubric{Ton solennel:}
\gscore{ORTDa}{T}{}{W@Te Deum laudamus!Tonus solemnis}

\pagebreak

\rubric{Ton simple:}
\gscore{ORTDb}{T}{}{W@Te Deum laudamus!Tonus simplex}

\vspace{\baselineskip}

\translation{Nous te louons ô Dieu : nous te reconnaissons pour le Seigneur.\\
Ô Père éternel, toute la terre te révère.\\
Tous les Anges les Cieux, et toutes les Puissances,\\
Les Chérubins et les Séraphins te proclament sans cesse :\\
Saint, Saint, Saint le Seigneur, le Dieu des armées.\\
Les Cieux et la terre sont remplis de la majesté de ta gloire.\\
Le chœur glorieux des Apôtres,\\
La phalange vénérable des Prophètes,\\
l'armée des Martyrs éclatante de blancheur célèbre tes louanges;\\
La sainte Église confesse ton nom par toute la terre,\\
Ô Père d'infinie majesté!\\
Et elle vénère ton Fils véritable et unique,\\
Ainsi que le Saint-Esprit consolateur.\\
Tu es le Roi de gloire ô Christ!\\
Tu es du Père le Fils éternel.\\
Pour délivrer l'homme, tu n'as pas eu horreur du sein d'une Vierge.\\
Tu as brisé l'aiguillon de la mort\\
~\hfill et ouvert aux fidèles le royaume des cieux.\\
Tu es assis à la droite de Dieu dans la gloire du Père.\\
Nous croyons que tu es le juge qui doit venir.\\
Nous te supplions donc de secourir tes serviteurs\\
~\hfill que tu as rachetés par ton Sang précieux.\\
Fais qu'ils soient au nombre des saints, dans la gloire éternelle.\\
Sauve ton peuple, Seigneur  et bénis ton héritage.\\
Conduis tes serviteurs et élèves-les jusque dans l'éternité.\\
Chaque jour nous te bénissons.\\
Et nous louons ton nom dans les siècles; et dans les siècles des siècles.\\
Daigne Seigneur, en ce jour nous préserver de tout péché.\\
Aie pitié de nous Seigneur, aie pitié de nous.\\
Que ta miséricorde, Seigneur soit sur nous,\\
~\hfill comme notre espérance est en toi.\\
J'ai éspéré en toi Seigneur; que je ne sois pas confondu à jamais.}

\newpage

\intermediatetitle{Conclusion}

\rubric{La conclusion est omise si on chante les Laudes en suivant.}

{\grechangedim{spaceabovelines}{-2.5mm}{scalable}\grechangedim{spacebeneathtext}{2mm}{scalable}

\gscore[n]{ORDV}{T}{}{Dominus vobiscum}
\translation{Le Seigneur soit avec vous. \rubric{\rrrub} Et avec votre esprit.}

\vspace{0.5\baselineskip}
\rubric{Ton simple, aux féries et aux fêtes mineures:}
\gscore[n]{ORDVb}{T}{}{}


\vspace{\baselineskip}

\rubric{Si le célébrant n'est pas au moins diacre:}

\gscore[n]{ORDE}{T}{}{Domine exaudi}
\translation{Seigneur, entends ma prière. \rubric{\rrrub} Et que mon cri parvienne jusqu'à toi.}

\vspace{0.5\baselineskip}
\rubric{Ton simple, aux féries et aux fêtes mineures:}
\gscore[n]{ORDEb}{T}{}{}

}% end grechangedim spaceabovelines/spacebeneathtext

\vspace{\baselineskip}

\rubric{Oraison au Propre, à laquelle on répond \normaltext{Amen.}}

\pagebreak

\twocoltext{
	\versiculus{Dóminus vobíscum.}{Et cum spíritu tuo.}
	\rubric{vel \normaltext{Dómine, exáudi}, etc.}
}{
	\versiculus{Le Seigneur soit avec vous.}{Et avec votre esprit.}
	\rubric{ou \normaltext{Seigneur, entends}, etc.}
}

\rubric{\normaltext{Benedicámus Dómino} propre au jour, ci-après, puis:}

\twocoltext{
	\versiculus{Fidélium ánimæ per misericórdiam Dei requiéscant in pace.}{Amen.}
}{
	\versiculus{Que par la miséricorde de Dieu, les âmes des fidèles trépassés reposent en paix.}{Amen.}
}

\vspace{1\baselineskip}

\rubric{On reste en silence le temps d'un Notre Père.}

\vspace{2\baselineskip}

\sep

\vspace{2\baselineskip}

\intermediatetitle{Tons du Benedicamus}

\vspace{1\baselineskip}

\rubric{Aux fêtes solennelles}
\gscore{ORBDa}{T}{}{Z@Benedicamus Domino!In Festis Solemnibus}
\vfill
\rubric{Aux fêtes doubles}
\gscore{ORBDb}{T}{}{Z@Benedicamus Domino!In Festis Duplicibus}
\newpage
\rubric{Dans l'Octave de Pâques}
\gscore{ORBDj}{T}{}{Z@Benedicamus Domino!Per Octavam Paschae}
\vfill
\rubric{Au Temps Pascal}
\gscore{ORBDk}{T}{}{Z@Benedicamus Domino!Tempore Paschali}
\vfill
\rubric{Aux dimanches de l'Avent et du Carême}
\gscore{ORBDm}{T}{}{Z@Benedicamus Domino!In Dominicis Adventus et Quadragesimae}
\vfill
\rubric{Aux dimanches pendant l'année}
\gscore{ORBDe}{T}{}{Z@Benedicamus Domino!In Dominicis per Annum}
\pagebreak
\vfill
\rubric{Aux fêtes de la Sainte Vierge}
\gscore{ORBDd}{T}{}{Z@Benedicamus Domino!In Festis B.M.V.}
\vfill
\rubric{Aux fêtes semi-doubles}
\gscore{ORBDc}{T}{}{Z@Benedicamus Domino!In Festis Semiduplicibus}
\vfill
\rubric{Aux fêtes simples}
\gscore{ORBDf}{T}{}{Z@Benedicamus Domino!In Festis Simplicibus}
\vfill
\rubric{À l'Office de Sainte Marie le samedi}
\gscore{ORBDg}{T}{}{Z@Benedicamus Domino!In Officio B.M.V. in Sabbato\linebreak\null}
\vfill
\rubric{Aux féries, hors du Temps Pascal}
\gscore{ORBDi}{T}{}{Z@Benedicamus Domino!In Feriis}
\pagebreak

~\\

\thispagestyle{empty}

\vspace{3cm}

\intermediatetitle{Prière après l'Office}

\vspace{\baselineskip}

\twocoltext{
\vspace{\baselineskip}

\lettrine{S}{acrosánctæ} et indivíduæ Trinitáti, crucifíxi Dómini nostri Jesu Christi humanitáti, beatíssimæ et gloriosíssimæ sempérque Vírginis Maríæ fœcúndæ integritáti, et ómnium Sanctórum universitáti sit sempitérna laus, honor, virtus et glória ab omni creatúra, nobísque remíssio ómnium peccatórum, per infiníta sǽcula sæculórum. Amen.
}{
\lettrine{À}{ la} Très Sainte et indivisible Trinité, à l'humanité de Notre Seigneur Jésus-Christ crucifié et à la féconde intégrité de la Bienheureuse et très glorieuse Marie toujours Vierge, ainsi qu'à toute l'assemblée des Saints, soient éternelle louange, honneur, puissance et gloire de la part de toute créature, et à nous rémission de tous nos péchés, pour l'infinie durée des siècles et des siècles. Amen.
}

\end{document}

 