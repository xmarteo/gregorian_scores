\documentclass[twoside]{book}

\usepackage[paperwidth=148mm, paperheight=210mm]{geometry}
\usepackage{fontspec}
%\usepackage[latin1]{inputenc}
\usepackage[medievallatin, french]{babel}
\usepackage[strict]{changepage}
\usepackage{fancyhdr}
\usepackage{paracol}
\usepackage{tableof}
\usepackage{setspace}
\usepackage{alltt}
\usepackage{titlesec}
\usepackage{xcolor}
\usepackage{xstring}
\usepackage{enumitem}

%%%%%%%%%%%%%%%%%%%%%%%%%%%%%%%%%%%%%%%%%%%%%%%%%%% Mise ne page %%%%%%%%%%%%%%%%%%%%%%%%%%%%%%
% on numérote les nbp par page et non globalement
\usepackage[perpage]{footmisc}

% définition des en-têtes et pieds de page
\pagestyle{empty}
\fancyhead{}
\fancyfoot{}
\renewcommand{\headrulewidth}{0pt}
\setlength{\headheight}{10pt}
\fancyhead[RO]{\small\thepage}
\fancyhead[LE]{\small\thepage}
% la commande titres permet de changer les titres de gauche et de droite.
\newcommand{\titres}[2]{
	\renewcommand{\rightmark}{\textcolor{red}{\sc #2}}
	\renewcommand{\leftmark}{\textcolor{red}{\sc #1}}
}
\titres{}{}

% pas d'indentation
\setlength{\parindent}{0mm}

\geometry{
inner=25mm,
outer=12mm,
top=15mm,
bottom=15mm,
headsep=3mm,
}

%%%%%%%%%%%%%%%%%%%%%%%%%%%%%%%%%%%%%%%%%%%%%%%%% Options gregorio %%%%%%%%%%%%%%%%%%%%%%%%%

\usepackage[forcecompile]{gregoriotex}
%\usepackage{gregoriotex}

%% style général de gregorio :
% lignes rouges, commenter pour du noir
\gresetlinecolor{gregoriocolor}

% texte <alt> (au-dessus de la portée) en rouge et en petit, avec réglage de sa position verticale
\grechangestyle{abovelinestext}{\color{gregoriocolor}\footnotesize}
\newcommand{\altraise}{-2.4mm}
\grechangedim{abovelinestextraise}{\altraise}{scalable}

% taille des initiales
\newcommand{\initialsize}[1]{
    \grechangestyle{initial}{\fontspec{Zallman}\fontsize{#1}{#1}\selectfont}
}
\newcommand{\defaultinitialsize}{32}
\initialsize{\defaultinitialsize}
% espace avant et après les initiales
\newcommand{\initialspace}[1]{
  \grechangedim{afterinitialshift}{#1}{scalable}
  \grechangedim{beforeinitialshift}{#1}{scalable}
}
\newcommand{\defaultinitialspace}{0cm}
\initialspace{\defaultinitialspace}


% on définit le système qui capture des headers pour générer des annotations
% cette commande sera appelée pour définir des abréviations ou autres substitutions
\newcommand{\resultat}{}
\newcommand{\abbrev}[3]{
  \IfSubStr{#1}{#2}{ \renewcommand{\resultat}{#3} }{}
}
\newcommand{\officepartannotation}[1]{
  \renewcommand{\resultat}{#1}
  \abbrev{#1}{ntro}{ {Intr.} }
  \abbrev{#1}{espo}{Resp.}
  \abbrev{#1}{ll}{All.}
  \abbrev{#1}{act}{Tract.}
  \abbrev{#1}{equen}{Seq.}
  \abbrev{#1}{ffert}{Off.}
  \abbrev{#1}{ommun}{Co.}
  \abbrev{#1}{ntip}{Ant.}
  \abbrev{#1}{ntic}{Cant.}
  \abbrev{#1}{Toni Communes}{}
  \abbrev{#1}{yrial}{}
  \abbrev{#1}{salm}{}
  \abbrev{#1}{mnus}{}
  \greannotation{\resultat}
}
\newcommand{\modeannotation}[1]{
  \renewcommand{\resultat}{#1}
  \abbrev{#1}{1}{ {\sc i} }
  \abbrev{#1}{2}{ {\sc ii} }
  \abbrev{#1}{3}{ {\sc iii} }
  \abbrev{#1}{4}{ {\sc iv} }
  \abbrev{#1}{5}{ {\sc v} }
  \abbrev{#1}{6}{ {\sc vi} }
  \abbrev{#1}{7}{ {\sc vii} }
  \abbrev{#1}{8}{ {\sc viii} }
  \greannotation{\resultat}
}
\gresetheadercapture{office-part}{officepartannotation}{}
\gresetheadercapture{mode}{modeannotation}{string}

%%%%%%%%%%%%%%%%%%%%%%%%%%%%%%%%%%%%%%%%%%%%%% Graphisme %%%%%%%%%%%%%%%%%%%%%%%%%%%
% on définit l'échelle générale

\newcommand{\echelle}{0.85}

% on centre les titres et on ne les numérote pas
\titleformat{\section}[block]{\Large\filcenter\sc}{}{}{}
\titleformat{\subsection}[block]{\large\filcenter\sc}{}{}{}
\titleformat{\paragraph}[block]{\filcenter\sc}{}{}{}
\setcounter{secnumdepth}{0}
% on diminue l'espace avant les titres
\titlespacing*{\paragraph}{0pt}{1ex}{.6ex}

% commandes versets, repons et croix
\newcommand{\vv}{\textcolor{red}{\fontspec[Scale=\echelle]{Charis SIL}℣.\hspace{3mm}}}
\newcommand{\rr}{\textcolor{red}{\fontspec[Scale=\echelle]{Charis SIL}℟.\hspace{3mm}}}
\newcommand{\cc}{\textcolor{red}{\fontspec[Scale=\echelle]{FreeSerif}\symbol{"2720}~}}
\renewcommand{\aa}{\textcolor{red}{\fontspec[Scale=\echelle]{Charis SIL}\Abar.\hspace{3mm}}}

% commandes diverses
\newcommand{\antiphona}{\textcolor{red}{\noindent Antiphona.\hspace{4mm}}}
\newcommand{\antienne}{\textcolor{red}{\noindent Antienne.\hspace{4mm}}}
\newcommand{\rubrique}[1]{\textcolor{red}{\emph{#1}}}
\newcommand{\saut}{\\ \null \hspace{1cm}}
\newcommand{\minisaut}{\\ \null \hspace{4mm}}
\newcommand{\sautRV}{\\ \null \hspace{5.95mm}}
\newcommand{\petitvspace}{\vspace{2mm}}
\newcommand{\microvspace}{\vspace{0.8mm}}
% pour affichier 1 en rouge et un peu d'espace
\newcommand{\un}{{\color{gregoriocolor} 1~~~}}

% abréviations
\newcommand{\tpalleluia}{\rubrique{(T.P.} \mbox{Allelúia.\rubrique{)}}}
\newcommand{\tpalleluiafr}{\rubrique{(T.P.} \mbox{Alléluia.\rubrique{)}}}

\newcommand{\tqomittitur}{{\small \rubrique{(In Tempore Quadragesimæ ommittitur} Allelúia.\rubrique{)}}}
\newcommand{\careme}{{\small \rubrique{(Pendant le Carême on omet l'}Alléluia.\rubrique{)}}}

% environnement hymne : alltt + normalfont + marges custom
\newenvironment{hymne}
  {
  \begin{adjustwidth}{1.6cm}{1mm}
  \begin{alltt}\normalfont
  }
  {
  \end{alltt}
  \end{adjustwidth}
  }
  
% la commande \u permet de souligner les inflexions
\let\u\underline

% on définit la police par défaut
\setmainfont[Ligatures=TeX, Scale=\echelle]{Charis SIL}
%renderer=ICU a l'air de ne plus marcher...
%\setmainfont[Renderer=ICU, Ligatures=TeX, Scale=\echelle]{Charis SIL}
\setstretch{0.9}

% paramétrage de paracol en mode 2 colonnes par page : taille des colonnes, séparateur
\columnratio{0.5}
\setlength{\columnsep}{1.5em}
\setlength{\columnseprule}{0.3pt}

\begin{document}

% ceci est pour conserver une numérotation ordinaire malgré paracol
\twosided[pb]

\begin{titlepage}
\centering\null

\vspace{1cm}

{\scshape\LARGE Dominica in Sexagesima}

\vspace{2cm}

{\scshape\Large Ad Matutinum --- Responsoria et Lectiones}

\vspace{5cm}

{\scshape\LARGE Dimanche de la Sexagésime}

\vspace{2cm}

{\scshape\Large À Matines --- Répons et Lectures}


\end{titlepage}

\null\newpage

% tolérance infinie sur les sauts de lignes pour les colonnes étroites
\sloppy

\section{Premier nocturne}

\subsection{Première leçon}

\begin{paracol}{2}
\vv Jube, domne, benedícere. \\
\vv Benedictióne perpétua benedícat nos Pater ætérnus.\\
\rr Amen.\\
\rubrique{Si le célébrant n'est pas au moins diacre, le lecteur dit} Dómine \rubrique{au lieu de} domne\rubrique{.}
\switchcolumn
\vv Veuillez, Seigneur, bénir. \\
\vv Que le Père éternel nous bénisse d'une bénédiction perpétuelle. \\
\rr Ainsi soit-il.
\end{paracol}

\paragraph{Lectio liber Génesis}

Noé était âgé de cinq cents ans quand il engendra Sem, Cham et Japhet.
Quand les hommes commencèrent à se multiplier sur la terre et qu’ils eurent des filles,
les fils des dieux s’aperçurent que les filles des hommes étaient belles. Ils prirent pour eux des femmes parmi toutes celles qu’ils avaient distinguées.
Alors le Seigneur dit : « Mon souffle n’habitera pas indéfiniment dans l’homme : celui-ci s’égare, il n’est qu’un être de chair, sa vie ne durera que cent vingt ans. »
En ces jours-là, et même plus tard, il y avait des géants sur la terre. Les fils des dieux s’approchaient des filles des hommes et elles en avaient des enfants : ce sont les héros du temps jadis, des hommes de renom.
\begin{paracol}{2}
\vv Tu autem, Dómine, miserére nobis. \\
\rr Deo grátias.
\switchcolumn
\vv Et vous Seigneur, ayez pitié de nous. \\
\rr Nous rendons grâces à Dieu.
\end{paracol}


\gregorioscore{partitions/r1.gabc}

\emph{\rr Le Seigneur dit à Noé : La fin de toute chair est décidée devant moi ; la terre a été remplie de leur iniquité,
* Et je les perdrai avec la terre.\\
\vv Fais-toi une arche de pièces de bois polies, tu feras dedans des compartiments.}

\subsection{Deuxième leçon}

\begin{paracol}{2}
\vv Jube, domne, benedícere. \\
\vv Unigénitus Dei Fílius nos benedícere et adjuváre dignétur.\\
\rr Amen.
\switchcolumn
\vv Veuillez, Seigneur, bénir. \\
\vv Que le Fils unique de Dieu daigne nous bénir et nous secourir.\\
\rr Ainsi soit-il.
\end{paracol}

Le Seigneur vit que la méchanceté de l’homme était grande sur la terre, et que toutes les pensées de son cœur se portaient uniquement vers le mal à longueur de journée.
Le Seigneur se repentit d’avoir fait l’homme sur la terre ; il s’irrita en son cœur et il dit :
« Je vais effacer de la surface du sol les hommes que j’ai créés – et non seulement les hommes mais aussi les bestiaux, les bestioles et les oiseaux du ciel – car je me repens de les avoir faits. »
Mais Noé trouva grâce aux yeux du Seigneur.

\begin{paracol}{2}
\vv Tu autem, Dómine, miserére nobis. \\
\rr Deo grátias.
\switchcolumn
\vv Et vous Seigneur, ayez pitié de nous. \\
\rr Nous rendons grâces à Dieu.
\end{paracol}

\gregorioscore{partitions/r2.gabc}

\emph{\rr Noé, homme juste et intègre, a marché avec Dieu ;
* Et il a fait tout ce que Dieu lui a commandé.\\
\vv Il s’est fait une arche, pour sauver toute semence de vie.}

\newpage 

\subsection{Troisième leçon}

\begin{paracol}{2}
\vv Jube, domne, benedícere. \\
\vv Spíritus Sancti grátia illúminet sensus et corda nostra.\\
\rr Amen.
\switchcolumn
\vv Veuillez, Seigneur, bénir. \\
\vv Que la grâce du Saint-Esprit illumine nos esprits et nos cœurs.\\
\rr Ainsi soit-il.
\end{paracol}

Voici l’histoire de Noé. Parmi ses contemporains, Noé fut un homme juste, parfait. Noé marchait avec Dieu.
Il engendra trois fils : Sem, Cham et Japhet.
Mais la terre s’était corrompue devant la face de Dieu, la terre était remplie de violence.
Dieu regarda la terre, et voici qu’elle était corrompue car, sur la terre, tout être de chair avait une conduite corrompue.
Dieu dit à Noé : « Je l’ai décidé, c’est la fin de tout être de chair ! À cause des hommes, la terre est remplie de violence. Eh bien ! je vais les détruire et la terre avec eux.
Fais-toi une arche en bois de cyprès. Tu la diviseras en cellules et tu l’enduiras de bitume à l’intérieur et à l’extérieur.
Tu la feras ainsi : trois cents coudées de long, cinquante de large et trente de haut.

\begin{paracol}{2}
\vv Tu autem, Dómine, miserére nobis. \\
\rr Deo grátias.
\switchcolumn
\vv Et vous Seigneur, ayez pitié de nous. \\
\rr Nous rendons grâces à Dieu.
\end{paracol}

\gregorioscore{partitions/r3.gabc}

\emph{\rr Durant quarante jours et quarante nuits, les cataractes du ciel furent ouvertes, et de toute chair ayant souffle de vie, il entra dans l’arche ;
* Et le Seigneur ferma la porte, du dehors. \\
\vv En ce jour même, Noé entra dans l’arche, et ses fils, et son épouse, et les épouses de ses fils.}

\section{Deuxième nocturne}

\subsection{Quatrième leçon}

\begin{paracol}{2}
\vv Jube, domne, benedícere. \\
\vv Deus Pater omnípotens sit nobis propítius et clemens. \\
\rr Amen.
\switchcolumn
\vv Veuillez, Seigneur, bénir. \\
\vv Que Dieu le Père tout-puissant soit pour nous propice et plein de clémence. \\
\rr Ainsi soit-il.
\end{paracol}

\paragraph{Ex libro sancti Ambrósii Epíscopi de Noë et arca}

On nous dit que le Seigneur fut irrité, car bien que Dieu eût pensé, c’est- à-dire eût su que l’homme, placé en ce bas monde et chargé du poids de la chair, ne peut être sans péché (car la terre est comme un lieu de tentation et la chair comme un appât de corruption), cependant les hommes ont un esprit doué de raison et une force d’âme infuse à leur corps, et ils avaient dû écarter toute réflexion, pour se précipiter dans une déchéance d’où ils ne voulaient pas revenir. Dieu ne pense point à la manière des hommes, en sorte qu’un sentiment nouveau puisse succéder pour lui à une opinion précédente, il ne s’irrite pas non plus comme s’il était sujet au changement ; mais ces expressions se trouvent dans l’Écriture afin de marquer la malice de nos péchés, qui a mérité la disgrâce divine. C’est comme si l’écrivain sacré nous disait que nos fautes sont montées jusqu’à un tel excès qu’elles ont même paru provoquer Dieu à la colère, tout incapable qu’il soit, par sa nature, d’être ému de colère, de haine ou de quelque autre passion.

\begin{paracol}{2}
\vv Tu autem, Dómine, miserére nobis. \\
\rr Deo grátias.
\switchcolumn
\vv Et vous Seigneur, ayez pitié de nous. \\
\rr Nous rendons grâces à Dieu.
\end{paracol}



\gregorioscore{partitions/r4.gabc}

\emph{\rr Noé édifia un autel au Seigneur, y offrant un holocauste dont Todeur fut agréable au Seigneur ; il bénit Noé disant :
* Croissez et multipliez-vous et remplissez la terre. \\
\vv Voici que j’établirai mon alliance avec vous et avec votre race après vous.}


\subsection{Cinquième leçon}

\begin{paracol}{2}
\vv Jube, domne, benedícere. \\
\vv Christus perpétuæ det nobis gáudia vitæ.
\rr Amen.
\switchcolumn
\vv Veuillez, Seigneur, bénir. \\
\vv Que le Christ nous donne les joies de l'éternelle vie.
\rr Ainsi soit-il.
\end{paracol}

De plus, Dieu menaee d’exterminer l’homme : « J’exterminerai, dit-il, depuis l’homme jusqu’aux animaux, depuis le reptile jusqu’aux oiseaux du ciel. » En quoi les créatures dépourvues de raison avaient- elles offensé Dieu ? Elles n’avaient point péché, mais comme elles étaient faites pourl 'homme, il était logique que leur destruction suivît celle de l’hcmme à cause de qui elles avaient été créées, du moment que celui-ci n’existerait plu3 pour s’en servir. Dans un sens plus élevé, ce passage nous manifeste ceci, que l’homme possède une intelligence capable de raison. L’homme se définit en effet un animal vivant, mortel et raisonnable. Quand ce qu’il y a de meilleur en l’homme vient à s’éteindre en lui, le sens s’éteint aussi ; il n’y a plus rien en lui à sauver, lorsque le fondement du salut, qui est la vertu, fait défaut.

\begin{paracol}{2}
\vv Tu autem, Dómine, miserére nobis. \\
\rr Deo grátias.
\switchcolumn
\vv Et vous Seigneur, ayez pitié de nous. \\
\rr Nous rendons grâces à Dieu.
\end{paracol}

\newpage

\gregorioscore{partitions/r5.gabc}

~

\emph{\rr Je poserai mon arc dans les nuées du ciel, dit le Seigneur à Noé ;
* Et je me souviendrai de l’alliance que j’ai conclue avec toi. \\
\vv Et quand je couvrirai le ciel de nuées, mon arc apparaîtra dans les nuées.}

\subsection{Sixième leçon}

\begin{paracol}{2}
\vv Jube, domne, benedícere. \\
\vv Ignem sui amóris accéndat Deus in córdibus nostris. \\
\rr Amen.
\switchcolumn
\vv Veuillez, Seigneur, bénir. \\
\vv Que Dieu daigne allumer dans nos cœurs le feu de son amour.\\
\rr Ainsi soit-il.
\end{paracol}

C'est pour condamner les autres hommes et nous manifester la bonté divine, que l’Ecriture nous dit que Noé a trouvé grâce devant Dieu. On voit en même temps que l’homme juste n’est point noirci par les crimes des pécheurs, puisque Noé, loin de périr, est réservé pour être le père de toute une race. Il est loué, non pas à cause de la noblesse de sa naissance, mais à raison du mérite de sa justice et de sa sainteté. Ce qui fait la race de l’homme de bien, c’est la noblesse de ses vertus. Les familles humaines sont ennoblies par la noblesse de leur race, celle des âmes leur vient de leurs vertus. Une famille est illustre par la splendeur de sa race ; c’est l’éclat des vertus qui, de sa grâce, illustre les âmes.

\begin{paracol}{2}
\vv Tu autem, Dómine, miserére nobis. \\
\rr Deo grátias.
\switchcolumn
\vv Et vous Seigneur, ayez pitié de nous. \\
\rr Nous rendons grâces à Dieu.
\end{paracol}

\newpage

\gregorioscore{partitions/r6.gabc}

\emph{\rr Par moi-même je l’ai juré, dit le Seigneur, je n’amènerai plus les eaux du déluge sur la terre ; je me souviendrai de mon alliance,
* Pour ne plus perdre toute chair par les eaux du déluge. \\
\vv Je mettrai mon arc dans les nuées et il sera le signe de mon alliance entre moi et la terre.}

\newpage 

\section{Troisième nocturne}

\subsection{Septième leçon}

\begin{paracol}{2}
\vv Jube, domne, benedícere. \\
\vv Evangélica léctio sit nobis salus et protéctio.
\rr Amen.
\switchcolumn
\vv Veuillez, Seigneur, bénir. \\
\vv Que la lecture du saint Evangile nous soit salut et protection. 
\rr Ainsi soit-il.
\end{paracol}

\paragraph{Léctio sancti Evangélii secúndum Lucam}

Comme une grande foule se rassemblait, et que de chaque ville on venait vers Jésus, il dit dans une parabole :
« Le semeur sortit pour semer la semence, et comme il semait, il en tomba au bord du chemin.

Et réliqua.

\paragraph{Homilía sancti Gregórii Papæ} \rubrique{Homil. 19 in Evangelium post princ.}

La lecture du saint Évangile que vous avez entendue tout à l’heure, frères bien-aimés, n'a pas besoin d'être expliquée, mais d’être signalée à votre attention. Ce que la Vérité elle- même a exposé, la faiblesse humaine ne doit pas avoir la présomption de le discuter. Mais il y a dans l’explication que le Seigneur nous a donnée, une chose que nous devons considérer attentivement. Si nous vous disions que la semence signifie la parole de Dieu ; Je champ, le monde ; les oiseaux, les démons ; les épines, les richesses ; votre esprit hésiterait peut-être à nous croire. C’est pourquoi le Seigneur a daigné expliquer lui-même ce qu’il disait, afin que vous appreniez à chercher ce que signifient les choses qu’il n’a pas voulu éclaircir lui- même.


\begin{paracol}{2}
\vv Tu autem, Dómine, miserére nobis. \\
\rr Deo grátias.
\switchcolumn
\vv Et vous Seigneur, ayez pitié de nous. \\
\rr Nous rendons grâces à Dieu.
\end{paracol}

\gregorioscore{partitions/r7.gabc}

\emph{\rr Dieu bénit Noé et ses fils et leur dit :
* Croissez et multipliez-vous et remplissez la terre.\\
\vv Voici que j’établirai mon alliance avec vous et avec votre race après vous.}

\subsection{Huitième leçon}

\begin{paracol}{2}
\vv Jube, domne, benedícere. \\
\vv Divínum auxílium máneat semper nobíscum.
\rr Amen.
\switchcolumn
\vv Veuillez, Seigneur, bénir. \\
\vv Que l'assistance divine soit toujours avec nous.
\rr Ainsi soit-il.
\end{paracol}

En expliquant ce qu'il venait de dire, Jésus-Christ nous a fait connaître qu’il avait parlé dans un sens figuré, de manière à nous donner pleine assurance, quand notre faiblesse nous découvrirait les figures que renferment ses paroles. Qui, en effet, m’eût jamais cru, si j’eusse dit de moi-même que les épines signifient les richesses ; d’autant que celles-là piquent tandis que celles-ci délectent ? Et néanmoins les richesses sont des épines, car elles déchirent l’esprit par les piqûres des soucis qu’elles donnent; et lorsqu’elles nous entraînent jusqu’au péché, elles nous font pour ainsi dire une blessure sanglante. Aussi est-ce justement qu’en cet endroit, selon le témoignage d’un autre Évangéliste, le Seigneur ne les appelle pas simplement richesses, mais richesses fallacieuses.


\begin{paracol}{2}
\vv Tu autem, Dómine, miserére nobis. \\
\rr Deo grátias.
\switchcolumn
\vv Et vous Seigneur, ayez pitié de nous. \\
\rr Nous rendons grâces à Dieu.
\end{paracol}

\gregorioscore{partitions/r8.gabc}

\emph{\rr Voici que j’établirai mon alliance avec vous et avec votre race après vous ;
* Et il n’y aura plus de déluge ravageant la terre.\\
\vv Je mettrai mon arc dans les nuées, et il sera signe de l’alliance entre moi et la terre.}

\subsection{Neuvième leçon}

\begin{paracol}{2}
\vv Jube, domne, benedícere. \\
\vv Ad societátem cívium supernórum perdúcat nos Rex Angelórum.
\rr Amen.
\switchcolumn
\vv Veuillez, Seigneur, bénir. \\
\vv Que le Roi des Anges nous fasse parvenir à la société des citoyens célestes.
\rr Ainsi soit-il.
\end{paracol}

Elles sont fallacieuses, puisqu’elles ne peuvent demeurer longtemps ; elles sont fallacieuses, puisqu’elles ne délivrent pas notre âme de son indigence. Or, il n’y a de véritables, parmi les richesses, que celles qui nous rendent riches de vertus. Si donc, mes très chers frères, vous désirez être riches, aimez les vraies richesses. Si vous cherchez à parvenir au faîte du véritable bonheur, tendez au royaume céleste. Si vous aimez la gloire et les dignités, travaillez sans délai à être admis dans la suprême assemblée des Anges. Gravez en votre esprit les paroles du Seigneur que vous entendez. Car la parole de Dieu est la nourriture de l’âme. Si elle ne demeure pas dans notre mémoire, quand nous 1’avons entendue, c’est comme l’aliment que rejette un estomac malade ; et certes l’on désespère de voir durer la vie de celui qui ne peut conserver aucune nourriture.


\begin{paracol}{2}
\vv Tu autem, Dómine, miserére nobis. \\
\rr Deo grátias.
\switchcolumn
\vv Et vous Seigneur, ayez pitié de nous. \\
\rr Nous rendons grâces à Dieu.
\end{paracol}

\gregorioscore{partitions/r9.gabc}



\emph{\rr Comme une foule nombreuse accourait à Jésus et, des villes, se hâtait vers lui, il dit en parabole :
* Un semeur sortit pour semer sa semence.\\
\vv Et tandis qu’il semait, une partie tomba en bonne terre et, ayant levé, fructifia au centuple.}

~

\begin{paracol}{2}

\vv Orémus. Deus, qui cónspicis, quia ex nulla nostra actióne confídimus: concéde propítius; ut, contra advérsa ómnia Doctóris géntium protectióne muniámur. Per Dóminum nostrum Jesum Christum Fílium tuum, qui tecum vivit et regnat in unitáte Spíritus Sancti, Deus, per ómnia sǽcula sæculórum.
\rr Amen.
\switchcolumn
\vv Prions. O Dieu, qui voyez que nous ne mettons notre confiance en aucune action qui soit nôtre, accordez-nous la faveur d’être défendus contre tout ce qui nous est contraire, par la protection du Docteur des nations. Par Jésus-Christ, votre Fils, notre Seigneur, qui vit et règne avec vous et l'Esprit Saint, un seul Dieu, pour les siècles des siècles.
\rr Amen.

\end{paracol}


\end{document} 
