\documentclass[10pt, twoside, french]{article}


%%%%%%%%%%%% GEOMETRY
\usepackage{geometry}
\usepackage{fancyhdr}
\geometry{
	paperwidth=148mm,
	paperheight=210mm,
	inner=20mm,
	outer=12mm,
	top=15mm,
	bottom=12mm,
	headsep=2mm,
}
\pagestyle{empty}

%%%%%%%%%%%% LANGUAGE
\usepackage[nolocalmarks]{polyglossia}
\setdefaultlanguage[variant=french, frenchitemlabels=false]{french}

%%%%%%%%%%%% FONTS AND BASE STYLES
\usepackage{fontspec}
\setmainfont[Ligatures=TeX, Scale=1]{Charis}
\usepackage{paracol}
\usepackage[forcecompile]{gregoriotex}

%% No paragraph indentation
\setlength{\parindent}{0mm}

%% Macro to print rubrics
\newcommand{\rubric}[1]{\textcolor{gregoriocolor}{\emph{#1}}}

%% Macros to print V/ R/ A/ * + symbols in various contexts
\newcommand{\specialcharhsep}{3mm} % space after invoking R/ or V/ or A/ outside rubrics
\newcommand{\vv}{%
	{%
		\fontspec[Scale=1]{Charis}%
		℣.~%
		\nolinebreak[4]%
	}%
}
\newcommand{\redvv}{%
	\textcolor{gregoriocolor}%
	\vv%
	\hspace{\specialcharhsep}%
	\nolinebreak[4]%
}
\newcommand{\aarub}{%
	{%
		\fontspec[Scale=1]{Charis}%
		\Abar.~%
		\nolinebreak[4]%
	}%
}
\newcommand{\redaa}{%
	\textcolor{gregoriocolor}%
	\aarub%
	\hspace{\specialcharhsep}%
	\nolinebreak[4]%
}
\newcommand{\rr}{%
	{%
		\fontspec[Scale=1]{Charis}%
		℟.~%
		\nolinebreak[4]%
	}%
}
\newcommand{\redrr}{%
	\textcolor{gregoriocolor}%
	\rr%
	\hspace{\specialcharhsep}%
	\nolinebreak[4]%
}
\newcommand{\cc}{
	\textcolor{gregoriocolor}{
		\normalfont
		\fontspec[Scale=1]{FreeSerif}
		\symbol{"2720}
	}
}

%% Same special characters, for in-score use (<sp>V/ R/ A/ +</sp>)
\gresetspecial{V/}{\textcolor{gregoriocolor}{\fontspec[Scale=1]{Charis}℣.~}}
\gresetspecial{R/}{\textcolor{gregoriocolor}{\fontspec[Scale=1]{Charis}℟.~}}
\gresetspecial{A/}{\textcolor{gregoriocolor}{\fontspec[Scale=1]{Charis}\Abar.~}}
\gresetspecial{+}{{\fontspec[Scale=1]{FreeSerif}†~}}
\gresetspecial{*}{\gresixstar}
\gresetspecial{cross}{\textcolor{gregoriocolor}{\fontspec[Scale=1]{FreeSerif}\symbol{"2720}}}
\gresetspecial{labiacross}{\textcolor{gregoriocolor}{+}}

%% the asterisk as found in the mediants of text-only psalms
\newcommand{\psstar}{\GreSpecial{*}}
\newcommand{\pscross}{\GreSpecial{+}}

%% Macro to print versicles in two languages
\newcommand{\versiculus}[4]{%
	\begin{paracol}{2}%
	\par\redvv #1 \\ \redrr #2\par%
	\switchcolumn%
	\par\redvv #3 \\ \redrr #4\par%
	\end{paracol}%
}

%% Macro to print capitulum
\newcommand{\capitulum}[3]{%
	\smalltitle{Capitule}
	\begin{paracol}{2}%
	\rubric{#1}
	#2\\%
	\gresetinitiallines{0}%
	\gabcsnippet{(c3) <sp>R/</sp> De(h)o(h) <b>grá</b>(f)ti(e)as.(ef..) (::)}%
	\gresetinitiallines{1}%
	\switchcolumn
	#3\\
	\redrr Nous rendons grâces à Dieu.
	\end{paracol}%
}

%% Macro to print oratio
\newcommand{\oratio}[2]{%
	\versiculus{Orémus.\\#1}{Amen.}{Prions.\\#2}{Amen.}
}

%%%%%%%%%%%% GREGORIO CONFIG

%% \officepartannotation converts a letter (IHARPT) into the office part to be printed as annotation,
%% storing the result into \result.
\newcommand{\result}{}
\newcommand{\lookup}[3]{%
  \IfSubStr{#2}{#1}{ \renewcommand{\result}{#3} }{}%
}%
\newcommand{\officepartannotation}[1]{%
  \renewcommand{\result}{#1}%
  \lookup{#1}{T}{}%
  \lookup{#1}{H}{Hymn.}%
  \lookup{#1}{A}{Ant.}%
  \lookup{#1}{P}{}%
  \lookup{#1}{R}{Resp.}%
  \lookup{#1}{I}{Invit.}%
  \result%
}%

%% header capture setup for the mode
\newcommand{\defaultannotationshift}{-2mm}
\newcommand{\modeannotation}[1]{\greannotation{\hspace{\defaultannotationshift}\hspace{1mm}#1}}
\gresetheadercapture{mode}{modeannotation}{string}

%% outputs a score without annotations or initial
\newcommand{\smallscore}[1]{
	\gresetinitiallines{0}
	\gregorioscore{nocturnale-romanum/gabc/#1}
	\gresetinitiallines{1}
}

%% outputs a score with annotations and initial
\newcommand{\gscore}[3]{
	\greannotation[c]{
		\hspace{-1.4mm}
		\hspace{\defaultannotationshift}
		\officepartannotation{#2}#3
	}
	\gregorioscore{nocturnale-romanum/gabc/#1}
}

%% outputs a hymn with translation
\usepackage{multicol}
\setlength\columnseprule{0.4pt}
\setlength{\multicolsep}{6pt plus 2pt minus 1.5pt}
\newcommand{\hymnus}[2]{
	\smalltitle{Hymne}
	\gscore{#1}{H}{}
	\begin{multicols}{2}%
	\translation{#2}%
	\end{multicols}%
}

%% Initial style
\grechangestyle{initial}{\fontspec{Zallman Caps}\fontsize{28}{28}\selectfont}


%%%%%%%%%%%% TRANSLATION STYLE
\newcommand{\translation}[1]{
	\emph{#1}
}

%%%%%%%%%%%% PSALMODY STYLE
\usepackage{enumitem}
\usepackage{needspace}
%% We want to allow large inter-words space 
%% to avoid overfull boxes in two-columns rubrics.
\sloppy

\newcommand{\parallelitems}[2]{
	\begin{paracol}{2}
	\begin{itemize}[
		label=\null, 
		leftmargin=0pt, 
		itemindent=10pt, 
		labelsep=0pt, 
		labelwidth=0pt, 
		rightmargin=0pt, 
		parsep=0pt, 
		itemsep=0pt,
		topsep=-2mm]
	\input{nocturnale-romanum/psalmi/#1_#2.tex}
	\end{itemize}
	\switchcolumn
	\begin{itemize}[
		label=\null, 
		leftmargin=0pt, 
		itemindent=10pt, 
		labelsep=0pt, 
		labelwidth=0pt, 
		rightmargin=0pt, 
		parsep=0pt, 
		itemsep=0pt,
		topsep=-2mm]
	\input{psalmi_fr/#1.tex}
	\end{itemize}
	\end{paracol}
}

\newcommand{\psalmus}[2]{
	\needspace{4\baselineskip}
	\smalltitle{Psaume #1}
	\parallelitems{#1}{#2}
}

\newcommand{\magnificat}[1]{
	\needspace{4\baselineskip}
	\smalltitle{Magnificat}
	\parallelitems{magn}{#1}
}

%%%%%%%%%%%% TITLE STYLES

\newcommand{\smalltitle}[1]{
  \vspace{0.3\baselineskip}
  \par{\centering\textbf{#1}\par}
  \vspace{0.3\baselineskip}
}

\newcommand{\largetitle}[1]{
  \par{\centering\Huge\textsc{#1}\par}
}

\newcommand{\intermediatetitle}[1]{
  \par{\centering\Large\textsc{#1}\par}
}


%%%%%%%%%%%% GRAPHICS

\newcommand{\sep}{{\centering\greseparator{3}{20}\par}}


\begin{document}

\thispagestyle{empty}

\largetitle{Dimanches de la Passion\\aux II\textsuperscript{ndes} Vêpres}

\vfill
{\centering\includegraphics[scale=0.7]{jeune.jpg}\par}
\vfill

\smallscore{DIA_festivus}{}{}
\translation{\vv Dieu \cc venez à mon aide. \rr Seigneur, hâtez-vous de me secourir. Gloire au Père, au Fils, et au Saint-Esprit, comme il était au commencement, maintenant et toujours, et dans les siècles des siècles. Ainsi soit-il. Louange à vous, Seigneur, Roi d'éternelle gloire.}

\pagebreak

\gscore{F1A1}{A}{1}
\translation{Le Seigneur a dit à mon Seigneur : * Asseyez-Vous à ma droite.}
\rubric{On ne répète pas le texte de l'antienne au début du psaume.}
\psalmus{109}{7}
\rubric{On répète l'antienne, et on fait ainsi après chaque psaume.}
\gscore{F1A2}{A}{2}
\translation{Les œuvres du Seigneur sont grandes, * proportionnées à toutes Ses volontés.}
\psalmus{110}{3}
\gscore{F1A3}{A}{3}
\translation{Celui qui craint le Seigneur * met ses délices dans Ses commandements.}
\psalmus{111}{4g}
\gscore{F1A4}{A}{4}
\translation{Que le Nom du Seigneur * soit béni dans tous les siècles.}
\psalmus{112}{7}
\gscore{F1A5}{A}{5}
\translation{Notre Dieu est * dans le Ciel : tout ce qu'Il a voulu, Il l'a fait.}
\psalmus{113}{p}

\newpage

\smalltitle{Capitule --- dimanche de la Passion}

\capitulum{He 9: 11-12}{Fratres: Christus assístens Póntifex futurórum bonórum, per ámplius et perféctius tabernáculum non manu factum, id est, non huius creatiónis: neque per sánguinem hircórum aut vitulórum, sed per próprium sánguinem introívit semel in Sancta, ætérna redemptióne invénta.}{ Frères : Le Christ ayant paru comme grand prêtre des biens à venir, c’est en passant par un tabernacle plus excellent et plus parfait, qui n’est pas construit de main d’homme, c’est-à-dire, qui n’appartient pas à cette création-ci et ce n’est pas avec le sang des boucs et des taureaux, mais avec son propre sang, qu’il est entré une fois pour toutes dans le saint des Saints, après avoir acquis une rédemption éternelle.}

\smalltitle{Capitule --- dimanche des Rameaux}

\capitulum{Phil. 2: 5-7}{Fratres: Hoc enim sentíte in vobis, quod et in Christo Iesu: qui, cum in forma Dei esset, non rapínam arbitrátus est esse se æquálem Deo: sed semetípsum exinanívit, formam servi accípiens, in similitúdinem hóminum factus, et hábitu invéntus ut homo.}{Frères : Ayez en vous les mêmes sentiments dont était animé le Christ Jésus : bien qu’il fût Dieu par nature, il n’a pas retenu avidement son égalité avec Dieu, mais il s’est anéanti lui-même en prenant la condition d’esclave, en devenant semblable aux hommes, à l’extérieur absolument comme un homme.}

\begin{paracol}{2}%
\gresetinitiallines{0}%
\gabcsnippet{(c3) <sp>R/</sp> De(h)o(h) <b>grá</b>(f)ti(e)as.(ef..) (::)}%
\gresetinitiallines{1}%
\switchcolumn
~\\~\\
\redrr Nous rendons grâces à Dieu.
\end{paracol}%


\hymnus{Q5H}{L'étendard du Roi s'avance,\\
Il resplendit, le mystère de la Croix,\\
Où la vie supporta la mort\\
Et par la mort donna la vie.\\
\\
C'est elle qui, blessée\\
Par le fer cruel de la lance,\\
Pour nous laver de nos souillures,\\
Laissa couler l'eau et le sang.\\
\\
Ainsi donc se trouve accompli\\
Ce que chanta David en un Psaume fidèle,\\
Quand il annonçait aux nations\\
Que Dieu régnerait par le bois.\\
\\
Arbre éclatant, resplendissant,\\
Orné de la pourpre du Roi,\\
Issu d'un lignage assez noble\\
Pour toucher des membres si saints !\\
\\
Heureux es-tu, car à ton bois\\
La rançon du monde a pendu,\\
Balance où fut pesé le corps\\
Qui ravit la proie des enfers !\\
\\
\rubric{La strophe suivante se dit à genoux}\\
Ô Croix, salut, unique espoir,\\
En ce jour triomphant,\\
Des fidèles augmente la grâce,\\
Des pécheurs efface les crimes.\\
\\
Source du salut, Trinité,\\
Que tout esprit Vous glorifie :\\
La Croix nous donne la victoire :\\
Ajoutez-y la récompense.\\
Ainsi soit-il.}

\smalltitle{Verset}
\versiculus{Éripe me, Dómine, ab hómine malo.}{A viro iníquo éripe me.}{Délivrez-moi, Seigneur, du malfaiteur.}{Sauvez-moi de l'impie.}


{\centering\includegraphics[scale=0.4]{ihs.jpg}\par}

\newpage

\smalltitle{Antienne à Magnificat --- dimanche de la Passion}

\gscore{Q5AM}{A}{}
\translation{Abraham, votre père, * a tressailli pour voir mon jour ; il l’a vu, et il s’est réjoui.}

\smalltitle{Magnificat du deuxième ton}
\smallscore{magn2}
\vspace{\baselineskip}
\magnificat{2}

\smalltitle{Antienne à Magnificat --- dimanche des Rameaux}

\gscore{Q6AM}{A}{}
\translation{Il est écrit : * Je frapperai le pasteur et les brebis du troupeau seront dispersées : mais après que je serai ressuscité, je vous précéderai en Galilée, c’est là que vous me verrez, dit le Seigneur.}

\smalltitle{Magnificat du huitième ton}
\smallscore{magn8}
\vspace{\baselineskip}
\magnificat{8}

\vspace{1cm}

\sep

\vspace{1cm}

\newpage

\versiculus{Dóminus vobíscum.}{Et cum spíritu tuo.}{Le Seigneur soit avec vous.}{Et avec votre esprit.}

\versiculus{Orémus.\\\rubric{Oraison page suivante, selon le dimanche, conclue par:}\\Per Dóminum nostrum Jesum Christum, Fílium tuum: qui tecum vivit et regnat in unitáte Spíritus Sancti, Deus, per ómnia sǽcula sæculórum.}{Amen.}{Prions.\\~\\~\\Par Notre Seigneur Jésus Christ, Votre Fils, qui vit et règne avec Vous et le Saint-Esprit, Dieu, maintenant et pour les siècles des siècles.}{Amen.}

\smalltitle{Oraison -- dimanche de la Passion}

\oratio{Quǽsumus, omnípotens Deus, famíliam tuam propítius réspice: ut, te largiénte, regátur in córpore; et, te servánte, custodiátur in mente. Per Dóminum.}{Nous vous en prions, Dieu tout-puissant, regardez vos enfants dans votre miséricorde ; accordez-leur votre grâce pour qu’ils soient gouvernés en leur corps, et veillez sur eux pour qu’ils soient gardés en leur âme.}

\smalltitle{Oraison -- dimanche des Rameaux}

\oratio{Omnípotens sempitérne Deus, qui humáno géneri, ad imitándum humilitátis exémplum, Salvatórem nostrum carnem súmere, et crucem subíre fecísti: concéde propítius; ut et patiéntiæ ipsíus habére documénta, et resurrectiónis consórtia mereámur. Per Dóminum.}{Dieu tout-puissant et éternel, qui avez voulu que notre Sauveur prît la chair humaine et supportât les tourments de la croix, afin de servir de modèle d’humilité au genre humain, accordez-nous, dans votre bonté, d’être, à son exemple, toujours courageux dans les épreuves et de mériter par là d’avoir part à sa résurrection.}

\smalltitle{Conclusion}

\versiculus{Dóminus vobíscum.}{Et cum spíritu tuo.}{Le Seigneur soit avec vous.}{Et avec votre esprit.}

\smallscore{ORBDVA}{}{}
\translation{\vv Bénissons le Seigneur. \rr Nous rendons grâces à Dieu.}
\vspace{\baselineskip}
\versiculus{Fidélium ánimæ per misericórdiam Dei requiéscant in pace.}{Amen.}{Que par la miséricorde de Dieu, les âmes des fidèles trépassés reposent en paix.}{Ainsi soit-il.}

\end{document}